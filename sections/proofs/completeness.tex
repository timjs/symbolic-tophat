% !TEX root=../../main.tex





%
% \begin{proof}[Proof of Theorem~\ref{thm:complete}]
%   \fixme{we cannot prove this, definition of drive function is incorrect}
% \end{proof}



\begin{proof}[Proof of Lemma~\ref{lem:completeEval}]
This lemma holds trivially; assume $\sigma''=\sigma$, then we have $t,\sigma\eval v,\sigma',\True$.
In other words; every concrete evaluation is also a valid symbolic execution.
\end{proof}



\begin{proof}[Proof of Lemma~\ref{lem:completeStride}]
This lemma holds trivially; assume $\sigma''=\sigma$, then we have $t,\sigma\stride t',\sigma',\True$.
In other words; every concrete evaluation is also a valid symbolic execution.
\end{proof}



\begin{proof}[Proof of Lemma~\ref{lem:completeNormalise}]
This lemma holds trivially; assume $\sigma''=\sigma$, then we have $t,\sigma\normalise t',\sigma',\True$.
In other words; every concrete evaluation is also a valid symbolic execution.
\end{proof}



\begin{proof}[Proof of Lemma~\ref{lem:completeHandle}]

  We prove Lemma~\ref{lem:completeHandle} by induction over $t$.\\

  \case{$t=\Edit v$}
  {One rule applies in this case, namely \userule{H-Change}\\
  Take $i=s$ where s is a free symbol in $\sigma$ and assume $\sigma''=\sigma$.
  Then by the Sym-H-Change rule,
  we know that a symbolic execution exists.
  When applying the substitution $[s\mapsto v']$,
  we get $\Edit s[s\mapsto v'] = \Edit v'$ and $\sigma[s\mapsto v']=\sigma$ since $s$ is free in $\sigma$.
  \fixme{what about H-Clear?????}
  }

  \case{$t=\Enter \tau$}
  {One rule applies in this case, namely \userule{H-Fill}\\
  Take $i=s$ where s is a free symbol in $\sigma$ and assume $\sigma''=\sigma$.
  Then by the Sym-H-Fill rule,
  we know that a symbolic execution exists.
  When applying the substitution $[s\mapsto v']$,
  we get $\Edit s[s\mapsto v'] = \Edit v'$ and $\sigma[s\mapsto v']=\sigma$ since $s$ is free in $\sigma$. }

  \case{$t=\Update l$}
  {One rule applies in this case, namely \userule{H-Update}\\
  Take $i=s$ where s is a free symbol in $\sigma$ and assume $\sigma''=\sigma$.
  Then by the Sym-H-Update rule,
  we know that a symbolic execution exists.
  When applying the substitution $[s\mapsto v']$,
  we get $\Edit s[s\mapsto v'] = \Edit v'$ and $\sigma[s\mapsto v']=\sigma$ since $s$ is free in $\sigma$. }

  \case{$t=t_1\Next e_2$}
  {Two rules apply in this case.
    \case{\userule{H-Next}}
    {
    Take $i=s$ where s is a free symbol in $\sigma$ and $t$, and assume $\sigma''=\sigma$.
    Then by the Sym-H-Next rule,
    we know that a symbolic execution exists.
    When applying the substitution $[s\mapsto \Continue]$,
    we get $t_2[s\mapsto \Continue] = t_2'$ and $\sigma'[s\mapsto \Continue]=\sigma'$ since $s$ is free in both $\sigma$ and $t$.
    }
    \case{\userule{H-PassNext}}
    {
    Take $i=s$ where s is a free symbol in $\sigma$ and $t$, and assume $\sigma''=\sigma$.
    Then by the Sym-H-PassNext rule,
    we know that a symbolic execution exists.
    When applying the substitution $[s\mapsto j]$, and by application of the induction hypothesis, we obtain the following.\\
    $t_1,\sigma\xrightarrow[]{j}t_1',\sigma'\implies \exists t_1'',\sigma''\handle{i}t_1''',\sigma'''\land t_1'''[s\mapsto j] = t_1'\land \sigma'''[s\mapsto j]=\sigma'$.\\
    From this we obtain $t_1'''\Next e_2[s\mapsto j] = t_1'\Next e_2$ and $\sigma'''[s\mapsto j]=\sigma'$.
    }
  }


  \case{$t=t_1\Then e_2$}{}
  \case{$t=e_1\Xor e_2$}{}
  \case{$t=t_1\Or t_2$}{}
  \case{$t=t_1\And t_2$}{}

\end{proof}



\begin{proof}[Proof of Theorem~\ref{thm:completeDrive}]
  \fixme{todo}
\end{proof}
