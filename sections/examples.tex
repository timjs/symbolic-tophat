% !TEX root=../main.tex


\section{Examples}
\label{sec:examples}

First we will introduce the task-oriented programming language \TOPHAT .
Then we will introduce two examples to illustrate the kind of properties we would like to prove.

\subsection{\TOPHAT}

\TOPHAT is a task-oriented programming language.
Its aim is to model real world collaboration.

Programs in \TOPHAT are called tasks.
The smallest elements of a task is called an editor.

Editors are the basic method for communicating with the outside world.
There are three different Editors.
\begin{description}
  \item[$\Edit v$] Valued editor. This editor holds a value of a certain type. A new value of that type can be given as input, or the editor can be cleared of a value.
  \item[$\Enter \tau$] Unvalued editor. This editor holds no value, and can receive a value of type $\tau$. It will then turn into a valued editor.
  \item[$\Update l$] Shared editor. This editor refers to a shared location $l$. Its observable value is the value stored at that location. It can receive a new value, this value will then be stored at location $l$.
\end{description}

Editors can be combined into tasks using combinators.
These combinators describe the way people collaborate.
The following combinators are available in \TOPHAT.

\begin{description}
  \item[$\Then$]
  \item[$\Next$]
  \item[$\And$]
  \item[$\Or$]
  \item[$\Xor$]
\end{description}


%additonally, we also have fail.

\subsection{Flightbooking}

\subsection{Tax subsidy request}
