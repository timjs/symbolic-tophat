% !TEX root=../main.tex



\section{Intuition}
\label{sec:intuition}

When a task has an observable value, we consider it done.
Tasks only have an observable value when they consist merely of editors or pairs of editors.
At this point, user interaction with the task can change its value, but it is impossible for the task to do anything meaningful with that new value.
This is the point where symbolic execution terminates.

The execution results in a list of all possible program results,
guarded by certain constraints on the symbolic input.
If these possible outcomes agree with the stated property,
it is guaranteed that his will hold for all paths.

\begin{example}
Consider the following example program.

\begin{TASK}
  enter Int >>= \ x. if x > 0 then edit x else fail
\end{TASK}

It asks the user to input a value of type $\Int$.
This value is then passed on to the right hand side.
If the value is larger than zero, as determined by the condition $x>0$,
an editor containing the entered value is returned.
Otherwise, the user is asked to input a different value, until the condition holds.

Imagine we would like to prove that no matter what value is given as input,
the only possible outcome is a value larger than zero.

The symbolic execution enters a symbol $s$ as input into the editor on the left.
The execution then arrives at a branching point.
In order to take the then-branch, the condition $s > 0$ needs to hold.
This branch will then result in $\Edit s$.
We can now observe the task value, $s$, which is guarded by $s > 0$.
The else branch applies if the condition does not hold.
But this does not yield a value, it instead asks for additional input.

So, from the symbolic execution we obtain that the only outcome of this program is $s > 0$.
From that outcome, we can conclude that no matter what input is given, the only result value possible has to be larger than zero.
\label{example:abs}
\end{example}
