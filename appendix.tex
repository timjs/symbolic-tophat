%% For double-blind review submission, w/o CCS and ACM Reference (max submission space)
%\documentclass[acmsmall,review,anonymous]{acmart}\settopmatter{printfolios=true,printccs=false,printacmref=false}
%% For double-blind review submission, w/ CCS and ACM Reference
% \documentclass[acmsmall,review,anonymous]{acmart}\settopmatter{printfolios=true}
%% For single-blind review submission, w/o CCS and ACM Reference (max submission space)
% \documentclass[acmsmall,review]{acmart}\settopmatter{printfolios=true,printccs=false,printacmref=false}
%% For single-blind review submission, w/ CCS and ACM Reference
% \documentclass[acmsmall,review]{acmart}\settopmatter{printfolios=true}
%% For author draft version, w/o CCS and ACM Reference (max submission space)
% \documentclass[sigconf]{acmart}\settopmatter{printfolios=true,printccs=false,printacmref=false}
%% For final camera-ready submission, w/ required CCS and ACM Reference
%\documentclass[acmsmall]{acmart}\settopmatter{}
\documentclass[sigconf]{acmart}


% !TEX root=../main.tex


%% Basics %%%%%%%%%%%%%%%%%%%%%%%%%%%%%%%%%%%%%%%%%%%%%%%%%%%%%%%%%%%%%%%%%%%%%%

%% Fixes %%

%\usepackage{underscore}


%% Fonts %%

\usepackage[utf8]{inputenc} % on ACM whitelist
% \usepackage[T1]{fontenc}
%%NOTE: T1 doesn't have the `Th` and `Qu` ligatures :-(
\usepackage[OT1]{fontenc} % on ACM whitelist

\usepackage{stmaryrd}  % on ACM whitelist
% \usepackage{mathtools}
% \usepackage{eurosym}

\usepackage{amsthm}  % on ACM whitelist %%NOTE: here because defines \openbox which will also be defined by newtxmath...
% \usepackage{xcolor}


% \usepackage{tgpagella}
% \usepackage{lucidabr}

% \usepackage{libertine}
% \usepackage[varqu]{zi4}
% \usepackage[libertine]{newtxmath}


%% Programming %%

\usepackage{xargs}
\usepackage{ifthen}  % on ACM whitelist


%% Layout %%

% \usepackage{microtype}
% \usepackage{xspace}


%% Additions %%%%%%%%%%%%%%%%%%%%%%%%%%%%%%%%%%%%%%%%%%%%%%%%%%%%%%%%%%%%%%%%%%%

%% Textual %%

% \usepackage{titlesec}
%\usepackage[inline]{enumitem}
%\usepackage{quoting}


%% Maths %%

\usepackage{amsmath}  % on ACM whitelist


%% Graphics %%

\usepackage{graphicx}  % on ACM whitelist
%\usepackage{xcolor}
%\usepackage{dblfloatfix}
%\usepackage[export]{adjustbox}
% \usepackage[xcolor]{mdframed}
%\usepackage{tikz}
%\usetikzlibrary{trees}

%% Tabulations %%

\usepackage{booktabs}  % on ACM whitelist
\usepackage{array}  % on ACM whitelist


%% Listings %%

\usepackage[final]{listings}   % on ACM whitelist


%% References & Bibliography %%

%\usepackage[capitalize]{cleveref}
\usepackage{natbib}   % on ACM whitelist
% \usepackage[natbibapa,nodoi]{apacite}

% !TEX root=../main.tex


%% Fixes %%

\frenchspacing

% \newlength{\hugeskipamount}
% \setlength{\hugeskipamount}  {1.2500\baselineskip plus 0.3750\baselineskip minus 0.3750\baselineskip}
% \setlength{\bigskipamount}   {0.7500\baselineskip plus 0.2500\baselineskip minus 0.2500\baselineskip}
% \setlength{\medskipamount}   {0.3750\baselineskip plus 0.1250\baselineskip minus 0.1250\baselineskip}
% \setlength{\smallskipamount} {0.1875\baselineskip plus 0.0625\baselineskip minus 0.0625\baselineskip}

% \widowpenalty=150
% \clubpenalty=150


%% Section spacing %%
%%NOTE: requires 'titlesec'

% \titlespacing*{\section}{0pt}{\hugeskipamount}{\bigskipamount}
% \titlespacing*{\subsection}{0pt}{\bigskipamount}{\medskipamount}
% \titlespacing*{\paragraph}{0pt}{\medskipamount}{1em}


%% Math spacing %%

\setlength{\abovedisplayskip}{\smallskipamount}
\setlength{\belowdisplayskip}{\smallskipamount}

% \setlength{\topsep}{\smallskipamount}



%% Float spacing %%

\setlength{\abovecaptionskip}{\medskipamount}
\setlength{\floatsep}        {\medskipamount}
\setlength{\textfloatsep}    {\bigskipamount}
\setlength{\intextsep}       {\bigskipamount}
\setlength{\dblfloatsep}     {\medskipamount}
\setlength{\dbltextfloatsep} {\bigskipamount}


%% Description spacing %%

%\setlist{noitemsep}
%\setlist[description]{leftmargin=\parindent}


%% List spacing %%
%% NOTE: requires `paralist`

% \setlength{\pltopsep}   {\medskipamount}
% \setlength{\plpartopsep}{\parskip}
% \setlength{\plitemsep}  {\parskip}
% \setlength{\plparsep}   {\parskip}


%% Listings spacing %%

% \lstset
%   {aboveskip=\smallskipamount
%   ,belowskip=\smallskipamount
%   }


%% Tabular strech %%
%% NOTE: requires `array`

% \renewmacro{arraystretch}
%   {1.1}


%% Quoting %%
%%NOTE: requires 'quoting'

%\quotingsetup
%  {font=itshape
%  ,leftmargin=\parindent
%  ,listvskip}


%% Boxes %%

% \mdfsetup
%   {hidealllines=true
%   ,backgroundcolor=lightgray
%   }

% !TEX root=../main.tex


\input macros/auxiliaries


%% Fixes %%%%%%%%%%%%%%%%%%%%%%%%%%%%%%%%%%%%%%%%%%%%%%%%%%%%%%%%%%%%%%%%%%%%%%%

\let\texttilde\textasciitilde


%% Text %%%%%%%%%%%%%%%%%%%%%%%%%%%%%%%%%%%%%%%%%%%%%%%%%%%%%%%%%%%%%%%%%%%%%%%%

\newmacro{separate}
  {\medskip\noindent}

\providemacro{marginnote}
  {\marginpar}
\providemacro{smallcaps}
  {\textsc}
\providemacro{marginwidth}
  {\marginparwidth}

\newmacro{alert}[1]
  {\textbf{#1}}
\newmacro{divert}[1]
  {\textcolor{gray}{#1}}
\newmacro{enquote}[1]
  {``#1''}
\newmacro{fixme}[1]
  {\colorbox{yellow}{#1}\marginnote{\colorbox{yellow}{$\star$}}}
\newmacro{todo}[1]
  {\textcolor{red}{$\star$}\marginnote{\textcolor{red}{#1}}}
  % {}
\newmacro{type}[1]
  {\texttt{#1}}

\newmacro{add}[1]
  {\textcolor{green}{#1}}
\newmacro{remove}[1]
  {\textcolor{red}{#1}}
\newmacro{change}[1]
  {\textcolor{orange}{#1}}
\newmacro{adjust}[2]
  {\remove{#1} \add{#2}}

\newenvironment{fadeout}
  {\color{gray}}
  {}
\newenvironment{emphasize}
  {\begin{quote}\itshape}
  {\end{quote}}
\newenvironment{margintext}[1]
  {\begin{marginfigure}
     \subsection*{#1}}
  {\end{marginfigure}}


%% Lists %%
%% NOTE: requires `paralist`

% %% Use compact lists by default
% \renewenvironment{itemize}
%   {\begin{compactitem}}
%   {\end{compactitem}}
% \renewenvironment{enumerate}
%   {\begin{compactenum}}
%   {\end{compactenum}}
% \renewenvironment{description}
%   {\begin{compactdesc}}
%   {\end{compactdesc}}
% %% Define starred versions as in-paragraph-lists
% \newenvironment{itemize*}
%   {\begin{inparaitem}}
%   {\end{inparaitem}}
% \newenvironment{enumerate*}[1][1=(i)]
%   {\begin{inparaenum}[#1]}
%   {\end{inparaenum}}
% \newenvironment{description*}
%   {\begin{inparadesc}}
%   {\end{inparadesc}}


%% Quotations %%
%% NOTE: requires `quoting`

\let\quote\quoting
\let\endquote\endquoting
\renewenvironment{quotation}
  {\ClassError{Please use the `quote` environment instead of `quotation`}}


%% Column types %%
%% NOTE: requires `array`

\newcolumntype{L}{>{$}l<{$}}
\newcolumntype{C}{>{$}c<{$}}
\newcolumntype{R}{>{$}r<{$}}
\newcolumntype{T}{>{\ttfamily}l}
\newcolumntype{S}{>{\sffamily}l}


%% References %%
%%NOTE: requires `cleveref`

\let\refer\cref
\let\Refer\Cref


%% Citations %%
%% NOTE: requires `natbib`

\let\cite\citep
\let\Cite\Citep
\let\textcite\citet
\let\Textcite\Citet


%% Blocks and Boxes %%

\newenvironment{block}
  {\begin{center}}
  {\end{center}}
% \newenvironment{box}
%   {\begin{mdframed}}
%   {\end{mdframed}}


%% Logos %%

%% \newlogo[.name.]{.text.}
\newmacro{newlogo}[2][1]
  {\ifthenelse{\isempty{#1}}
     {\newlogoaux{#2}{\smallcaps{\lowercase{#2}}}}
     {\newlogoaux{#1}{#2}}}
\newmacro{newlogoaux}[2]
  {\newmacro{#1}{#2}}


%% Languages %%%%%%%%%%%%%%%%%%%%%%%%%%%%%%%%%%%%%%%%%%%%%%%%%%%%%%%%%%%%%%%%%%%

%%NOTE: `\mathrel` gives a single space width between keywords but removes it after another relational operator.
%%      `\mathop`  gives just a small skip, but doesn't has above bug.
\newmacro{newoperator}[1]
  {\newmathcommand{#1}[op]}
\newmacro{newkeyword}[2][1]
  %%FIXME: this is to complicated: {\newoperator{\ifthenelse{\isempty{#1}}{#2}{#1}}{\text{\sffamily\bfseries #2}}}
  {\ifthenelse{\isempty{#1}}
    {\newoperator{#2}{\text{\normalfont\sffamily\bfseries #2}}}
    {\newoperator{#1}{\text{\normalfont\sffamily\bfseries #2}}}}
\newmacro{newvalue}[2][1]
  {\ifthenelse{\isempty{#1}}
    {\newoperator{#2}{\text{\normalfont\sffamily #2}}}
    {\newoperator{#1}{\text{\normalfont\sffamily #2}}}}
\newmacro{newtype}[2][1]
  {\ifthenelse{\isempty{#1}}
    {\newoperator{#2}{\text{\normalfont\sffamily\scshape #2}}}
    {\newoperator{#1}{\text{\normalfont\sffamily\scshape #2}}}}


%% Math %%%%%%%%%%%%%%%%%%%%%%%%%%%%%%%%%%%%%%%%%%%%%%%%%%%%%%%%%%%%%%%%%%%%%%%%

%% Boxes %%

\newmacro{obox}[2]
  {\makebox[0pt][l]{\ensuremath{#2}}\phantom{\ensuremath{#1}}}

\newmacro{highlight}[1]
  {\colorbox{lightgray}{\ensuremath{#1}}}

%% Spacing %%

\newmacro{Quad}
  {\hspace{1.5em}}
\newmacro{Break}
  {\\[\smallskipamount]}



%% Fractions %%

\newmacro{upon}
  {\genfrac{}{}{0pt}{0}}



%% Symbols %%

%% NOTE: change this to \emptyset when using a font that includes a nice standard emptyset
\let\nothing\varnothing



%% Braces %%

\let\<\langle
\let\>\rangle

\newmathcommand{llbrace}[open] {\{\!|}
\newmathcommand{rrbrace}[close]{|\!\}}

\newmacro{set}[1]
  {\ensuremath{\{#1\}}}
\newmacro{tuple}[1]
  {\ensuremath{\<#1\>}}


%% Operators %%

\let\lt<
\let\gt>
\let\To\Rightarrow

\newmathcommand{pp}[bin]
  {+\!\!+}
\newmathcommand{Mid}
  {\;\mid\;}


%% Shortcuts %%

\newmacro{powerset}[1]
  % {2^{#1}}
  {\mathcal{P}(#1)}

\newmathcommand{n}{\underline{n}}

\newmathcommand{NN}  [bb]{N}
\newmathcommand{ZZ}  [bb]{Z}
\newmathcommand{EE}  [bb]{E}
\newmathcommand{OO}  [bb]{O}
\newmathcommand{QQ}  [bb]{A}
\newmathcommand{RR}  [bb]{R}
\newmathcommand{CC}  [bb]{C}
\newmathcommand{HH}  [bb]{H}

\newmathcommand{LL}  [bb]{L}
\newmathcommand{UU}  [bb]{U}
\newmathcommand{BB}  [bb]{B}
\renewmathcommand{SS}[bb]{S}

\let\to\rightarrow
% \let\implies\Rightarrow
% \let\implies\Longrightarrow
\let\implies\supset
\let\infers\vdash


%% Hints and local definitions %%

\newmacro{hint}[1]
  {\quad\text{\{ #1 \}}}

\newmathcommand{when}[op]
  {\mathbf{when}}
\newmathcommand{where}[op]
  {\mathbf{where}}
\renewmathcommand{and}[op]
  {\mathbf{and}}
\newmathcommand{otherwise}[op]
  {\mathbf{otherwise}}
\newmathcommand{impossible}[op]
  {\mathrm{impossible}}


%% Environments %%

\let\group\begingroup

\newenvironment*{marginequation}[1][1=0pt]
  {\begin{marginfigure}[#1]\equation}
  {\endequation\end{marginfigure}}

\newenvironment*{marginequation*}[1][1=0pt]
  {\begin{marginfigure}[#1]\equation\nonumber}
  {\endequation\end{marginfigure}}


\newenvironment*{function}
  {\begin{tabular}{@{}L@{\ \ }C@{\ \ }L@{}}}
  {\end{tabular}}
\newmacro{signature}[1]
  {\multicolumn{3}{@{}L@{}}{#1}}
\newmacro{inset}[1]
  {\multicolumn{3}{L}{\quad #1}}


\newenvironment*{grammar}
  %%NOTE: the `@{}` suppreses `\tabcolsep` before the first column
  {\begin{block}\begin{tabular}{@{}rRCLl}}
  {\end{tabular}\end{block}}
\newenvironment*{grammar*}
  %%NOTE: the `@{}` suppreses `\tabcolsep` before the first column
  {\begin{block}\begin{tabular}{@{}RLl}}
  {\end{tabular}\end{block}}



%% Theorems %%

% \newtheoremstyle{plain}%
%   {\medskipamount}% space above
%   {\medskipamount}% space below
%   {\itshape}% body font
%   {0pt}% indent amount
%   {\bfseries}% head font
%   {.}% punctuation after head
%   {.5em}% spacing after head
%   {\thmname{#1}\thmnumber{ #2}\thmnote{ {\normalfont(#3)}}}% head spec
% \newtheoremstyle{definition}%
%   {\medskipamount}% space above
%   {\medskipamount}% space below
%   {\normalfont}% body font
%   {0pt}% indent amount
%   {\bfseries}% head font
%   {.}% punctuation after head
%   {.5em}% spacing after head
%   {\thmname{#1}\thmnumber{ #2}\thmnote{ {\normalfont(#3)}}}% head spec
%
%
\theoremstyle{acmplain}

\newtheorem{theorem}{Theorem}[section]
\newtheorem{conjecture}[theorem]{Conjecture}
\newtheorem{proposition}[theorem]{Proposition}
\newtheorem{lemma}[theorem]{Lemma}
\newtheorem{corollary}[theorem]{Corollary}


\theoremstyle{acmdefinition}

\newtheorem{example}[theorem]{Example}
\newtheorem{definition}[theorem]{Definition}



%% Inference rules %%

\newmacro{placerule}[4][1,4]
  {\ensuremath{
    \upon
      {\text{\smallcaps{#1}}\hfill}
      {\dfrac{#2}{#3}\ #4}
  }}

\newmacro{newrule}[4][4]
  {\newmacro{#1}{\placerule[#1]{#2}{#3}[#4]}}
\newmacro{userule}
  {\usemacro}
\newmacro{refrule}[1]
  {\ifthenelse{\isundefined{#1}}
    {\GenericError{}{Rule `#1` is not defined}{}{}}
    {\textsc{#1}}}
% \newmacro{refrule}
%   {\textsc}

% !TEX root=../main.tex


%% Styles %%%%%%%%%%%%%%%%%%%%%%%%%%%%%%%%%%%%%%%%%%%%%%%%%%%%%%%%%%%%%%%%%%%%%%

\lstdefinestyle{common}
  {escapechar=|
  ,numbersep=-9pt % to make numbers appear inside the column; otherwise they are in the margin
  ,aboveskip=0pt
  ,belowskip=0pt
  }

\lstdefinestyle{natural}
  {style=common
  ,columns=fullflexible
  ,gobble=2
  ,breaklines=true
  ,breakatwhitespace=true
  ,literate=
    %{.}{{$\cdot$}}1
    %{.}{{\ }}1
    {<<}{{$\<$}}1
    {>>}{{$\>$}}1
    {->}{{$\to$\ }}2
    % {--}{{--}}1
    %{_}{{\ }}1
    %{\ "}{{\ \textquotedblleft}}2
    %{"\ }{{\textquotedblright\ }}2
  ,basicstyle={\sffamily}
  ,keywordstyle=[1]{\bfseries}
  ,keywordstyle=[2]{\scshape}
  ,keywordstyle=[3]{}
  %,commentstyle={\itshape}
  %,identifierstyle={\itshape}
  ,emphstyle={\itshape}
  %,stringstyle={\rmfamily}
  ,showstringspaces=false
  ,texcl=true
  ,mathescape=true
  %,escapechar=\$
  %,escapeinside={\{\}}
  ,xleftmargin=1\parindent
  }

\lstdefinestyle{flexible}
  {columns=flexible
  ,gobble=2
  ,fontadjust=true
  ,basicstyle={\ttfamily\small}
  ,commentstyle={\itshape}
  ,keywordstyle={\bfseries}
  %,identifierstyle={\itshape}
  %,stringstyle={\ttfamily}
  ,emphstyle={\itshape}
  ,showstringspaces=false
  ,texcl=true
  ,mathescape=true
  %,escapechar=\$
  %,escapeinside={\{\}}
  ,xleftmargin=1\parindent
  }

\lstdefinestyle{literate}
  {style=natural
  ,literate=
    {\\}{{$\lambda$}}1
    {\\\$}{{\$}}1 %NOTE: otherwise eaten by `\`, NOTE: prevents \$ to be parsed as math escape
    {\\/}{{$\vee$}}1
    {/\\}{{$\wedge$}}1
    {A.}{{$\forall$}}1
    {E.}{{$\exist$}}1
    {->}{{$\rightarrow$ }}1
    {<-}{{$\leftarrow$}}1
    {==}{{$\equiv$\ }}1
    {/=}{{$\nequiv$\ }}1
    {<=}{{$\leq$}}1
    {>=}{{$\geq$}}1
    {>>=}{{>>=}}3 %NOTE: otherwise eaten by `>=`
    {\{|}{{$\{\!|\!$}}1
    {|\}}{{$\!|\!\}$}}1
    {\{|*|\}}{{$\{\!|\!\!\star\!\!|\!\}$}}3
  }


%% Definitions %%%%%%%%%%%%%%%%%%%%%%%%%%%%%%%%%%%%%%%%%%%%%%%%%%%%%%%%%%%%%%%%%

%% Tasks %%

\lstdefinelanguage{tasks}
  {sensitive=true
  ,morekeywords=[1]{let,in,if,then,else,case,of,ref,assert,type}
  ,morekeywords=[2]{Bool,Int,String,Unit,List, Ref,Task, Passenger,Seat,Booking, Snack}
  ,moreemph={a,b,c,d,e,f,g,h,i,j,k,l,m,n,o,p,q,r,s,t,u,v,w,x,y,z as,bs,cs,ds,es,fs,gs,hs,is,js,ks,ls,ms,ns,os,ps,qs,rs,ss,ts,us,vs,ws,xs,ys,zs}
  ,morestring=[b]"
  ,morecomment=[l]--
  ,morecomment=[n]{\{-}{-\}}
  }[keywords,strings,comments]
\lstdefinestyle{tasks}
  {style=natural
  ,literate=
    {\\}{{$\lambda$}}1
    {<<}{{$\<$}}1
    {>>}{{$\>$ }}1
    {->}{{$\to$ }}1
    {==}{{$\equiv$ }}1
    {/=}{{$\nequiv$ }}1
    {<=}{{$\leq$ }}1
    {>=}{{$\geq$ }}1
    {*}{{$\times$ }}1
    {`elem`}{{$\in$ }}1
    {\\/}{{$\vee$ }}1
    {/\\}{{$\wedge$ }}1
    {>>=}{{$\Then$ }}1
    {>>?}{{$\Next$ }}1
    {<&>}{{$\And$ }}1
    {<|>}{{$\Or$ }}1
    {<?>}{{$\Xor$ }}1
    {++}{{$\pp$ }}1
    {edit}{{$\Edit$}}1
    {enter}{{$\Enter$}}1
    {update}{{$\Update$}}1
    {fail}{{$\Fail$ }}1
  }

\lstnewenvironment{TASK}[1][]
  {\lstset{language=tasks,style=tasks,#1}}
  {}
\newmacro{TS}[1][]
  {\lstinline[language=tasks,style=tasks,#1]}
\newmacro{includeTASK}[2][]
  {\lstinputlisting[language=tasks,style=tasks,#1]{#2}}


%% Flows %%

\lstdefinelanguage{flows}
  {sensitive=true
  ,morekeywords=[1]{module,where,define,using,as,yielding,share,holding,with,do,for,fork,then,when,next,done,on,and,or,not,readonly,writeonly,readwrite}
  ,morekeywords=[2]{Bool,Int,String,Shared,List, Date,Document,Photo, Citizen,Company,Declaration}
  ,morekeywords=[3]{True,False,Just,Nothing,List}
  ,morestring=[b]"
  ,morecomment=[l]--
  ,morecomment=[n]{\{-}{-\}}
  }[keywords,strings,comments]

% \lstMakeShortInline[language=flows,style=natural] | % |
\lstnewenvironment{FLOW}[1][]
  {\lstset{language=flows,style=natural,#1}}
  {}
\newmacro{FL}[1][]
  {\lstinline[language=flows,style=natural,#1]}
\newmacro{includeFLOW}[2][]
  {\lstinputlisting[language=flows,style=natural,#1]{#2}}


%% Clean %%

\lstdefinelanguage{clean}
  {sensitive=true
  %,alsoletter={ABCDEFGHIJKLMNOPQRSTUVWXYZabcdefghijklmnopqrstuvwxyz_`}
  %,alsoletter={~!@\#$\%^\&*-+=?<>:|\\} %$
  ,morekeywords={from,definition,implementation,import,module,system,code,inline,if,case,of,let,let!,in,where,with,class,instance,generic,derive,dynamic,infix,infixl,infixr}
  ,morestring=[b]"
  ,morestring=[b]'
  ,morecomment=[l]//
  ,morecomment=[n]{/*}{*/}
  }[keywords,strings,comments]

\lstnewenvironment{CLEAN}[1][]
  {\lstset{language=clean,style=flexible,#1}}
  {}
\newmacro{CL}[1][]
  {\lstinline[language=clean,style=flexible,#1]}
\newmacro{includeCLEAN}[2][]
  {\lstinputlisting[language=clean,style=flexible,#1]{#2}}

\input{macros/abbreviations}
% !TEX root=main.tex



\let\phi\varphi

%% Host language %%%%%%%%%%%%%%%%%%%%%%%%%%%%%%%%%%%%%%%%%%%%%%%%%%%%%%%%%%%%%%%


\newkeyword[IF]  {if}
\newkeyword[THEN]{then}
\newkeyword[ELSE]{else}

\newkeyword[Let]{let}
\newkeyword[In]{in}

\newkeyword[Ref] {ref}


\newmacro{If}[3]
  {\IF #1 \THEN #2 \ELSE #3}



%% Values %%


\newmathcommand{unit}{\<\>}


\newvalue{True}
\newvalue{False}
\newvalue[Not]{not}


\newmacro{str}[1]
  {\text{``#1''}}

\newvalue[Map]{map}
\newvalue[Fst]{fst}
\newvalue[Snd]{snd}
\newvalue[Head]{head}
\newvalue[Tail]{tail}
\newvalue[Uniq]{uniq}
\newvalue[Len]{len}



%% Types %%


\newtype{Unit}
\newtype{Bool}
\newtype{Nat}
\newtype{Int}
\newtype{String}
\newtype[Reference]{Ref}
\newtype{Task}
\newtype{Maybe}
\newtype{List}

\newtype{Euro}



%% Object language %%%%%%%%%%%%%%%%%%%%%%%%%%%%%%%%%%%%%%%%%%%%%%%%%%%%%%%%%%%%%


\newmacro{TOPHAT}
  {$\widehat{\text{\smallcaps{top}}}$}
\newmacro{STOPHAT}
  {Symbolic $\widehat{\smallcaps{top}}$}


\let\And\relax
\newoperator{Then}  {\blacktriangleright}
\newoperator{Next}  {\vartriangleright}
\newoperator{And}   {\Join}
\newoperator{Or}    {\blacklozenge}
\newoperator{Xor}   {\lozenge}
\newoperator{Edit}  {\square}
\newoperator{View}  {\overline{\square}}
\newoperator{Enter} {\boxtimes}
\newoperator{Update}{\blacksquare}
\newoperator{Watch} {\overline{\blacksquare}}
\newoperator{Fail}  {\lightning}
\newoperator{At}    {@}

\newoperator{AndOr} {\DEPRECATED}



%% Events %%


\newvalue[Left]   {L}
\newvalue[Right]  {R}


\newvalue[Empty]   {E}
\newvalue[Continue]{C}
\newvalue[Pick]    {P}


\newvalue[First]  {F}
\newvalue[Second] {S}
\newvalue[Here]   {H}



%% Semantic functions %%%%%%%%%%%%%%%%%%%%%%%%%%%%%%%%%%%%%%%%%%%%%%%%%%%%%%%%%%


\newmathcommand{eval}[rel]
  {\;\downarrow\;}
\newmathcommand{stride}[rel]
  % {\;\rightarrow\!\shortmid\;}
  {\;\mapsto\;}
\newmathcommand{normalise}[rel]
  {\;\Downarrow\;}
\newmacro{handle}[1]
  {\mathrel{\;\xrightarrow{#1}\;}}
\newmacro{interact}[1]
  {\mathrel{\;\Rightarrow{#1}\;}}

  \newmathcommand{Leadsto}[rel]
   {\rotatebox[origin=c]{90}{\rotatebox[origin=c]{-90}{$\leadsto$}\!\rotatebox[origin=c]{-90}{$\leadsto$}}}

  \newmathcommand{simeval}[rel]
    {\;\rotatebox[origin=c]{-90}{$\leadsto$}\;}
  \newmathcommand{simstride}[rel]
    {\;\mapstochar\kern+0.08em\leadsto\;}
  \newmathcommand{simnormalise}[rel]
    {\;\rotatebox[origin=c]{-90}{$\Leadsto$}\;}
  \newmathcommand{simhandle}[rel]
    {\;\leadsto\;}
  \newmathcommand{siminteract}[rel]
    {\;\Leadsto\;}


    \newmathcommand{simi}[]
      {\tilde{\imath}}
    \newmathcommand{simt}[]
        {\tilde{t}}
    \newmathcommand{sims}[]
        {\tilde{\sigma}}
      \newmathcommand{sime}[]
        {\tilde{e}}
    \newmathcommand{simv}[]
        {\tilde{v}}

\newmathcommand{Simulate}[it]
  {simulate}
\newmathcommand{Again}[it]
  {again}

\newmathcommand{Value}[cal]
  {V}
\newmathcommand{Inputs}[cal]
  {I}
\newmathcommand{Interface}[cal]
  {U}
\newmathcommand{Failing}[cal]
  {F}
\newmathcommand{Watching}[cal]
  {W}
\newmathcommand{Dirty}
  {\Delta}
\newmathcommand{UserInterface}[cal]
  {U}
\newmathcommand{Sat}[cal]
  {S}



%% Proofs %%%%%%%%%%%%%%%%%%%%%%%%%%%%%%%%%%%%%%%%%%%%%%%%%%%%%%%%%%%%%%%%%%%%%%


% \newcommand{\case}[2]{
%     \noindent\textbf{Case} #1\\
%     \vspace{5mm}
%     \indent\begin{minipage}{\dimexpr\textwidth-3cm}
%     #2
%   \end{minipage}\\\\}

\newcommand{\case}[2]{
  \bigskip
  \noindent\textbf{Case} #1
  \nopagebreak[4]
  \smallskip
  \par
  \begingroup
    \leftskip\parindent
    \noindent
    #2
    \par
  \endgroup
}

% \newcommand{\case}[2]{
%   \noindent
%   \begin{tabular*}{\textwidth}{lp{0.8\textwidth}}
%     \textbf{Case} & #1 \\
%     \addlinespace
%     & #2
%   \end{tabular*}
%   \medskip
% }



%% Depricated %%%%%%%%%%%%%%%%%%%%%%%%%%%%%%%%%%%%%%%%%%%%%%%%%%%%%%%%%%%%%%%%%%

% !TEX root=main.tex


%% Typing %%%%%%%%%%%%%%%%%%%%%%%%%%%%%%%%%%%%%%%%%%%%%%%%%%%%%%%%%%%%%%%%%%%%%%

\newrule{T-Sym}
  {s:\beta \in \Gamma}
  {\Gamma,\Sigma \infers s:\tau}
  {}


%% Evaluation %%%%%%%%%%%%%%%%%%%%%%%%%%%%%%%%%%%%%%%%%%%%%%%%%%%%%%%%%%%%%%%%%%


\newmacro{RelationSE}
  {\sime,\sims \simeval \overline{\simv,\sims',\highlight{\phi}}}


\newrule{SE-Value}
  {}
  {\simv,\sims\simeval \simv,\highlight{\sims,\True}}
  {}


\newrule{SE-App}
  {\sime_1,\sims\simeval \overline{\lambda x:\tau.\sime_1',\sims',\highlight{\phi_1}}\Quad
   \sime_2,\sims'\simeval \overline{\simv_2,\sims'',\highlight{\phi_2}}
   \sime_1'[x\mapsto \simv_2],\sims''\simeval \overline{\simv_1,\sims''',{\phi_3}}}
  {\sime_1 \sime_2,\sims \simeval \overline{\simv_1,\sims''',\highlight{\phi_1\land\phi_2\land\phi_3}}}
  {}

\newrule{SE-If}
  {\sime_1,\sims\simeval \overline{\simv_1,\sims',\highlight{\phi_1}} \Quad
   \highlight{\sime_2,\sims'\simeval \overline{\simv_2,\sims'',\phi_2}} \Quad
   \highlight{\sime_3,\sims'\simeval \overline{\simv_3,\sims''',\phi_3}}}
  {\If{\sime_1}{\sime_2}{\sime_3},\sims\simeval \highlight{\overline{\simv_2,\sims'',\phi_1 \land \phi_2\land \simv_1} \cup \overline{\simv_3,\sims''',\phi_1 \land \phi_3 \land \lnot \simv_1}}}
  {}



\newrule{SE-Pair}
  {\upon{\sime_1,\sims\simeval \overline{\simv_1,\sims',\highlight{\phi_1}}}
   {\sime_2,\sims'\simeval \overline{\simv_2,\sims'',\highlight{\phi_2}}}}
  {\tuple{\sime_1,\sime_2},\sims\simeval\overline{\tuple{\simv_1,\simv_2},\sims'',\highlight{\phi_1\land\phi_2}}}
  {}

\newrule{SE-First}
  {\sime,\sims\simeval\overline{\tuple{\simv_1,\simv_2},\sims',\highlight{\phi}}}
  {\Fst \sime,\sims\simeval\overline{\simv_1,\sims',\highlight{\phi}} }
  {}

\newrule{SE-Second}
  {\sime,\sims\simeval\overline{\tuple{\simv_1,\simv_2},\sims',\highlight{\phi}}}
  {\Snd\sime,\sims\simeval\overline{\simv_2,\sims',\highlight{\phi}} }
  {}


%%%%%%%

\newrule{SE-Cons}
  {\sime_1,\sims \simeval \simv_1,\sims',\highlight{\phi_1}
   \sime_2,\sims' \simeval \simv_2,\sims'',\highlight{\phi_2}}
  {\sime_1 :: \sime_2,\sims \simeval \simv_1:: \simv_2,\sims'',\highlight{\phi_1\land\phi_2}}
  {}

\newrule{SE-Head}
  {\sime,\sims \simeval \simv_1::\simv_2,\sims',\highlight{\phi}}
  {\Head \sime,\sims \simeval \simv_1,\sims',\highlight{\phi}}
  {}

\newrule{SE-Tail}
  {\sime,\sims \simeval \simv_1::\simv_2,\sims',\highlight{\phi}}
  {\Tail \sime,\sims \simeval \simv_2,\sims',\highlight{\phi}}
  {}


%%%%%
\newrule{SE-Ref}
  {\sime,\sims\simeval \overline{\simv,\sims',\highlight{\phi}}\Quad
   l\not\in Dom(\sigma')}
  {\Ref \sime,\sims\simeval \overline{l,\sims'[l\mapsto \simv],\highlight{\phi}}}
  {}


\newrule{SE-Deref}
  {\sime,\sims\simeval \overline{l,\sims',\highlight{\phi}}}
  {!\sime,\sims\simeval \overline{\sims'(l),\sims',\highlight{\phi}}}
  {}

\newrule{SE-Assign}
  {\sime_1,\sims\simeval \overline{l,\sims',\highlight{\phi_1}} \Quad
   \sime_2,\sims'\simeval \overline{\simv_2,\sims'',\highlight{\phi_2}}}
  {\sime_1:=\sime_2,\sims\simeval \overline{\unit,\sims''[l\mapsto \simv_2],\highlight{\phi_1\wedge\phi_2}}}
  {}

\newrule{SE-Edit}
  {\sime,\sims \simeval \overline{\simv,\sims',\highlight{\phi}}}
  {\Edit \sime , \sims\simeval \overline{\Edit \simv,\sims',\highlight{\phi}}}
  {}

\newrule{SE-Update}
  {\sime,\sims\simeval \overline{l,\sims',\highlight{\phi}}}
  {\Update \sime ,\sims\simeval \overline{\Update l,\sims',\highlight{\phi}}}
  {}


\newrule{SE-Fail}
  {}
  {\Fail,\sims \simeval \Fail,\sims,\highlight{\True}}
  {}


\newrule{SE-Then}
  {\sime_1 ,\sims\simeval \overline{\simt_1,\sims',\highlight{\phi}}}
  {\sime_1 \Then \sime_2,\sims \simeval \overline{\simt_1 \Then \sime_2,\sims',\highlight{\phi}}}
  {}

\newrule{SE-Next}
  {\sime_1 ,\sims\simeval \overline{\simt_1,\sims',\highlight{\phi}}}
  {\sime_1 \Next \sime_2 ,\sims\simeval \overline{\simt_1 \Next \sime_2,\sims',\highlight{\phi}}}
  {}


\newrule{SE-And}
  {\sime_1 ,\sims\simeval \overline{\simt_1 ,\sims',\highlight{\phi_1}} \Quad
   \sime_2 ,\sims'\simeval \overline{\simt_2,\sims'',\highlight{\phi_2}}}
  {\sime_1 \And \sime_2 ,\sims\simeval \overline{\simt_1 \And \simt_2,\sims'',\highlight{\phi_1\land\phi_2}}}
  {}


\newrule{SE-Or}
  {\sime_1 ,\sims\simeval \overline{\simt_1 ,\sims',\highlight{\phi_1}} \Quad
   \sime_2 ,\sims'\simeval \overline{\simt_2,\sims'',\highlight{\phi_2}}}
  {\sime_1 \Or \sime_2 ,\sims\simeval \overline{\simt_1 \Or \simt_2,\sims'',\highlight{\phi_1\land\phi_2}}}
  {}

%% Normalisation %%%%%%%%%%%%%%%%%%%%%%%%%%%%%%%%%%%%%%%%%%%%%%%%%%%%%%%%%%%%%%%


\newmacro{RelationSS}
  {\simt,\sims\simstride \overline{\simt',\sims',\highlight{\phi}}}


\newrule{SS-Edit}
  { }
  {\Edit \simv,\sims \simstride \Edit \simv,\sims,\highlight{\True}}
  {}

\newrule{SS-Fill}
  { }
  {\Enter \beta,\sims \simstride \Enter \beta,\sims,\highlight{\True}}
  {}

\newrule{SS-Update}
  { }
  {\Update l,\sims \simstride \Update l,\sims,\highlight{\True}}
  {}


\newrule{SS-Fail}
  { }
  {\Fail,\sims \simstride \Fail,\sims,\highlight{\True}}
  {}


\newrule{SS-ThenStay}
  {\simt_1,\sims \simstride \overline{\simt_1',\sims',\highlight{\phi}}}
  {\simt_1 \Then \sime_2,\sims \simstride \overline{\simt_1' \Then \sime_2,\sims',\highlight{\phi}}}
  {{\Value\ (\simt_1',\sims') = \bot}}

\newrule{SS-ThenFail}
  {\simt_1,\sims \simstride \overline{\simt_1',\sims',\highlight{\phi}} \Quad
   \sime_2\ \simv_1,\sims' \simeval \overline{\simt_2,\sims'',\highlight{\_}}}
  {\simt_1 \Then \sime_2,\sims \simstride \overline{\simt_1' \Then \sime_2,\sims',\highlight{\phi}}}
  {\Value\ (\simt_1',\sims') = \simv_1 \land \Failing\ (\simt_2,\sims'')}

\newrule{SS-ThenCont}
  {\simt_1,\sims \simstride \overline{\simt_1',\sims',\highlight{\phi_1}} \Quad
   \sime_2\ \simv_1,\sims' \simeval \overline{\simt_2 ,\sims'',\highlight{\phi_2}}}
   % t_2,\sigma'' \stride t_2',\sigma'''}
  {\simt_1 \Then \sime_2,\sims \simstride \overline{t_2,\sigma'',\highlight{\phi_1\land\phi_2}}}
  {\Value\ (\simt_1',\sims') = \simv_1 \land \lnot\Failing\ (\simt_2,\sims'')}

\newrule{SS-Next}
  {\simt_1,\sims \simstride \overline{\simt_1',\sims',\highlight{\phi}}}
  {\simt_1 \Next \sime_2,\sims \simstride \overline{\simt_1' \Next \sime_2,\sims',\highlight{\phi}}}
  {}


\newrule{SS-And}
  {\simt_1,\sims  \simstride \overline{\simt_1',\sims',\highlight{\phi_1 }} \Quad
   \simt_2,\sims' \simstride \overline{\simt_2',\sims'',\highlight{\phi_2}}}
  {\simt_1 \And \simt_2,\sims \simstride \overline{\simt_1' \And \simt_2',\sims'',\highlight{\phi_1\land\phi_2}}}
  {}


\newrule{SS-OrLeft}
  {\simt_1,\sims  \simstride \overline{\simt_1',\sims',\highlight{\phi}}}
  {\simt_1 \Or \simt_2,\sims \simstride \overline{\simt_1',\sims',\highlight{\phi}}}
  {{\Value\ (\simt_1',\sims') = \simv_1}}

\newrule{SS-OrRight}
  {\simt_1,\sims  \simstride \overline{\simt_1',\sims',\highlight{\phi_1}}  \Quad
   \simt_2,\sims' \simstride \overline{\simt_2',\sims'',\highlight{\phi_2}}}
  {\simt_1 \Or \simt_2,\sims \simstride \overline{\simt_2',\sims'',\highlight{\phi_1\land\phi_2}}}
  {\Value\ (\simt_1',\sims') = \bot\land \Value\ (\simt_2',\sims'') = \simv_2}

\newrule{SS-OrNone}
  {\simt_1,\sims  \simstride \overline{\simt_1',\sims' ,\highlight{\phi_1}} \Quad
   \simt_2,\sims' \simstride \overline{\simt_2',\sims'',\highlight{\phi_2}}}
  {\simt_1 \Or \simt_2,\sims \simstride \overline{\simt_1' \Or \simt_2',\sims'',\highlight{\phi_1\land\phi_2}}}
  {\Value\ (\simt_1',\sims') = \bot \land \Value\ (\simt_2',\sims'') = \bot}

\newrule{SS-Xor}
  {\ }
  {\sime_1 \Xor \sime_2,\sims \simstride \sime_1 \Xor \sime_2,\sims,\highlight{\True}}
  {}


%% Normalisation %%


\newmacro{RelationSN}
  {\sime,\sims \simnormalise \overline{\simt,\sims',\highlight{\phi}}}


\newrule{SN-Done}
  {\sime,\sims \simeval \overline{\simt,\sims',\highlight{\phi_1}} \Quad
   \simt,\sims' \simstride \overline{\simt',\sims'',\highlight{\phi_2}}}
  {\sime,\sims \simnormalise \overline{\simt,\sims',\highlight{\phi_1\land\phi_2}}}
  {\sims'=\sims'' \land \simt=\simt'}

\newrule{SN-Repeat}
  {\upon{\sime,\sims \simeval \overline{\simt,\sims',\highlight{\phi_1}}}
   {{\simt,\sims' \simstride \overline{\simt',\sims'',\highlight{\phi_2}}}
   {\simt',\sims'' \simnormalise \overline{\simt'',\sims''',\highlight{\phi_3}}}}}
  {\sime,\sims \simnormalise \overline{\simt'',\sims''',\highlight{\phi_1 \land \phi_2 \land \phi_3}}}
  {\sims'\neq \sims''\vee \simt\neq \simt'}


%% Handling %%


\newmacro{RelationSH}
  {\simt,\sims \simhandle \overline{\simt',\sims',\highlight{\simi,\phi}}}


\newrule{SH-Change}
  { \text{fresh }s}
  {\Edit \simv,\sims \simhandle \Edit s,\sims,\highlight{s,\True}}
  {\simv,s:\beta}

\newrule{SH-Fill}
  { \text{fresh }s \Quad
    s:\beta}
  {\Enter \beta,\sims \simhandle \Edit s,\sims,\highlight{s,\True}}
  {}

\newrule{SH-Update}
  { \text{fresh }s \Quad
    \sims(l),s:\beta}
  {\Update l,\sims \simhandle \Update l,\sims[l \mapsto s],\highlight{s,\True}}
  {}

\newrule{SH-PassThen}
  {\simt_1,\sims \simhandle \overline{\simt_1',\sims',\simi,\phi}}
  {\simt_1 \Then \sime_2,\sims \simhandle \overline{\simt_1' \Then \sime_2,\sims',\highlight{\simi,\phi}}}
  {}

\newrule{SH-PassNext}
  {\simt_1,\sims \simhandle \overline{\simt_1',\sims',\simi,\phi} \Quad
   \Value\ {(\simt_1',\sims')} = \bot}
  {\simt_1 \Next \sime_2,\sims \simhandle \overline{\simt_1' \Next \sime_2,\sims',\highlight{\simi,\phi}}}
  {}


\newrule{SH-PassNextFail}
  {\upon{
   \simt_1,\sims \simhandle \overline{\simt_1',\sims_1,\simi,\phi} \Quad
   \Value\ {(\simt_1',\sims_1)} = \simv_1 }
   {\sime_2\ \simv_1,\sims_1 \simnormalise \overline{\simt_2,\sims_2,{\vphantom{i}\_}} \Quad
   \Failing\ (\simt_2,\sims_2)}}
  {\simt_1 \Next \sime_2,\sims \simhandle \overline{\simt_1' \Next \sime_2,\sims_1,\simi,\phi}}
  {}


\newrule{SH-Next}
  {\simt_1,\sims \simhandle \overline{\simt_1',\sims_1,\highlight{\simi,\phi_1}} \Quad
   \sime_2\ \simv_1,\sims_1 \simnormalise \overline{\simt_2,\sims_2,\highlight{\phi_2}}}
  {\simt_1 \Next \sime_2,\sims \simhandle\highlight{\overline{\simt_1' \Next \sime_2,\sims_1,\simi,\phi_1} \cup\overline{\simt_2,\sims_2,\Continue,\phi_2}}}
  {\Value\ {(\simt_1',\sims_1)} = \simv_1\land \neg\Failing\ (\simt_2,\sims_2)}

\newrule{SH-And}
  {\simt_1,\sims \simhandle \overline{\simt_1',\sims_1,\highlight{\simi_1,\phi_1}} \Quad
   \simt_2,\sims \simhandle \overline{\simt_2',\sims_2,\highlight{\simi_2,\phi_2}}}
  {\simt_1 \And \simt_2,\sims \simhandle \highlight{\overline{\simt_1' \And \simt_2,\sims_1,\First \simi_1,\phi_1}\cup \overline{\simt_1 \And \simt_2',\sims_2,\Second \simi_2,\phi_2}}}
  {}

\newrule{SH-Or}
  {\simt_1,\sims \simhandle \overline{\simt_1',\sims_1,\highlight{\simi_1,\phi_1}}\Quad
   \simt_2,\sims \simhandle \overline{\simt_2',\sims_2,\highlight{\simi_2,\phi_2}}}
  {\simt_1 \Or \simt_2,\sims \simhandle \highlight{\overline{\simt_1' \Or \simt_2,\sims_1,\First \simi_1,\phi_1}\cup\overline{\simt_1 \Or \simt_2',\sims_2,\Second \simi_2,\phi_2}}}
  {}


\newrule{SH-PickLeft}
  {\sime_1,\sims\simnormalise \overline{\simt_1,\sims_1,\highlight{\phi_1}} \Quad
   \sime_2,\sims \simnormalise \overline{\simt_2,\sims_2,\highlight{\phi_2}}}
  {\sime_1 \Xor \sime_2,\sims \simhandle \simt_1,\sims_1,\highlight{\Left,\phi_1}}
  {\neg\Failing\ (\simt_1,\sims_1)\land \Failing\ (\simt_2,\sims_2) }

\newrule{SH-PickRight}
  {\sime_1,\sims \simnormalise \overline{\simt_1,\sims_1,\highlight{\phi_1}} \Quad
   \sime_2,\sims \simnormalise \overline{\simt_2,\sims_2,\highlight{\phi_2}}}
  {\sime_1 \Xor \sime_2,\sims \simhandle \simt_2,\sims_2,\highlight{\Right,\phi_2}}
  {\Failing\ (\simt_1,\sims_1)\land \neg\Failing\ (\simt_2,\sims_2)}

\newrule{SH-Pick}
  {\sime_1,\sims \simnormalise \overline{\simt_1,\sims_1,\highlight{\phi_1}} \Quad
   \sime_2,\sims \simnormalise \overline{\simt_2,\sims_2,\highlight{\phi_2}}}
  {\sime_1 \Xor \sime_2,\sims \simhandle \highlight{\overline{\simt_1,\sims_1,\Left,\phi_1}\cup\overline{\simt_2,\sims_2,\Right,\phi_2}}}
  { \neg\Failing\ (\simt_1,\sims_1)\land \neg\Failing\ (\simt_2,\sims_2) }

%% Driving %%


\newmacro{RelationSI}
  {\simt,\sims \siminteract \overline{\simt',\sims',\highlight{\simi,\phi}}}


\newrule{SI-Handle}
  {\simt,\sims \simhandle \overline{\simt',\sims',\highlight{\simi,\phi_1}} \Quad
   \simt',\sims' \simnormalise \overline{\simt'',\sims'',\highlight{\phi_2}}}
  {\simt,\sims \siminteract \overline{\simt'',\sims'',\highlight{\simi,\phi_1 \land \phi_2}}}
  {}

%% Firsts %%

\newrule{R-Firsts}
  {t,\sigma\simulate\overline{\simv,\simi:\tilde{is},\Phi}}
  {\Firsts(t,\sigma,g) = \overline{\simi,\Phi\land g \simv}}
  {\Sat(\Phi\land g\simv)}

% !TEX root=main.tex


%% Typing %%%%%%%%%%%%%%%%%%%%%%%%%%%%%%%%%%%%%%%%%%%%%%%%%%%%%%%%%%%%%%%%%%%%%%


\newmacro{RelationT}
  {\Gamma,\Sigma \infers e : \tau}


\newrule{T-ConstBool}
  {c\in B}
  {\Gamma,\Sigma\infers c : \Bool}
  {}

\newrule{T-ConstInt}
  {c\in I}
  {\Gamma,\Sigma\infers c : \Int}
  {}

\newrule{T-ConstString}
  {c\in S}
  {\Gamma,\Sigma\infers c : \String}
  {}


\newrule{T-Unit}
  { }
  {\Gamma,\Sigma\infers \unit : \Unit}
  {}


\newrule{T-Var}
  {x:\tau\in\Gamma}
  {\Gamma,\Sigma\infers x:\tau}
  {}


\newrule{T-Abs}
  {\Gamma[x:\tau_1] ,\Sigma \infers e:\tau_2}
  {\Gamma,\Sigma \infers \lambda x : \tau_1 . e :\tau_1 \to \tau_2}
  {}

\newrule{T-App}
  { {\Gamma,\Sigma \infers e_1:\tau_1\to\tau_2 } \Quad
    {\Gamma,\Sigma \infers e_2:\tau_1}}
  {\Gamma,\Sigma \infers e_1 e_2 :\tau_2}
  {}


\newrule{T-If}
  { {\Gamma,\Sigma \infers e_1:\Bool}\Quad
   { {\Gamma,\Sigma \infers e_2:\tau}\Quad
   {\Gamma,\Sigma \infers e_3:\tau}}}
  {\Gamma,\Sigma \infers \If{e_1}{e_2}{e_3}:\tau}
  {}


\newrule{T-Pair}
  {\Gamma,\Sigma \infers e_1 : \tau_1  \Quad
   \Gamma,\Sigma \infers e_2 : \tau_2}
  {\Gamma,\Sigma \infers \tuple{e_1, e_2} :\tau_1 \times \tau_2}
  {}

\newrule{T-First}
  {\Gamma,\Sigma\infers e_1:\tau}
  {\Gamma,\Sigma\infers \Fst \tuple{e_1,e_2}:\tau}
  {}

\newrule{T-Second}
  {\Gamma,\Sigma\infers e_2:\tau}
  {\Gamma,\Sigma\infers \Snd \tuple{e_1,e_2}:\tau}
  {}

%%%%%
\newrule{T-ListEmpty}
  { }
  {\Gamma,\Sigma\infers [\ ]_\beta : \List\beta}
  {}

\newrule{T-ListCons}
  { {\Gamma,\Sigma\infers e_1:\beta}\Quad
   {\Gamma,\Sigma\infers e_2:\List\beta}}
  {\Gamma,\Sigma\infers e_1 :: e_2 : \List \beta}
  {}

\newrule{T-ListHead}
  {\Gamma,\Sigma\infers e:\List\beta}
  {\Gamma,\Sigma\infers \Head e:\beta}
  {}

\newrule{T-ListTail}
  {\Gamma,\Sigma\infers e:\List\beta}
  {\Gamma,\Sigma\infers \Tail e:\List\beta}
  {}

%%%%%


\newrule{T-Ref}
  {\Gamma,\Sigma \infers e:\beta}
  {\Gamma,\Sigma \infers \Ref e :\Reference \beta}
  {}

\newrule{T-Deref}
  {\Gamma,\Sigma \infers e:\Reference \beta}
  {\Gamma,\Sigma\infers\ !e:\beta}
  {}

\newrule{T-Assign}
  { {\Gamma,\Sigma\infers e_1:\Reference \beta}\Quad
   {\Gamma,\Sigma\infers e_2:\beta}}
  {\Gamma,\Sigma\infers e_1 := e_2:\Unit}
  {}

\newrule{T-Loc}
  {\Sigma(l) = \beta}
  {\Gamma,\Sigma\infers l:\Reference \beta}
  {}


\newrule{T-Edit}
  {\Gamma,\Sigma \infers e : \beta}
  {\Gamma,\Sigma \infers \Edit e : \Task \beta}
  {}

\newrule{T-Enter}
  {}
  {\Gamma,\Sigma \infers \Enter \beta : \Task \beta}
  {}

\newrule{T-Update}
  {\Gamma,\Sigma \infers e : \Reference \beta}
  {\Gamma,\Sigma \infers \Update e : \Task \beta}
  {}


\newrule{T-Fail}
  {}
  {\Gamma,\Sigma \infers \Fail : \Task \tau}
  {}


\newrule{T-Then}
  { {\Gamma,\Sigma \infers e_1 : \Task \tau_1}\Quad
   {\Gamma,\Sigma \infers e_2 : \tau_1 \to \Task \tau_2}}
  {\Gamma,\Sigma \infers e_1 \Then e_2 : \Task \tau_2}
  {}


\newrule{T-Next}
  { {\Gamma,\Sigma \infers e_1 : \Task \tau_1}\Quad
   {\Gamma,\Sigma \infers e_2 : \tau_1 \to \Task \tau_2}}
  {\Gamma,\Sigma \infers e_1 \Next e_2 : \Task \tau_2}
  {}


\newrule{T-And}
  { {\Gamma,\Sigma \infers e_1 : \Task \tau_1}\Quad
   {\Gamma,\Sigma \infers e_2 : \Task \tau_2}}
  {\Gamma,\Sigma \infers e_1 \And e_2 : \Task\,(\tau_1 \times \tau_2)}
  {}


\newrule{T-Or}
  { {\Gamma,\Sigma \infers e_1 : \Task \tau}\Quad
   {\Gamma,\Sigma \infers e_2 : \Task \tau}}
  {\Gamma,\Sigma \infers e_1 \Or e_2 : \Task \tau}
  {}


\newrule{T-Xor}
  { {\Gamma,\Sigma \infers e_1 : \Task \tau}\Quad
   {\Gamma,\Sigma \infers e_2 : \Task \tau}}
  {\Gamma,\Sigma \infers e_1 \Xor e_2 : \Task \tau}
  {}


%% Evaluation %%%%%%%%%%%%%%%%%%%%%%%%%%%%%%%%%%%%%%%%%%%%%%%%%%%%%%%%%%%%%%%%%%

\newmacro{RelationE}
  {e,\sigma \eval v,\sigma'}


\newrule{E-Value}
  {}
  {v,{\sigma}{\eval} v,{\sigma}}
  {}


\newrule{E-App}
  { {e_1               ,\sigma   \eval \lambda x:\tau.e_1',\sigma'}\Quad
   { {e_2               ,\sigma'  \eval v_2                ,\sigma''}\Quad
   {e_1'[x\mapsto v_2],\sigma'' \eval v_1                ,\sigma'''}}}
  {e_1 e_2           ,\sigma   \eval v_1                ,\sigma'''}
  {}


\newrule{E-IfTrue}
  { {e_1,{\sigma}{\eval} \True,{\sigma}'}\Quad
   {e_2,{\sigma}'{\eval} {v_2},{\sigma}''}}
  {\If{e_1}{e_2}{e_3},{\sigma}{\eval} {v_2},{\sigma}''}
  {}

\newrule{E-IfFalse}
  { {e_1,{\sigma}{\eval} \False ,{\sigma}'}\Quad
   {e_3,{\sigma}'{\eval} {v_3},{\sigma}''}}
  {\If{e_1}{e_2}{e_3},{\sigma}{\eval} {v_3},{\sigma}''}
  {}


\newrule{E-Pair}
  {e_1,{\sigma}{\eval} {v_1},{\sigma}' \Quad
   e_2,{\sigma}'{\eval} {v_2},{\sigma}''}
  {\tuple{e_1,e_2},{\sigma}{\eval}\tuple{{v_1},{v_2}},{\sigma}''}
  {}

\newrule{E-First}
  {e,\sigma\eval \tuple{v_1,v_2},\sigma'}
  {\Fst e,\sigma\eval v_1,\sigma'}
  {}

\newrule{E-Second}
  {e,\sigma\eval\tuple{v_1,v_2},\sigma'}
  {\Snd e,\sigma \eval v_2,\sigma' }
  {}

%%%%%%%%%

\newrule{E-Cons}
  {e_1,{\sigma}{\eval}{v_1},{\sigma}'\Quad
   e_2,{\sigma}'{\eval}{v_2},{\sigma}''}
  {e_1 :: e_2,{\sigma}{\eval}{v_1}::{v_2},{\sigma}''}
  {}

\newrule{E-Head}
  {e,{\sigma}{\eval} {v_1}::{v_2},{\sigma}'}
  {\Head e,{\sigma}{\eval}{v_1},{\sigma}'}
  {}

\newrule{E-Tail}
  {e,{\sigma}{\eval} {v_1}::{v_2},{\sigma}'}
  {\Tail e,{\sigma}{\eval}{v_2},{\sigma}'}
  {}

%%%%%%


\newrule{E-Ref}
  {e,{\sigma}{\eval} {v},{\sigma}' \Quad
   l\not\in Dom({\sigma}')}
  {\Ref e,{\sigma}{\eval} l,{\sigma}'[l\mapsto {v}]}
  {}

\newrule{E-Deref}
  {e,{\sigma}{\eval} l,{\sigma}'}
  {!e,{\sigma}{\eval} {\sigma}'(l),{\sigma}'}
  {}

\newrule{E-Assign}
  {e_1,{\sigma}{\eval} l,{\sigma}' \Quad
   e_2,{\sigma}'{\eval} {v_2},{\sigma}''}
  {e_1:=e_2,{\sigma}{\eval} \unit,{\sigma}''[l\mapsto {v_2}]}
  {}

\newrule{E-Edit}
  {e,{\sigma} {\eval} {v},{\sigma}'}
  {\Edit e , {\sigma}{\eval} \Edit {v},{\sigma}'}
  {}

\newrule{E-Update}
  {e,{\sigma}{\eval} l,{\sigma}'}
  {\Update e ,{\sigma}{\eval} \Update l,{\sigma}'}
  {}


\newrule{E-Fail}
  {}
  {\Fail,{\sigma} {\eval} \Fail,{\sigma}}
  {}


\newrule{E-Then}
  {e_1 ,{\sigma}{\eval} {t_1},{\sigma}'}
  {e_1 \Then e_2,{\sigma}{\eval}{t_1} \Then e_2,{\sigma}'}
  {}

\newrule{E-Next}
  {e_1 ,{\sigma}{\eval} {t_1},{\sigma}'}
  {e_1 \Next e_2 ,{\sigma}{\eval} {t_1} \Next e_2,{\sigma}'}
  {}


\newrule{E-And}
  {e_1 ,{\sigma}{\eval}{ t_1 },{\sigma}'\Quad
   e_2 ,{\sigma}'{\eval} {t_2},{\sigma}''}
  {e_1 \And e_2 ,{\sigma}{\eval}{ t_1} \And {t_2},{\sigma}''}
  {}


\newrule{E-Or}
  {e_1 ,{\sigma}{\eval}{ t_1} ,{\sigma}'\Quad
   e_2 ,{\sigma}'{\eval} {t_2},{\sigma}''}
  {e_1 \Or e_2 ,{\sigma}{\eval} {t_1} \Or {t_2},{\sigma}''}
  {}


%% Normalisation %%%%%%%%%%%%%%%%%%%%%%%%%%%%%%%%%%%%%%%%%%%%%%%%%%%%%%%%%%%%%%%

\newmacro{RelationS}
  {t,\sigma \stride t',\sigma'}


\newrule{S-Edit}
  { }
  {\Edit v,{\sigma} {\stride} \Edit v,{\sigma}}
  {}

\newrule{S-Fill}
  { }
  {\Enter \beta,{\sigma} {\stride} \Enter \beta,{\sigma}}
  {}

\newrule{S-Update}
  { }
  {\Update l,{\sigma} {\stride} \Update l,{\sigma}}
  {}


\newrule{S-Fail}
  { }
  {\Fail,{\sigma} {\stride} \Fail,{\sigma}}
  {}


\newrule{S-ThenStay}
  {t_1,{\sigma} {\stride} {t_1}',{\sigma}'}
  {t_1 \Then e_2,{\sigma} {\stride} {t_1}' \Then e_2,{\sigma}'}
  {\Value\ ({t_1}',{\sigma}') = \bot}

\newrule{S-ThenFail}
  { t_1,{\sigma} {\stride} {t_1}',{\sigma}' \Quad
    e_2\ {v_1},{\sigma}' {\eval} {t_2},{\sigma}''}
  {t_1 \Then e_2,{\sigma} {\stride} {t_1}' \Then e_2,{\sigma}'}
  {\Value\ ({t_1}',{\sigma}') = {v_1} \land \Failing\ ({t_2},{\sigma}'')}

\newrule{S-ThenCont}
  {t_1,{\sigma} {\stride} {t_1}',{\sigma}' \Quad
   e_2\ {v_1},{\sigma}' {\eval} {t_2 },{\sigma}''}
  {t_1 \Then e_2,{\sigma} {\stride} {t_2},{\sigma}''}
  {\Value\ ({t_1}',{\sigma}') = {v_1} \land \lnot\Failing\ ({t_2},{\sigma}'') }

\newrule{S-Next}
  {t_1,{\sigma} {\stride} {t_1}',{\sigma}'}
  {t_1 \Next e_2,{\sigma} {\stride} {t_1}' \Next e_2,{\sigma}'}
  {}


\newrule{S-And}
  {t_1,{\sigma}  {\stride} {t_1}',{\sigma}'  \Quad
   t_2,{\sigma}' {\stride} {t_2}',{\sigma}''}
  {t_1 \And t_2,{\sigma} {\stride} {t_1}' \And {t_2}',{\sigma}''}
  {}


\newrule{S-OrLeft}
  {t_1,{\sigma}  {\stride} {t_1}',{\sigma}'}
  {t_1 \Or t_2,{\sigma} {\stride} {t_1}',{\sigma}'}
  {\Value\ ({t_1}',{\sigma}') = {v_1}}

\newrule{S-OrRight}
  { {t_1,{\sigma}  {\stride} {t_1}',{\sigma}'}\Quad
   {t_2,{\sigma}' {\stride} {t_2}',{\sigma}''}}
  {t_1 \Or t_2,{\sigma} {\stride} {t_2}',{\sigma}''}
  {\Value\ ({t_1}',{\sigma}') = \bot \land \Value\ ({t_2}',{\sigma}'') = {v_2}}

\newrule{S-OrNone}
  { {t_1,{\sigma}  {\stride }{t_1}',{\sigma}'}\Quad
   { t_2,{\sigma' }{\stride} {t_2}',{\sigma}''}}
  {t_1 \Or t_2,{\sigma} {\stride} {t_1}' \Or {t_2}',{\sigma}''}
  {\Value\ ({t_1}',{\sigma}') = \bot \land \Value\ ({t_2}',{\sigma}'') = \bot }


\newrule{S-Xor}
  { }
  {e_1 \Xor e_2,{\sigma} {\stride} e_1 \Xor e_2,{\sigma}}
  {}

\newrule{S-Eval}
  {e,{\sigma} {\eval} {e}',{\sigma}'  \Quad
   e',{\sigma}' {\stride} {e}'',{\sigma}''}
  {e,{\sigma} {\stride} {e}'',{\sigma}''}
  {e \neq {e}'}


%% Normalisation %%

\newmacro{RelationN}
  {e,\sigma \normalise t,\sigma'}


\newrule{N-Done}
  { {e,{\sigma} {\eval} {t},{\sigma}'}\Quad
   { {t,\sigma' \stride t',\sigma''}\Quad
   {\sigma'=\sigma'' \land t=t'}}}
  {e,{\sigma} {\normalise} {t},{\sigma}'}
  {}
    %[\sigma'=\sigma'' \land t=t']

\newrule{N-Repeat}
  { e,{\sigma} {\eval} {t},{\sigma}'\Quad
    t,\sigma' \stride t',{\sigma}''\Quad
    t',\sigma'' \normalise t'',\sigma'''}
  {e,{\sigma} {\normalise} {t}'',{\sigma}'''}
  {{\sigma}'\neq {\sigma}''\lor {t}\neq {t}'}


%% Handling %%

\newmacro{RelationH}
  {t,\sigma \handle{i} t',\sigma'}


\newrule{H-Change}
  {v,v':\beta}
  {\Edit v,{\sigma} \xrightarrow[]{v'} \Edit v',{\sigma}}
  {}

\newrule{H-Fill}
  { v:\beta}
  {\Enter \beta,{\sigma} \xrightarrow[]{v} \Edit v,{\sigma}}
  {}

\newrule{H-Update}
  {\sigma(l),v:\beta }
  {\Update l,{\sigma} \xrightarrow[]{v} \Update l,{\sigma}[l \mapsto v]}
  {}

\newrule{H-PassThen}
  {t_1,\sigma \xrightarrow[]{i} {t_1'},\sigma'}
  {t_1 \Then e_2,\sigma \xrightarrow[]{i} {t_1'} \Then e_2,\sigma'}
  {}

\newrule{H-PassNext}
  {t_1,\sigma \xrightarrow[]{i} {t_1'},\sigma'}
  {t_1 \Next e_2,\sigma \xrightarrow[]{i} {t_1'} \Next e_2,\sigma'}
  {}

\newrule{H-Next}
  { {e_2\ {v_1},\sigma {\normalise} {t_2},{\sigma}'}}
  {t_1 \Next e_2,\sigma \xrightarrow[]{\Continue} {t_2},{\sigma}'}
  {\Value\ {(t_1,\sigma)} = {v_1} \land \neg\Failing\ ({t_2},{\sigma}')}

\newrule{H-FirstAnd}
  {t_1,\sigma \xrightarrow[]{i} {t_1}',{\sigma}'}
  {t_1 \And t_2,\sigma \xrightarrow[]{\First i} {t_1}' \And t_2,{\sigma}'}
  {}

\newrule{H-SecondAnd}
  {t_2,\sigma \xrightarrow[]{i} {t_2}',{\sigma}'}
  {t_1 \And t_2,\sigma \xrightarrow[]{\Second i} t_1 \And {t_2}',{\sigma}'}
  {}


\newrule{H-FirstOr}
  {t_1,\sigma \xrightarrow[]{i} {t_1}',{\sigma}'}
  {t_1 \Or t_2,\sigma \xrightarrow[]{\First i} {t_1}' \Or t_2,{\sigma}'}
  {}

\newrule{H-SecondOr}
  {t_2,\sigma \xrightarrow[]{i} {t_2}',{\sigma}' }
  {t_1 \Or t_2,\sigma \xrightarrow[]{\Second i} t_1 \Or {t_2}',{\sigma}'}
  {}


\newrule{H-PickLeft}
  {e_1,\sigma \normalise {t_1},{\sigma}'}
  {e_1 \Xor e_2,\sigma \xrightarrow[]{\Left} {t_1},{\sigma}'}
  {\neg\Failing\ ({t_1},{\sigma}')}

\newrule{H-PickRight}
  {e_2,\sigma {\normalise} {t_2},{\sigma}'}
  {e_1 \Xor e_2,\sigma \xrightarrow[]{\Right} {t_2},{\sigma}'}
  { \neg\Failing\ ({t_2},{\sigma}')}



%% Driving %%

\newmacro{RelationI}
  {t,\sigma \interact{i} t',\sigma'}



\newrule{I-Handle}
  {t,\sigma \xrightarrow[]{i} {t}',{\sigma}' \Quad
   {t}',{\sigma}' {\normalise} {t}'',{\sigma}''}
  {t,\sigma \interact{i} {t}'',{\sigma}''}
  {}

% !TEX root=main.tex


%% Language %%%%%%%%%%%%%%%%%%%%%%%%%%%%%%%%%%%%%%%%%%%%%%%%%%%%%%%%%%%%%%%%%%%%

\newmacro{G-Language}{
  \begin{grammar}
    Expressions
      & e    &::= & \lambda x:\tau.\ e   & – abstraction \\
      &      &\mid& e_1\ e_2             & – application \\
      &      &\mid& x                    & – variable \\
      &      &\mid& s                    & – symbol \\
      &      &\mid& c                    & – constant \\
    \addlinespace
      &      &\mid& u\ e_1               & – unary operation \\
      &      &\mid& e_1\ o\ e_2          & – binary operation \\
      &      &\mid& \If{e_1}{e_2}{e_3}   & – conditional \\
    \addlinespace
      &      &\mid& \unit                & – unit \\
      &      &\mid& \tuple{e_1, e_2}     & – pair \\
      &      &\mid& \Fst e               & – first projection \\
      &      &\mid& \Snd e               & – second projection \\
    \addlinespace
      &      &\mid& \Ref e               & – reference \\
      &      &\mid& !e                   & – dereference \\
      &      &\mid& e_1 := e_2           & – assignment \\
      % &      &\mid& e_1; e_2             & – sequence \\
      &      &\mid& l                    & – location \\
    \addlinespace
      &      &\mid& p                    & – pretask \\
    \addlinespace
    Constants
      & c    &::= & B                    & – boolean \\
      &      &\mid& I                    & – integer \\
      &      &\mid& S                    & – string \\
    \addlinespace
    Unary operations
      & u    &::= & \lnot                & – logical \\
      &      &\mid& -                    & – numerical \\
      &      &\mid& \Len                 & – sequential \\
    \addlinespace
    Binary operations
      & o    &::= & \land \mid \lor                                     & – logical \\
      &      &\mid& < \mid \le \mid \equiv \mid \nequiv \mid \ge \mid > & – equational \\
      &      &\mid& + \mid - \mid \times \mid /                         & – numerical \\
      &      &\mid& \pp                                                 & – sequential \\
  \end{grammar}
}

\newmacro{G-Language-Compact}{
  \begin{grammar}
    \noalign{Expressions}
    \addlinespace
    & \sime &::= & \lambda x:\tau.\ \sime  \Mid  \sime_1\ \sime_2                  & – abstraction, application \\
    &       & \mid  & x  \Mid  c \Mid \unit                        & – variable, constant, unit \\
    &       & \mid  & u\ \sime_1 \Mid \sime_1\ o\ \sime_2                      & – unary, binary operation \\
    &       & \mid  & \If{\sime_1}{\sime_2}{\sime_3}                           & – conditional \\
    &       & \mid  & \tuple{\sime_1, \sime_2}  \Mid  \Fst \sime  \Mid  \Snd \sime & – pair, projections \\
    &       & \mid  & [\ ]_\beta \Mid \sime_1 :: \sime_2                   & – nil, cons \\
    &       & \mid  & \Head\ \sime \Mid \Tail\ \sime                       & – first element, list tail \\
    &       & \mid  & \Ref \sime  \Mid  !\sime  \Mid  \sime_1 := \sime_2  \Mid  l  & – references, location \\
    &       & \mid  & \tilde{p} \Mid \highlight{s}                         & – pretask, symbol \\
    \\
    \noalign{Constants}
    \addlinespace
    & c& ::= &  B  \Mid  I  \Mid  S                                & – boolean, integer, string \\
    \\
    \noalign{Unary Operations}
    \addlinespace
    & \highlight{u} &::= &  \lnot \Mid - \Mid \Len \Mid \Uniq                  & – not, negate, length, unique \\
    \\
    \noalign{Binary Operations}
    \addlinespace
    & \highlight{o} &::= & < \Mid \le \Mid \equiv \Mid \nequiv \Mid \ge \Mid > & – equational \\
    &       & \mid  & + \Mid - \Mid \times \Mid /                  & – numerical \\
    &       & \mid  & \land  \Mid \lor                             & – conjunction, disjunction \\
    &       & \mid  & \pp  \Mid \in                                & – append, elementhood \\
  \end{grammar}
}


\newmacro{G-Pretasks}{
  \begin{grammar}
    Pretasks
      & \tilde{p}    &::= & \Edit \sime              & – valued editor \\
      % &      &\mid& \View e              & – valued read-only editor \\
      &      &\mid& \Enter \beta          & – unvalued editor \\
      &      &\mid& \Update \sime            & – shared editor \\
      % &      &\mid& \Watch e             & – shared read-only editor \\
    \addlinespace
      &      &\mid& e_1 \Then e_2        & – step \\
      &      &\mid& e_1 \Next e_2        & – user step \\
    \addlinespace
      &      &\mid& e_1 \And e_2         & – composition \\
    \addlinespace
      &      &\mid& e_1 \Or e_2          & – choice \\
      &      &\mid& e_1 \Xor e_2         & – user choice \\
    \addlinespace
      &      &\mid& \Fail                & – fail task \\
  \end{grammar}
}

\newmacro{G-Pretasks-Compact}{
  \begin{grammar}
    \noalign{Pretasks}
    \addlinespace
      & \tilde{p}    &::= & \Edit \sime \Mid \highlight{\Enter \beta} \Mid \Update \sime            & – editors: valued, unvalued, shared \\
      &      &\mid& \sime_1 \Then \sime_2 \Mid \sime_1 \Next \sime_2                   & – steps: internal, external \\
      &      &\mid& \Fail \Mid \sime_1 \And \sime_2                            & – fail, composition \\
      &      &\mid& \sime_1 \Or \sime_2 \Mid \sime_1 \Xor \sime_2                      & – choice: internal, external\\
  \end{grammar}
}


\newmacro{G-Types}{
  \begin{grammar}
    Types
      & \tau &::= & \tau_1 \to \tau_2    & – function type \\
      &      &\mid& \Reference \tau      & – reference type \\
      &      &\mid& \Task \tau           & – task type \\
      &      &\mid& \beta                & – basic type \\
      % &      &\mid& \alpha               & – universal type \\
    Basic types
      &\beta &::= & \tau_1 \times \tau_2 & – product type \\
      &      &\mid& \text{List} \beta    & - list type\\
      &      &\mid& \Unit                & – unit type \\
      &      &\mid& \Bool                & – boolean type \\
      &      &\mid& \Int                 & – integer type \\
      &      &\mid& \String              & – string type \\
  \end{grammar}
}

\newmacro{G-Types-Compact}{
  \begin{grammar}
    \noalign{Types}
    \addlinespace
      & \tau  &  ::= & \tau_1 \to \tau_2 \Mid \beta                      & – function, basic \\
      &       & \mid & \Reference \tau \Mid \Task \tau                   & – reference, task \\
      \\
    \noalign{Basic types}
    \addlinespace
      & \beta &  ::= & \highlight{\beta_1 \times \beta_2 \Mid \List \beta} \Mid \Unit  & – product, list, unit \\
      &       & \mid & \Bool \Mid \Int \Mid \String                      & – boolean, integer, string \\
  \end{grammar}
}


\newmacro{G-Values}{
  \begin{grammar}
    Values
      & v    &::= & \lambda x:\tau.\ e   & – abstraction \\
      &      &\mid& c                    & – constant \\
      &      &\mid& l                    & – location \\
    \addlinespace
      &      &\mid& s                    & – symbol \\
      &      &\mid& u\ v                 & – symbolic unary operation \\
      &      &\mid& v_1\ o\ v_2          & – symbolic binary operation \\
    \addlinespace
      &      &\mid& \tuple{v_1, v_2}     & – pair value \\
      &      &\mid& \unit                & – unit \\
    \addlinespace
      &      &\mid& t                    & – task \\
  \end{grammar}
}

\newmacro{G-Values-Compact}{
  \begin{grammar}
    \noalign{Values}
    \addlinespace
      & \simv &  ::= & \lambda x:\tau.\ \sime \Mid \tuple{\simv_1, \simv_2} \Mid \unit & – abstraction, pair, unit \\
      &   & \mid & [\ ]_\beta \Mid \simv_1 :: \simv_2 \Mid c                   & - nil, cons, constant \\
      &   & \mid & l \Mid \simt \Mid \highlight{s}                         & – location, task, symbol \\
      &   & \mid & \highlight{u\ \simv \Mid \simv_1\ o\ \simv_2}                   & – unary/binary operation \\
  \end{grammar}
}


\newmacro{G-Tasks}{
  \begin{grammar}
    Tasks
      & t    &::= & \Edit v              & – valued editor \\
      % &      &\mid& \View v              & – valued read-only editor \\
      &      &\mid& \Enter \tau          & – unvalued editor \\
      &      &\mid& \Update l            & – shared editor \\
      % &      &\mid& \Watch l             & – shared read-only editor \\
    \addlinespace
      &      &\mid& t_1 \Then e_2        & – step \\
      &      &\mid& t_1 \Next e_2        & – user step \\
    \addlinespace
      &      &\mid& t_1 \And t_2         & – composition \\
    \addlinespace
      &      &\mid& t_1 \Or t_2          & – choice \\
      &      &\mid& e_1 \Xor e_2         & – user choice \\
    \addlinespace
      &      &\mid& \Fail                & – fail task \\
  \end{grammar}
}

\newmacro{G-Tasks-Compact}{
  \begin{grammar}
    \noalign{Tasks}
    \addlinespace
      & \simt &  ::= & \Edit \simv \Mid \highlight{\Enter \beta} \Mid \Update l           & – editors \\
      &   & \mid & \simt_1 \Then \sime_2 \Mid \simt_1 \Next \sime_2                  & – steps \\
      &   & \mid & \Fail \Mid \simt_1 \And \simt_2                           & – fail, combination \\
      &   & \mid & \simt_1 \Or \simt_2 \Mid \sime_1 \Xor \sime_2                     & – choices \\
  \end{grammar}
}

\newmacro{G-Inputs}{
  \begin{grammar}
    Symbolic inputs
      & i    & ::=& $s$                  & – symbolic action \\
      &      &\mid& \First i             & – pass to first \\
      &      &\mid& \Second i            & – pass to second
  \end{grammar}
}

\newmacro{G-Inputs-Compact}{
  \begin{grammar}
    \noalign{Symbolic inputs}
    \addlinespace
      & \simi    & ::=& \tilde{a} \Mid \First \simi \Mid \Second \simi  & – symbolic action, to first, to second \\
      \\
    \noalign{Symbolic actions}
    \addlinespace
      & \tilde{a}  & ::=& \highlight{s}                     & – symbol \\
      &    &\mid& \Continue \Mid \Left \Mid \Right  & – continue, go left, go right \\
  \end{grammar}
}

\newmacro{G-CInputs}{
  \begin{grammar}
    Concrete inputs
      & i    & ::=& a              & – concrete action \\
      &      &\mid& \First i             & – pass to first \\
      &      &\mid& \Second i           & – pass to second \\
    Concrete actions
      & a & ::=& c                    & – constant \\
      &         &\mid& \Continue            & – continue with next task \\
      &         &\mid& \Left                & – go left \\
      &         &\mid& \Right               & – go right \\
  \end{grammar}
}

\newmacro{G-CInputs-Compact}{
  \begin{grammar}
    \noalign{Concrete inputs}
    \addlinespace
      & j    & ::=& a \Mid \First j \Mid \Second j             & – action, to first, to second \\
      \\
    \noalign{Concrete actions}
    \addlinespace
      & a  & ::=& c                    & – constant \\
      &      &\mid& \Continue  \Mid \Left \Mid \Right          & – continue, go left, go right \\
  \end{grammar}
}


\newmacro{G-Predicates}{
  \begin{grammar}
    Predicates
      & \phi &::= & c                    & – constant \\
      &      &\mid& s                    & – symbol \\
      &      &\mid& \Continue            & – continue\\
      &      &\mid& \Left                & – go left \\
      &      &\mid& \Right               & – go right \\
      &      &\mid& u\ \phi              & – symbolic unary operation \\
      &      &\mid& \phi_1\ o\ \phi_2    & – symbolic binary operation \\
  \end{grammar}
}

\newmacro{G-Predicates-Compact}{
  \begin{grammar}
    \noalign{Path conditions}
    \addlinespace
      & \highlight{\phi} &::= & c \Mid s              & – constant, symbol\\
      % &      &\mid& \Continue  \Mid \Left \Mid \Right          & – continue, go left, go right\\
      &      &\mid& \unit \Mid \tuple{\phi_1, \phi_2} & – unit, pairs \\
      &      &\mid& [\ ]_\beta \Mid \phi_1 :: \phi_2   & – nil, cons \\
      &      &\mid& u\ \phi \Mid \phi_1\ o\ \phi_2    & – symbolic unary/binary operation \\
  \end{grammar}
}

% !TEX root=main.tex


%% Language %%%%%%%%%%%%%%%%%%%%%%%%%%%%%%%%%%%%%%%%%%%%%%%%%%%%%%%%%%%%%%%%%%%%

\newmacro{O-Value}{
  \begin{function}
    \signature{\Value : \mathrm{Tasks} \times \mathrm{States} \rightharpoonup \mathrm{Values}} \\
    \Value(\Edit \simv, \sims)                &=& \simv \\
    \Value(\Enter \beta, \sims)            &=& \bot \\
    \Value(\Update l, \sims)              &=& \sims(l) \\
    \Value(\Fail, \sims)                  &=& \bot \\
    \Value(\simt_1 \Then \sime_2, \sims)          &=& \bot \\
    \Value(\simt_1 \Next \sime_2, \sims)          &=& \bot \\
    \Value(\simt_1 \And \simt_2, \sims)           &=& \left\{
      \begin{tabular}{ll}
        $\tuple{\simv_1, \simv_2}  $&$ \ \when\ \Value(\simt_1, \sims) = \
        \simv_1 \land \Value(\simt_2, \sims) = \simv_2 $\\
        $\bot                          $&$ \ \otherwise$
      \end{tabular}
    \right. \\
    \Value(\simt_1 \Or \simt_2, \sims)            &=& \left\{
      \begin{tabular}{ll}
        $\simv_1  $                         & $\ \when\ \Value(\simt_1, \sims) = \simv_1 $\\
        $\simv_2 $                          & $\ \when\ \Value(\simt_1, \sims) = \bot \land \Value(\simt_2, \sims) = \simv_2 $\\
        $\obox{\tuple{\simv_1, \simv_2}}{\bot}$ &$ \ \otherwise$
      \end{tabular}
    \right. \\
    \Value(\simt_1 \Xor \simt_2, \sims)           &=& \bot
  \end{function}
}

\newmacro{O-Failing}{
  \begin{function}
    \signature{\Failing : \mathrm{Tasks} \times \mathrm{States} \to \mathrm{Booleans}} \\
    \Failing(\Edit \simv,\sims)       &=& \False \\
    \Failing(\Enter \beta,\sims)   &=& \False \\
    \Failing(\Update l,\sims)     &=& \False \\
    \Failing(\Fail,\sims)         &=& \True \\
    \Failing(\simt_1 \Then \sime_2,\sims) &=& \Failing(\simt_1,\sims) \\
    \Failing(\simt_1 \Next \sime_2,\sims) &=& \Failing(\simt_1,\sims) \\
    \Failing(\simt_1 \And \simt_2,\sims)  &=& \Failing(\simt_1,\sims) \land \Failing(\simt_2,\sims) \\
    \Failing(\simt_1 \Or \simt_2,\sims)   &=& \Failing(\simt_1,\sims) \land \Failing(\simt_2,\sims) \\
    \Failing(\sime_1 \Xor \sime_2,\sims)  &=& \highlight{\bigwedge \Big( \set{\Failing(\simt_1,\sims_1') \mid \sime_1, \sims \normalise \overline{\simt_1,\sims_1'}}\ \cup} \\
                          & & \highlight{\phantom{\bigwedge \Big(}\set{\Failing(\simt_2,\sims_2') \mid \sime_2, \sims \normalise \overline{\simt_2,\sims_2'}} \Big)} \\
    % \Failing(e_1 \Xor e_2,\sims)  &=& \Failing(t_1,s_1') \land \Failing(t_2,s_2')\\
    % &&\quad \where\ e_1,\sims \normalise t_1,s_1' \mathbf{\ and\ } e_2,\sims \normalise t_2,s_2'
  \end{function}
}


\newmacro{O-Inputs}{
  \begin{function}
    \signature{\Inputs : \mathrm{Tasks} \times \mathrm{States} \to \powerset{\mathrm{Inputs}}} \\
    \Inputs(\Edit v,\sims)             &=& \set{\simv':\tau}                       \quad\where\ \Edit \simv : \Task\tau \\
    \Inputs(\Enter \beta,\sims)         &=& \set{\simv':\tau} \\
    \Inputs(\Update l,\sims)           &=& \obox{\set{\simv':\tau, \Empty}}{\set{\simv':\tau}} \quad\where\ \Update l : \Task\tau \\
    \Inputs(\Fail,\sims)               &=& \nothing \\
    \Inputs(\simt_1 \Then \sime_2,\sims)       &=& \Inputs(\simt_1,\sims) \\
    \Inputs(\simt_1 \Next \sime_2,\sims)       &=& \Inputs(\simt_1,\sims) \cup \set{\Continue \mid \Value(\simt_1, \sims) = \simv_1 \land \\
                                         && \sime_2\ \simv_1, \sims \normalise \simt_2, \sims',\phi \land \lnot\Failing(\simt_2, \sims')} \\
    \Inputs(\simt_1 \And \simt_2,\sims)        &=& \set{\First\ \simi \mid \simi \in \Inputs(\simt_1,\sims)} \cup \set{\Second\ \simi \mid i \in \Inputs(\simt_2,\sims)} \\
    \Inputs(\simt_1 \Or t_2,\sims)         &=& \set{\First\ \simi \mid \simi \in \Inputs(\simt_1,\sims)} \cup \set{\Second\ \simi \mid \simi \in \Inputs(\simt_2,\sims)} \\
    \Inputs(\sime_1 \Xor \sime_2,\sims)        &=& \set{\Left \mid \sime_1, \sims \normalise \simt_1, \sims',\phi \land \lnot\Failing(\simt_1, \sims')} \cup\\
                                         && \set{\Right \mid \sime_2, \sims \normalise \simt_2, \sims',\phi \land \lnot\Failing(\simt_2, \sims')}
  \end{function}
}




%% Journal information
%% Supplied to authors by publisher for camera-ready submission;
%% use defaults for review submission.
% \acmJournal{PACMPL}
% \acmVolume{1}
% \acmNumber{ICFP} % CONF = POPL or ICFP or OOPSLA
% \acmArticle{1}
% \acmYear{2018}
% \acmMonth{1}
% \acmDOI{} % \acmDOI{10.1145/nnnnnnn.nnnnnnn}
% \startPage{1}


\acmConference[IFL'19]{International Symposium on Implementation and Application of Functional Languages}{September 2019}{Singapore}
\acmYear{2020}
\copyrightyear{2020}

%% Copyright information
%% Supplied to authors (based on authors' rights management selection;
%% see authors.acm.org) by publisher for camera-ready submission;
%% use 'none' for review submission.
\setcopyright{none}
%\setcopyright{acmcopyright}
%\setcopyright{acmlicensed}
%\setcopyright{rightsretained}
%\copyrightyear{2018}           %% If different from \acmYear

%% Bibliography style
\bibliographystyle{ACM-Reference-Format}
%% Citation style
%% Note: author/year citations are required for papers published as an
%% issue of PACMPL.
\citestyle{acmauthoryear}   %% For author/year citations






% version.tex must define the command \version
\IfFileExists{version.tex}
  {\input{version.tex}}
  {\newcommand{\version}{unknown version}}

\hypersetup
{ pdfcreator=\version
}

\usepackage{fancyhdr}
\fancyfoot[C]{\thepage}
%\fancyfoot[R]{v.\version}






\begin{document}

%% Title information
\title{A symbolic execution semantics for TopHat}
% \title{TopHat: A calculus for modular interactive workflows}
                                        %% [Short Title] is optional;
                                        %% when present, will be used in
                                        %% header instead of Full Title.
%\titlenote{with title note}             %% \titlenote is optional;
                                        %% can be repeated if necessary;
                                        %% contents suppressed with 'anonymous'
\subtitle{Appendices}                   %% \subtitle is optional
%\subtitlenote{with subtitle note}       %% \subtitlenote is optional;
                                        %% can be repeated if necessary;
                                        %% contents suppressed with 'anonymous'


%% Author information
%% Contents and number of authors suppressed with 'anonymous'.
%% Each author should be introduced by \author, followed by
%% \authornote (optional), \orcid (optional), \affiliation, and
%% \email.
%% An author may have multiple affiliations and/or emails; repeat the
%% appropriate command.
%% Many elements are not rendered, but should be provided for metadata
%% extraction tools.

\author{Nico Naus}
%\authornote{with author1 note}          %% \authornote is optional; can be repeated if necessary
%\orcid{nnnn-nnnn-nnnn-nnnn}             %% \orcid is optional
\affiliation{
  %\position{PhD}
  \department{Computer Science}
                                        %% \department is recommended
  \institution{Open University of the Netherlands}      %% \institution is required
  \streetaddress{Valkenburgerweg 177}
  \postcode{6419 AT}
  \city{Heerlen}
  %\state{State1}
  \country{The Netherlands}
}
\email{nico.naus@ou.nl}                    %% \email is recommended

\author{Tim Steenvoorden}
%\authornote{with author1 note}          %% \authornote is optional; can be repeated if necessary
%\orcid{nnnn-nnnn-nnnn-nnnn}             %% \orcid is optional
\affiliation{
  %\position{PhD}
  \department{Software Science}
  %\department{Institute for Computing and Information Sciences}
                                        %% \department is recommended
  \institution{Radboud University}      %% \institution is required
  \streetaddress{Toernooiveld 212}
  \postcode{6525 EC}
  \city{Nijmegen}
  %\state{State1}
  \country{The Netherlands}
}
\email{tim@cs.ru.nl}                     %% \email is recommended

\author{Markus Klinik}
%\authornote{with author1 note}          %% \authornote is optional; can be repeated if necessary
%\orcid{nnnn-nnnn-nnnn-nnnn}             %% \orcid is optional
\affiliation{
  %\position{PhD}
  \department{Software Science}
  %\department{Institute for Computing and Information Sciences}
                                        %% \department is recommended
  \institution{Radboud University}
                                        %% \institution is required
  \streetaddress{Toernooiveld 212}
  \postcode{6525 EC}
  \city{Nijmegen}
  %\state{State1}
  \country{The Netherlands}
}
\email{m.klinik@cs.ru.nl}               %% \email is recommended




%% Paper note
%% The \thanks command may be used to create a "paper note" ---
%% similar to a title note or an author note, but not explicitly
%% associated with a particular element.  It will appear immediately
%% above the permission/copyright statement.
%\thanks{with paper note}                %% \thanks is optional
                                        %% can be repeated if necesary
                                        %% contents suppressed with 'anonymous'


%% Abstract
%% Must come before \maketitle command
% \begin{abstract}
%   \input{sections/abstract}
% \end{abstract}
%
% \begin{teaserfigure}
%    \includegraphics[width=\textwidth]{figures/declrequest-part.pdf}
%    \caption{This is a teaser}
%    \label{fig:teaser}
% \end{teaserfigure}

%% 2012 ACM Computing Classification System (CSS) concepts
%% Generate at 'http://dl.acm.org/ccs/ccs.cfm'.

%% End of generated code


%% Keywords
%% comma separated list, optional
%\keywords{workflow, dataflow, visual programming, program generation}


%% Note: \maketitle command must come after title commands, author
%% commands, abstract environment, Computing Classification System
%% environment and commands, and keywords command.
\maketitle

%% Acknowledgments
% \begin{acks}                            %% acks environment is optional
%                                         %% contents suppressed with 'anonymous'
%   %% Commands \grantsponsor{<sponsorID>}{<name>}{<url>} and
%   %% \grantnum[<url>]{<sponsorID>}{<number>} should be used to
%   %% acknowledge financial support and will be used by metadata
%   %% extraction tools.
%   \small
%   \input{sections/acknowledgements}
% \end{acks}
%

%% Bibliography
% \bibliography{bibliography}


%% Appendix
%\appendix


% \input{sections/proofs/unsat}
% \input{sections/proofs/stuck}
% !TEX root=../../appendix.tex

\section{Complete symbolic semantics}

\subsection{Symbolic evaluation rules}
\label{sec:symbolic-evaluation-rules}

\begin{gather*}
  % \small
  \boxed{\RelationSE} \Break
  \userule{SE-Value}\Quad
  \userule{SE-Pair} \Quad
  \userule{SE-First}\Quad
  \userule{SE-Second}\Break
  \userule{SE-Cons}\Quad
  \userule{SE-Head}\Quad
  \userule{SE-Tail}\Break
  \userule{SE-App} \Break
  \userule{SE-If} \Quad
  \userule{SE-Ref} \Quad
  \userule{SE-Deref} \Break
  \userule{SE-Assign} \Quad
  \userule{SE-Edit} \Quad
  \userule{SE-Enter}\Quad
  \userule{SE-Update}\Break
  \userule{SE-Then}\Quad
  \userule{SE-Next}\Quad
  \userule{SE-And}\Break
  \userule{SE-Or} \Quad
  \userule{SE-Xor}\Quad
  \userule{SE-Fail}
\end{gather*}
\onecolumn
\subsection{Symbolic striding rules}

\begin{gather*}
  % \small
  \boxed{\RelationSS} \Break
  \userule{SS-ThenStay} \Quad
  \userule{SS-ThenFail} \Break
  \userule{SS-ThenCont} \Quad
  \userule{SS-OrLeft} \Break
  \userule{SS-OrRight} \Break
  \userule{SS-OrNone} \Break
  \userule{SS-Edit} \Quad
  \userule{SS-Fill} \Quad
  \userule{SS-Update} \Quad
  \userule{SS-Fail} \Break
  \userule{SS-Xor} \Quad
  \userule{SS-Next} \Quad
  \userule{SS-And}
\end{gather*}

\subsection{Symbolic normalisation rules}

\begin{gather*}
  % \small
  \boxed{\RelationSN} \Break
  \userule{SN-Done} \Quad
  \userule{SN-Repeat}
\end{gather*}


\subsection{Symbolic handling rules}
\label{sec:symbolic-handling-rules}

\begin{gather*}
  % \small
  \boxed{\RelationSH} \Break
  \userule{SH-Change} \Quad
  \userule{SH-Fill} \Quad
  \userule{SH-Update}\Break
  \userule{SH-PassNext} \Quad
  \userule{SH-PassNextFail}\Break
  \userule{SH-Next} \Break
  \userule{SH-PassThen} \Quad
  \userule{SH-Pick} \Break
  \userule{SH-PickLeft} \Quad
  \userule{SH-PickRight}\Break
  \userule{SH-And}\Quad
  \userule{SH-Or}
\end{gather*}

\subsection{Symbolic driving rules}

\begin{gather*}
  % \small
  \boxed{\RelationSI}\Break
  \userule{SI-Handle}
\end{gather*}

% !TEX root=../../appendix.tex

\section{TopHat semantics}

\subsection{Typing rules}

\begin{gather*}
  \boxed{\RelationT} \Break
  \userule{T-ConstBool} \Quad
  \userule{T-ConstInt} \Quad
  \userule{T-ConstString} \Quad
  \userule{T-Unit}\Quad
  \userule{T-Var} \Quad
  \userule{T-Loc} \Break
  \userule{T-Pair} \Quad
  \userule{T-First} \Quad
  \userule{T-Second} \Quad
  \userule{T-ListEmpty} \Quad
  \userule{T-ListCons} \Break
  \userule{T-ListHead} \Quad
  \userule{T-ListTail}\Quad
  \userule{T-Abs} \Quad
  \userule{T-App} \Quad
  \userule{T-Ref} \Break
  \userule{T-If} \Quad
  \userule{T-Deref} \Quad
  \userule{T-Assign}\Quad
  \userule{T-Xor} \Break
  \userule{T-Edit}\Quad
  \userule{T-Enter}\Quad
  \userule{T-Update}\Quad
  \userule{T-Fail}\Quad
  \userule{T-Then}\Break
  \userule{T-Next}\Quad
  \userule{T-And}\Quad
  \userule{T-Or}
\end{gather*}

\subsection{Evaluation rules}

\begin{gather*}
  \boxed{\RelationE} \Break
  \userule{E-App} \Quad
  \userule{E-IfTrue} \Quad
  \userule{E-Ref} \Break
  \userule{E-IfFalse} \Quad
  \userule{E-Deref} \Quad
  \userule{E-Value} \Quad
  \userule{E-Assign} \Quad
  \userule{E-Pair} \Break
  \userule{E-First} \Quad
  \userule{E-Second}\Quad
  \userule{E-Cons}\Quad
  \userule{E-Head}\Quad
  \userule{E-Tail}\Break
  \userule{E-Edit} \Quad
  \userule{E-Update} \Quad
  \userule{E-Then} \Quad
  \userule{E-Next} \Quad
  \userule{E-And} \Break
  \userule{E-Or}
\end{gather*}

\subsection{Striding rules}

\begin{gather*}
  \boxed{\RelationS} \Break
  \userule{S-ThenStay} \Quad
  \userule{S-ThenFail} \Break
  \userule{S-ThenCont}\Quad
  \userule{S-OrLeft} \Break
  \userule{S-OrRight} \Break
  \userule{S-OrNone}\Quad
  \userule{S-Edit} \Quad \userule{S-Fill} \Quad \userule{S-Update} \Break
  \userule{S-Fail} \Quad \userule{S-Xor}\Quad
  \userule{S-Next} \Quad
  \userule{S-And}
\end{gather*}

\subsection{Normalisation rules}

\begin{gather*}
  \boxed{\RelationN} \Break
  \userule{N-Done} \Quad
  \userule{N-Repeat}
\end{gather*}

\subsection{Handling rules}

\begin{gather*}
  \boxed{\RelationH} \Break
  \userule{H-Change} \Quad
  \userule{H-Fill} \Quad
  \userule{H-Update}\Break
  \userule{H-Next} \Quad
  \userule{H-PickLeft} \Quad
  \userule{H-PickRight}\Quad
  \userule{H-PassThen} \Break
  \userule{H-PassNext} \Quad
  \userule{H-FirstAnd} \Quad \userule{H-SecondAnd} \Quad
  \userule{H-FirstOr}  \Quad \userule{H-SecondOr}
\end{gather*}


\subsection{Driving rules}

\begin{gather*}
  \boxed{\RelationI} \Break
  \userule{I-Handle}
\end{gather*}

% !TEX root=../../appendix.tex


% \begin{proof}[Proof of Theorem~\ref{thm:sound}]
%     \fixme{we cannot prove this, definition of drive function is incorrect}
% \end{proof}

\section{Soundness proofs}
\label{sec:soundness-proofs}

\subsection{Proof of soundness of symbolic evaluation semantics}
\begin{proof}
  We prove Lemma 6.5 by induction over the derivation of the symbolic evaluation $e,\sigma\simeval\overline{\sime,\sims,\phi}$.

  \case{\refrule{SE-Value}}
    {Since this case does not generate constraints, any $M$ will do.
    Since neither the state, nor the expression is altered by the evaluation rule \refrule{E-Value},
    this case holds trivially.
    }

\case{\refrule{SE-Fail}}
      {Since this case does not generate constraints, any $M$ will do.
      Since neither the state, nor the expression $\Fail$ is altered by the evaluation rule \refrule{E-Fail},
      this case holds trivially.
      }
  \case{\refrule{SE-Pair}}
    {For all mappings $M$ such that $M(\phi_1\wedge\phi_2)$,we need to demonstrate that
    $\tuple{e_1,e_2},\sigma\eval\tuple{v_1,v_2},\sigma''$ with
    $M\tuple{\simv_1,\simv_2} \equiv \tuple{v_1,v_2}$ and $M\sims''\equiv\sigma''$.

    From the induction hypothesis, we obtain the following.\\
    $\forall M_1 . \sime_1,\sims\simeval\overline{\simv_1,\sims',\phi_1}\land M_1\phi_1 \implies e_1,\sigma \eval v_1,\sigma'\land  M_1\simv_1\equiv v_1 \land  M_1\sims'\equiv\sigma'$ and
    $\forall M_2 . M_2\phi_2 \implies e_2, \sigma'\eval v_2,\sigma''\land M_2 \simv_2\equiv v_2\land M_2\sims''\equiv\sigma''$.

    Note that we have omitted from the second application of the induction hypothesis, the requirement that the symbolic step exists.
    The fact that this step exists is obtained from \refrule{SE-pair} and omitted to increase readability of this and any following proofs.

    Since $M$ satisfies both $\phi_1$ and $\phi_2$,
    we obtain from \refrule{E-Pair} and the induction steps above that $\tuple{e_1,e_2},\sigma\eval\tuple{v_1,v_2},\sigma''$, $M\tuple{\simv_1,\simv_2} \equiv \tuple{{v_1},{v_2}}$ and $M\sims''\equiv\sigma''$.
    }

\case{\refrule{SE-First}}
{
  For all mappings $M$ such that $M\phi$, we need to show that
  $\Fst e,\sigma\eval v_1,\sigma'$ with
  $M \simv_1\equiv {v_1}$ and $M\sims'\equiv{\sigma'}$.

  From the induction hypothesis, we obtain the following.\\
  $\forall M_1 . M_1\phi \implies e,\sigma\eval\tuple{v_1,v_2},\sigma'\land M_1\tuple{\simv_1,\simv_2}\equiv \tuple{{v_1},{v_2}}\land M_1\sims'\equiv{\sigma'}$

  Since $M$ satisfies $\phi$,
  we obtain from \refrule{E-First} and the induction step above that $\Fst e,\sigma\eval v_1,\sigma'$ with $M \simv_1\equiv v_1s$ and $M\sims'\equiv\sigma'$.
  }

\case{\refrule{SE-Second}}
{
  For all mappings $M$ such that $M\phi$, we need to show that
  $\Snd e,\sigma\eval v_2,\sigma'$ with $M \simv_2\equiv {v_2}$ and $M\sims'\equiv{\sigma'}$.

  From the induction hypothesis, we obtain the following.\\
  $\forall M_1 . M_1\phi \implies e, \sigma \eval\tuple{{v_1},{v_2}},{\sigma'}\land M_1\tuple{\simv_1,\simv_2}\equiv \tuple{{v_1},{v_2}}\land M_1\sims'\equiv{\sigma'}$

  Since $M$ satisfies $\phi$,
  we obtain from \refrule{E-Second} and the induction step above that $\Snd e,\sigma\eval v_2,\sigma'$ with
  $M \simv_2\equiv {v_2}$ and $M\sims'\equiv{\sigma'}$.
  }

\case{\refrule{SE-Cons}}
  {
  For all mappings $M$ such that $M\phi$, we need to demonstrate that
  $e_1 :: e_2,\sigma\eval v_1 :: v_2,\sigma''$ with
  $M \simv_1::\simv_2\equiv v_1 :: v_2$ and $M\sims''\equiv \sigma''$.

  From the induction hypothesis, we obtain the following.\\
  $\forall M_1 .  M_1\phi_1 \implies e_1, \sigma \eval v_1, \sigma' \land  M_1\simv_1\equiv{v_1} \land  M_1\sims'\equiv{\sigma'}$ and
  $\forall M_2 . M_2\phi_2 \implies e_2,\sigma' \eval v_2,\sigma'' \land M_2\simv_2\equiv{v_2} \wedge M_2\sims''\equiv{\sigma''}$

  Since $M$ satisfies both $\phi_1$ and $\phi_2$,
  we obtain from \refrule{E-Cons} and the induction steps above that $e_1 :: e_2,\sigma\eval v_1 :: v_2,\sigma''$ with $M(\simv_1::\simv_2) \equiv {v_1}::{v_2}$ and $M \sims'' \equiv{\sigma''}$.
  }

\case{\refrule{SE-Head}}
  {For all mappings $M$ such that $M\phi$, we need to show that
  $\Head e,\sigma\eval v_1,\sigma'$ with
  $M \simv_1\equiv {v_1}$ and $M\sims'\equiv{\sigma'}$.

  From the induction hypothesis, we obtain the following.\\
  $\forall M_1 . M_1\phi \implies e, \sigma\eval v_1 :: v_2 , \sigma' \land M_1 (\simv_1::\simv_2)\equiv {v_1}::{v_2}\land M_1\sims'\equiv{\sigma'}$

  Since $M$ satisfies $\phi$,
  we obtain from \refrule{E-Head} and the induction step above that $\Head e,\sigma{\eval}{v_1},\sigma'$ with
  $M \simv_1\equiv {v_1}$ and $M\sims'\equiv\sigma'$.}

\case{\refrule{SE-Tail}}
  { For all mappings $M$ such that $M\phi$, we need to show that
    $\Tail e,\sigma\eval v_2,\sigma'$ with
    $M \simv_2\equiv{v_2}$ and $M\sims'\equiv{\sigma'}$.

    From the induction hypothesis, we obtain the following.\\
    $\forall M_1 . M_1\phi \implies e, \sigma\eval {v_1}::{v_2},{\sigma'}\land M_1(\simv_1::\simv_2)\equiv {v_1}::{v_2}\land M_1\sims'\equiv{\sigma'}$

    Since $M$ satisfies $\phi$,
    we obtain from \refrule{E-Tail} and the induction step above that $\Tail e,\sigma{\eval}{v_2},\sigma'$ with
    $M \simv_2\equiv {v_2}$ and $M\sims'\equiv{\sigma'}$.}

\case{\refrule{SE-App}}
  {For all mappings $M$ such that $M(\phi_1 \land \phi_2\land \phi_3)$, we need to demonstrate that
  $e_1e_2,\sigma\eval v_1,\sigma'''$ with
   $M \simv_1 \equiv {v_1}$ and $M\sims'''\equiv{\sigma'''}$.

  From the induction hypothesis, we obtain the following.\\
  $\forall M_1 . M_1\phi_1 \implies e_1,\sigma{\eval}\lambda x : \tau.{e_1'},{\sigma'}
  \land M_1\lambda x : \tau.\sime_1' \equiv \lambda x : \tau.{e_1'} \land M_1\sims'\equiv{\sigma'}$
  and
  $\forall M_2 . M_2\phi_2 \implies e_2,\sigma'{\eval}{v_2},{\sigma''}
  \land M_2 \simv_2 \equiv {v_2} \land M_2\sims'' \equiv{\sigma''}$\\
  and
  $\forall M_3 . M_3\phi_3 \implies e_1'[x\mapsto {v_2}],\sigma''{\eval}{v_1},{\sigma'''}
  \land M_3 \simv_1\equiv {v_1} \land M_3 \sims'''\equiv{\sigma'''}$.

  Since $M$ satisfies $\phi_1$, $\phi_2$ and $\phi_3$, we obtain from \refrule{E-App} and the induction steps above that $e_1e_2,\sigma\eval v_1,\sigma'''$ with $M \simv_1 \equiv {v_1}$ and $M\sims'''\equiv{\sigma'''}$.
  }

\case{\refrule{SE-If}}
   {For all mappings $M$
   such that $M(\phi_1\land \phi_2 \land \simv_1)$, we need to demonstrate that
   $\If{e_1}{e_2}{e_3},\sigma\eval v_2,\sigma''$ with
   $M \simv_2 = {v_2}$ and $M\sims''=\sigma''$.

   From the induction hypothesis, we obtain the following.\\
   $\forall M_1 . M_1\phi_1 \implies e_1,\sigma\eval v_1,\sigma'
   \land M_1\simv_1 \equiv {v_1} \land M_1\sims'\equiv{\sigma'}$
   and
   $\forall M_2 . M_2\phi_2 \implies e_2,\sigma'\eval v_2,\sigma''
   \land M_2 \simv_2 \equiv {v_2} \land M_2\sims'' \equiv{\sigma''}$.

   Since $M$ satisfies $\phi_1$, $\phi_2$ and $\simv_1$, we know that $v_1=\True$.\\
   From \refrule{E-IfTrue} and the induction steps above, we obtain that
   $\If{e_1}{e_2}{e_3},\sigma\eval v_2,\sigma''$ with $M\simv_2={v_2}$ and $M\sims''={\sigma''}$.

   For all mappings $M$ such that $M(\phi_1\land \phi_3 \land \neg \simv_1)$,
   we need to demonstrate that
   $\If{e_1}{e_2}{e_3},\sigma\eval v_3,\sigma''$ with
   $M \simv_3 = {v_3}$ and $M\sims''={\sigma''}$.

   From the induction hypothesis, we obtain the following.\\
   $\forall M_1 . M_1\phi_1 \implies e_1,\sigma\eval{v_1},{\sigma'}
   \land M_1\simv_1 \equiv {v_1} \land M_1\sims'\equiv{\sigma'}$
   and
   $\forall M_3 . M_3\phi_3 \implies e_3,\sigma'\eval{v_3},{\sigma''}
   \land M_3 \simv_3 \equiv {v_3} \land M_3\sims'' \equiv{\sigma''}$.

   Since $M$ satisfies $\phi_1$, $\phi_3$ and $\neg\simv_1$, we know that $v_1=\False$.\\
   From \refrule{E-IfFalse} and the induction steps above, we obtain that
    $\If{e_1}{e_2}{e_3},\sigma\eval v_3,\sigma''$
   with $M\simv_3={v_3}$ and $M\sims''={\sigma''}$.
  }

\case{\refrule{SE-Ref}}
  {For all mappings $M$ such that $M\phi$,
  we need to demonstrate that
  $\Ref e,\sigma\eval l,\sigma'[l\mapsto v]$ with
  $M l \equiv l$ and $M\sims'[l\mapsto \simv]\equiv{\sigma'}[l\mapsto{v}]$.

  From the induction hypothesis, we obtain the following.\\
  $\forall M_1 .  M_1\phi \implies e, \sigma {\eval}{v},{\sigma'}\land  M_1\simv\equiv {v} \wedge  M_1\sims'\equiv{\sigma'}$.

  Since $M$ satisfies $\phi$,
  we obtain from \refrule{E-Ref} and the induction steps above that $\Ref e,\sigma\eval l,\sigma'[l\mapsto v]$.

  We assume that the assignment of location references happens in a deterministic manner, and that we can therefore conclude that exactly the same $l$ is used in both cases. Since $l$ cannot contain any symbols, $M l \equiv l$ holds trivially.

  This, together with $M \sims' \equiv{\sigma'}$ obtained from the induction hypothesis, we can conclude that $M\sims'[l\mapsto \simv]\equiv{\sigma'}[l\mapsto{v}]$.
  }

\case{\refrule{SE-Deref}}
  {For all mappings $M$ such that $M\phi$, we need to demonstrate that $!e,\sigma\eval\sigma'(l),\sigma'$ with
  $M \sims'(l) \equiv {\sigma'}(l)$ and $M\sims'\equiv{\sigma'}$.

  From the induction hypothesis, we obtain the following.\\
  $\forall M_1 .  M_1\phi \implies e, \sigma {\eval}l,{\sigma'}\land  M_1l\equiv l \land  M_1\sims'\equiv{\sigma'}$.

  Since $M$ satisfies $\phi$,
  we obtain from \refrule{E-Deref} and the induction step above that $!e,\sigma {\eval}\sigma'(l),{\sigma'}$ with $M \sims'(l) \equiv {\sigma'}(l)$
  and $M\sims'\equiv{\sigma'}$.
}

\case{\refrule{SE-Assign}}
  {
  For all mappings $M$ such that $M(\phi_1\wedge \phi_2)$,
  we need to demonstrate that\\
  $e_1 := e_2,\sigma\eval\unit,\sigma''[l\mapsto v_2]$ with
  $M\unit \equiv \unit$, which holds true trivially,
  and $M\sims''[l\mapsto \simv_2]\equiv{\sigma''}[l\mapsto{v_2}]$.

  From the induction hypothesis, we obtain the following.\\
  $\forall M_1 .  M_1\phi_1 \implies e_1, \sigma {\eval}l,{\sigma'}\land  M_1 l\equiv l \land  M_1\sims'\equiv{\sigma'}$ and
  $\forall M_2 . M_2\phi_2 \implies e_2,\sigma' {\eval}{v_2},{\sigma''}\land M_2\simv_2\equiv {v_2} \land M_2\sims''\equiv{\sigma''}$

  Since $M$ satisfies both $\phi_1$ and $\phi_2$, we obtain from \refrule{E-Assign} and the induction steps above that $e_1 := e_2,\sigma\eval\unit,\sigma''[l\mapsto v_2]$ with $M\sims''[l\mapsto \simv_2]\equiv{\sigma''}[l\mapsto{v_2}]$.
  }

\case{\refrule{SE-Edit}}
  {For all mappings $M$ such that $M\phi$,
  we need to demonstrate that $\Edit e,\sigma\eval\Edit v,\sigma'$ with
  $M \Edit \simv \equiv \Edit {v}$ and $M\sims'\equiv{\sigma'}$.

  From the induction hypothesis, we obtain the following.
  $\forall M_1 .  M_1\phi \implies e, \sigma {\eval}{v},{\sigma'}\land  M_1\simv\equiv {v} \land  M_1\sims'\equiv{\sigma'}$.

  Since $M$ satisfies $\phi$,
  we obtain from \refrule{E-Edit} and the induction step above that $\Edit e,\sigma\eval\Edit v,\sigma'$ with $M \Edit \simv\equiv \Edit {v}$ and $ M\sims'\equiv{\sigma'}$.

  }

\case{\refrule{SE-Update}}
  {For all mappings $M$ such that $M\phi$, we need to demonstrate that $\Update e,\sigma\eval\Update l,\sigma'$ with
  $M \Update l \equiv \Update l$ and $M\sims'\equiv{\sigma'}$.

  From the induction hypothesis, we obtain the following.
  $\forall M_1 .  M_1\phi \implies e, \sigma {\eval}l,{\sigma'}\land  M_1 l\equiv l \land  M_1\sims'\equiv{\sigma'}$.

  Since $M$ satisfies $\phi$,
  we obtain from \refrule{E-Update} and the induction step above that $\Update e,\sigma\eval\Update l,\sigma'$ with $M \Update l\equiv \Update l$ and $M \sims' \equiv{\sigma'}$.

  }

\case{\refrule{SE-Then}}
  {For all mappings $M$ such that $M\phi$, we need to demonstrate that
  $e_1\Then e_2,\sigma\eval t_1\Then e_2,\sigma'$ with
  $M \simt_1\Then \sime_2 \equiv {t_1}\Then e_2$ and $M\sims'\equiv{\sigma'}$.

  From the induction hypothesis, we obtain the following.
  $\forall M_1 .  M_1\phi \implies e, \sigma {\eval}{t_1},{\sigma'}\land  M_1 \simt_1\equiv {t_1} \land  M_1\sims'\equiv{\sigma'}$.

  Since $M$ satisfies $\phi$,
  we obtain from \refrule{E-Then} and the induction step above that $e_1\Then e_2,\sigma\eval t_1\Then e_2,\sigma'$ with $M \simt_1\Then \sime_2 \equiv {t_1}\Then e_2$ and $M \sims' \equiv\sigma'$.

  }

\case{\refrule{SE-Next}}
  {For all mappings $M$ such that $M\phi$, we need to demonstrate that
  $e_1\Next e_2,\sigma\eval t_1\Next e_2,\sigma'$ with
  $M \simt_1\Next e_2 \equiv {t_1}\Next e_2$ and $M\sims'\equiv{\sigma'}$.

  From the induction hypothesis, we obtain the following.
  $\forall M_1 .  M_1\phi \implies e,\sigma {\eval}{t_1},{\sigma'}\land  M_1 \simt_1\equiv {t_1} \land  M_1\sims'\equiv{\sigma'}$.

  Since $M$ satisfies $\phi$,
  we obtain from \refrule{E-Next} and the induction step above that $e_1\Next e_2,\sigma\eval t_1\Next e_2,\sigma'$ with $M \simt_1\Next e_2 \equiv {t_1}\Next e_2$ and $M \sims' \equiv{\sigma'}$.

  }

\case{\refrule{SE-Or}}
  {For all mappings $M$ such that $M(\phi_1\wedge \phi_2)$, we need to demonstrate that
   $e_1\Or e_2,\sigma\eval t_1\Or t_2,\sigma''$ with $M \simt_1\Or \simt_2 \equiv {t_1}\Or{t_2}$ and $M\sims''\equiv{\sigma''}$.

  From the induction hypothesis, we obtain the following.\\
  $\forall M_1 .  M_1\phi_1 \implies e_1, \sigma {\eval}{t_1},{\sigma'}\land  M_1 \simt_1\equiv {t_1} \land  M_1\sims'\equiv{\sigma'}$ and
  $\forall M_2 . M_2\phi_2 \implies e_2,\sigma' {\eval}{t_2},{\sigma''}\land M_2\simt_2\equiv {t_2} \land M_2\sims''\equiv{\sigma''}$

  Since $M$ satisfies both $\phi_1$ and $\phi_2$, we obtain from \refrule{E-Or} and the induction steps above that $e_1\Or e_2,\sigma\eval t_1\Or t_2,\sigma''$ with $M \simt_1\Or \simt_2 \equiv {t_1}\Or{t_2}$ and $M \sims'' \equiv{\sigma''}$.

  }

  \case{\refrule{SE-And}}
    {For all mappings $M$ such that $M(\phi_1\wedge \phi_2)$, we need to demonstrate that
     $e_1\And e_2,\sigma\eval t_1\And t_2,\sigma''$ with $M \simt_1\And \simt_2 \equiv {t_1}\And{t_2}$ and $M\sims''\equiv{\sigma''}$.

    From the induction hypothesis, we obtain the following.\\
    $\forall M_1 .  M_1\phi_1 \implies e_1, \sigma {\eval}{t_1},{\sigma'}\land  M_1 \simt_1\equiv {t_1} \land  M_1\sims'\equiv{\sigma'}$ and
    $\forall M_2 . M_2\phi_2 \implies e_2,\sigma' {\eval}{t_2},{\sigma''}\land M_2\simt_2\equiv {t_2} \land M_2\sims''\equiv{\sigma''}$

    Since $M$ satisfies both $\phi_1$ and $\phi_2$, we obtain from \refrule{E-And} and the induction steps above that $e_1\And e_2,\sigma\eval t_1\And t_2,\sigma''$ with $M \simt_1\And \simt_2 \equiv {t_1}\And{t_2}$ and $M \sims'' \equiv{\sigma''}$.

    }

\end{proof}


\subsection{Proof of soundness of symbolic striding semantics}
\begin{proof}
  We prove Lemma 6.4 by induction over the derivation $t,\sigma\simstride\overline{\simt,\sims,\phi}$.

  \case{\refrule{SS-ThenStay},\refrule{SS-ThenFail}}
    {For all mappings $M$ such that $M\phi$
    we need to demonstrate that
    $t_1\Then e_2,\sigma\stride  t_1'\Then e_2,\sigma'$ with
    $M \simt_1'\Then e_2 \equiv {t_1'}\Then e_2 $ and $ M\sims'\equiv {\sigma'}$.

    From the induction hypothesis, we obtain the following.
    $\forall M_1 . M_1 \phi \implies t_1,\sigma {\stride} {t_1'},{\sigma'}\land M_1 \simt_1'\equiv{t_1'}\land M_1\sims' \equiv {\sigma'}$.

    Since $M$ satisfies $\phi$,
    we obtain from \refrule{S-ThenStay} and \refrule{S-ThenFail} respectively, and the induction step above that $t_1\Then e_2,\sigma\stride  t_1'\Then e_2,\sigma'$ with
    $M \simt_1'\Then e_2 \equiv {t_1'}\Then e_2 $ and $ M\sims'\equiv {\sigma'}$.
    }

  \case{\refrule{SS-ThenCont}}
    {For all mappings $M$ such that $M\phi_1\land M\phi_2$
    we need to demonstrate that
    $t_1\Then e_2,\sigma\stride t_2,\sigma''$ with
    $M \simt_2 \equiv {t_2}$ and $M\sims''\equiv {\sigma''}$.

    From the induction hypothesis, we obtain the following.
    $\forall M_1 . M_1 \phi_1 \implies t_1,\sigma {\stride} {t_1'},{\sigma'}\implies M_1 \simt_1'\equiv{t_1'}\land M_1\sims' \equiv {\sigma'}$.\\
    From Lemma 6.5 we know that
    $\forall M_2 . M_2 \phi_2 \implies e_2{v_1},\sigma'{\eval}{t_2},{\sigma''}\and M_2 \simt_2\equiv {t_2}\land M_2\sims''\equiv {\sigma''}$.

    Since $M$ satisfies both $\phi_1$ and $\phi_2$,
    we obtain from \refrule{S-ThenCont}, the induction step and application of Lemma 6.5 above that
    $t_1\Then e_2,\sigma\stride t_2,\sigma''$ with
    $M \simt_2 \equiv t_2$ and $M\sims''\equiv {\sigma''}$.
    }

  \case{\refrule{SS-OrLeft}}
    {For all mappings $M$ such that $M\phi$
    we have to demonstrate that
    $t_1\Or t_2,\sigma\stride t_1',\sigma'$ with
    $M \simt_1'\equiv {t_1'}$ and $M\sims'\equiv {\sigma'}$.

    From the induction hypothesis, we obtain the following.
    $\forall M_1 . M_1 \phi \implies t_1,\sigma {\stride} {t_1'},{\sigma'}\and M_1 \simt_1'\equiv{t_1'}\land M_1\sims' \equiv {\sigma'}$.

    Since $M$ satisfies $\phi$, we obtain from \refrule{S-OrLeft} and the induction step above that $t_1\Or t_2,\sigma\stride t_1',\sigma'$ with
    $M \simt_1'\equiv{t_1'}$ and $M\sims'\equiv {\sigma'}$.

    }

  \case{\refrule{SS-OrRight}}
    {For all mappings $M$ such that $M(\phi_1\land \phi_2)$
    we need to demonstrate that
    $t_1\Or t_2,\sigma\stride t_2',\sigma''$ with
    $M \simt_2'\equiv {t_2'}$ and $M\sims''\equiv {\sigma''}$.

    From the induction hypothesis, we obtain the following.\\
    $\forall M_1 . M_1 \phi_1 \implies t_1,\sigma {\stride} {t_1'},{\sigma'}\land M_1 \simt_1'\equiv{t_1'}\land M_1\sims' \equiv {\sigma'}$ and
    $\forall M_2 . M_2 \phi_2 \implies t_2,\sigma' {\stride} {t_2'},{\sigma''}\land M_2 \simt_2'\equiv{t_2'}\land M_2\sims'' \equiv {\sigma''}$.

    Since $M$ satisfies both $\phi_1$ and $\phi_2$, and from the premise we have that $\Value\ (\simt',\sims')=\bot$,
    we obtain from \refrule{S-OrRight} and the induction steps above that $t_1\Or t_2,\sigma\stride t_2',\sigma''$ with
    $M \simt_2'\equiv{t_2'}$ and $M\sims''\equiv{\sigma''}$.
    }

  \case{\refrule{SS-OrNone}}
    {For all mappings $M$ such that $M(\phi_1\land \phi_2)$
    we need to demonstrate that $t_1\Or t_2,\sigma\stride t_1'\Or t_2',\sigma''$ with
    $M \simt_1'\Or \simt_2'\equiv {t_1'}\Or{t_2'}$ and $M\sims''\equiv {\sigma''}$.

    From the induction hypothesis, we obtain the following.\\
    $\forall M_1 . M_1 \phi_1 \implies t_1,\sigma {\stride} {t_1'},{\sigma'}\land M_1 \simt_1'\equiv{t_1'}\land M_1\sims' \equiv {\sigma'}$ and
    $\forall M_2 . M_2 \phi_2 \implies t_2,\sigma' {\stride} {t_2'},{\sigma''}\land M_2 \simt_2'\equiv{t_2'}\land M_2\sims'' \equiv {\sigma''}$.

    Since $M$ satisfies both $\phi_1$ and $\phi_2$,
    we obtain from \refrule{S-OrNone} and the induction steps above that $t_1\Or t_2,\sigma\stride t_1'\Or t_2',\sigma''$ with
    $M \simt_1'\Or \simt_2'\equiv {t_1'}\Or{t_2'}$ and $M\sims''\equiv {\sigma''}$.

    }

\case{\refrule{SS-Edit}}
  {For all mappings $M$, we need to demonstrate that
  $\Edit v,\sigma\stride \Edit v,\sigma$ with $M\Edit v \equiv \Edit v$ and $M\sigma \equiv \sigma$.
  This follows trivially from \refrule{S-Edit}.

  }

\case{\refrule{SS-Fill}}
  {For all mappings $M$, we need to demonstrate that $\Enter\beta,\sigma\stride\Enter\beta,\sigma$ with
  $M\Enter \beta \equiv \Enter \beta$ and $M\sigma \equiv {\sigma}$.
  This follows trivially from \refrule{S-Fill}.
  }

\case{\refrule{SS-Update}}
  {For all mappings $M$, we need to demonstrate that $\Update l,\sigma\stride\Update l,\sigma$ with
  $M\Update l \equiv \Update l$ and $M\sigma \equiv {\sigma}$.
  This follows trivially from \refrule{S-Update}.
  }

\case{\refrule{SS-Fail}}
  {For all mappings $M$, we need to demonstrate that $\Fail,\sigma\stride\Fail,\sigma$
  with
  $M\Fail \equiv \Fail$ and $M\sigma \equiv {\sigma}$.
  This follows trivially from \refrule{S-Fail}.
  }

\case{\refrule{SS-Xor}}
  {For all mappings $M$, we need to demonstrate that $e_1\Xor e_2,\sigma\stride e_1\Xor e_2,\sigma$ with
  $M e_1\Xor e_2 \equiv e_1\Xor e_2$ and $M\sims \equiv {\sigma}$.
  This follows trivially from \refrule{S-Xor}.
  }

\case{\refrule{SS-Next}}
  {For all mappings $M$ such that $M\phi$,
  we need to demonstrate that
  $t_1\Next e_2,\sigma\stride t_1'\Next e_2,\sigma'$ with
  $M \simt_1'\Next e_2\equiv {t_1'}\Next e_2$ and $M \sims'\equiv{\sigma'}$.

  From the induction hypothesis, we obtain the following.
  $\forall M_1 . M_1 \phi \implies t_1,\sigma {\stride} {t_1'},{\sigma'}\land M_1 \simt_1'\equiv{t_1'}\land M_1\sims' \equiv{\sigma'}$.

  Since $M$ satisfies $\phi$, we obtain from \refrule{S-Next} and the induction step above that $t_1\Next e_2,\sigma\stride t_1'\Next e_2,\sigma'$ with
  $M \simt_1'\Next e_2\equiv {t_1'}\Next e_2$ and $M \sims'\equiv{\sigma'}$.

  }

\case{\refrule{SS-And}}
  {For all mappings $M$ such that $M(\phi_1\land \phi_2)$
  we need to demonstrate that $t_1\And t_2,\sigma\stride t_1'\And t_2',\sigma''$ with
  $M \simt_1'\And \simt_2'\equiv {t_1'}\And{t_2'}$ and $M\sims''\equiv {\sigma''}$.

  From the induction hypothesis, we obtain the following.\\
  $\forall M_1 . M_1 \phi_1 \implies t_1,\sigma {\stride} {t_1'},{\sigma'}\and M_1 \simt_1'\equiv{t_1'}\land M_1\sims' \equiv {\sigma'}$ and
  $\forall M_2 . M_2 \phi_2 \implies t_2,\sigma' {\stride} {t_2'},{\sigma''}\and M_2 \simt_2'\equiv{t_2'}\land M_2\sims'' \equiv {\sigma''}$.

  Since $M$ satisfies both $\phi_1$ and $\phi_2$,
  we obtain from \refrule{S-And} and the induction steps above that $t_1\And t_2,\sigma\stride t_1'\And t_2',\sigma''$ with
  $M \simt_1'\And \simt_2'\equiv {t_1'}\And{t_2'}$ and $M\sims''\equiv {\sigma''}$.

  }

\end{proof}


\subsection{Proof of soundness of symbolic normalisation semantics}
\begin{proof}
  We prove Lemma 6.3 by induction over the derivation $e,\sigma\simnormalise \overline{\simt,\sims,\phi}$.

  The base case is when the SN-Done rule applies.
  Provided that $M(\phi_1\land \phi_2)$,
  we need to demonstrate that
  $e,\sigma\normalise t,\sigma'$ with
  $M \simt\equiv {t}$ and $M\sims'\equiv {\sigma'}$.

  By Lemma 6.5 and 6.4, we know that\\
  $\forall M_1. M_1\phi_1 \implies e,\sigma {\eval}{t},{\sigma'}\land M_1 \simt \equiv {t} \land M_1 \sims'\equiv {\sigma'}$ and
  $\forall M_2.M_2\phi_2\implies t,\sigma'{\stride}{t'},{\sigma''}\land M_2 \simt'\equiv{t'}\land M_2\sims''\equiv {\sigma''}$.

  Since $M$ satisfies both $\phi_1$ and $\phi_2$, we have $e,\sigma \normalise {t},{\sigma'}$ with $M\sims'\equiv {\sigma'}$.

  The induction step is when \refrule{SN-Repeat} applies.
  In this case, for all mappings $M$ such that
  $M(\phi_1\land \phi_2 \land \phi_3)$,
  we need to demonstrate that $e,\sigma\normalise t'',\sigma'''$
  with $M \simt''\equiv {t''}$ and $M\sims'''\equiv {\sigma'''}$.

  Again by Lemma 6.5 and 6.4, we know that\\
  $\forall M_1. M_1\phi_1 \implies e,\sigma {\eval}{t},{\sigma'}\land M_1 \simt \equiv {t} \land M_1 \sims'\equiv {\sigma'}$ and
  $\forall M_2.M_2\phi_2\implies t,\sigma'{\stride}{t'},{\sigma''}\land M_2 \simt'\equiv{t'}\land M_2\sims''\equiv {\sigma''}$.\\
  Furthermore, we know by applying the induction hypothesis that
  $\forall M_3.M_3\phi_3 \implies t',\sigma''{\normalise} {t''},{\sigma'''}\land M_3 \simt''\equiv {t''}\land M_3 \sims'''\equiv {\sigma'''}$.

  Since $M$ satisfies $\phi_1$, $\phi_2$ and $\phi_3$,
  we obtain from \refrule{N-Repeat}, the application of lemmas and the induction step above that $e,\sigma\normalise t'',\sigma'''$
  with $M \simt''\equiv {t''}$ and $M\sims'''\equiv {\sigma'''}$.
\end{proof}

\subsection{Proof of soundness of symbolic handling semantics}

\label{appendix:symbolicsoundhandle}
\begin{proof}
  We prove Lemma 6.2 by induction over the derivation $t,\sigma\simhandle\simt,\sims,\simi,\phi$.

  \case{\refrule{SH-Change}}
    {
    For all mappings $M$, we need to demonstrate that $\Edit v,\sigma\handle{M s}\Edit M s,\sigma$ with $M\Edit s\equiv \Edit M s$ and $ M\sigma\equiv {\sigma}$.

    This follows trivially from \refrule{H-Change}.

    }

  \case{\refrule{SH-Fill}}
  {For all mappings $M$, we need to demonstrate that
  $\Enter\beta,\sigma\handle{M s} \Edit M s,\sigma$ with
  $M \Edit s \equiv \Edit M s$ and $ M\sigma\equiv {\sigma}$.

  This follows trivially from \refrule{H-Fill}.

  }

  \case{\refrule{SH-Update}}
  {
  For all mappings $M$,
  we need to demonstrate that\\
  $\Update l,\sigma\handle{M s}\Update l,\sigma[l\mapsto M s]$ with
  $M \Update l \equiv \Update l$ and $ M\sigma[l\mapsto s]\equiv {\sigma}[l\mapsto M s]$.

  $\Update l,\sigma\handle{M s}\Update l,\sigma[l\mapsto M s]$ follows trivially from \refrule{H-Update}.
  $M \Update l \equiv \Update l$ follows trivially, since locations cannot contain symbols.
  $ M\sigma[l\mapsto s]\equiv {\sigma}[l\mapsto M s]$ follows trivially.

  }

  \case{\refrule{SH-Next}}

  {For all mappings $M$ such that $M\phi_1$, we need to demonstrate that
  $t_1\Next e_2,\sigma\handle{M \simi} t_1'\Next e_2,\sigma'$ with
  $M \simt_1' \Next e_2 \equiv {t_1'}\Next e_2$ and $M\sims'\equiv{\sigma'}$.

  By the induction hypothesis we obtain the following.
  $\forall M_1 . M_1 \phi_1 \implies t_1,\sigma \handle{M_1 \simi} {t_1'},{\sigma'}\land M_1 \simt_1'\equiv{t_1'}\land M_1\sims' \equiv {\sigma'}$

  Since $M$ satisfies $\phi_1$, we obtain from \refrule{H-PassNext} and the induction step above that $t_1\Next e_2,\sigma\handle{M \simi} t_1'\Next e_2,\sigma'$ with
  $M \simt_1' \Next e_2 \equiv {t_1'}\Next e_2$ and $M\sims'\equiv{\sigma'}$.



  For all mappings $M$ such that $M\phi_2$, we need to demonstrate that
  $t_1\Next e_2,\sigma\handle{\Continue}t_2,\sigma'$ with
  $M \simt_2 \equiv {t_2}$ and $M\sims'\equiv{\sigma'}$.

  From Lemma 6.3 we obtain that $\forall M_1. M_1 \phi \implies e_2 v_1,\sigma{\normalise}{t_2},{\sigma'}\land M \simt_2\equiv{t_2}\land M \sims'\equiv{\sigma'}$.

  This together with \refrule{H-Next} gives us exactly what we need to prove this case.
    }

  \case{\refrule{SH-PassNext}}
  {
  For all mappings $M$ such that $M\phi$, we need to demonstrate that
  $t_1\Next e_2,\sigma\handle{M\simi}t_1'\Next e_2,\sigma'$ with
  $M \simt_1' \Next e_2 \equiv {t_1'}\Next e_2$ and $M\sims'\equiv{\sigma'}$.

  By the induction hypothesis we obtain the following.
  $\forall M_1 . M_1 \phi_1 \implies t_1,\sigma \handle{M_1 \simi} {t_1'},{\sigma'}\land M_1 \simt_1'\equiv{t_1'}\land M_1\sims' \equiv {\sigma'}$

  Since $M$ satisfies $\phi$, we obtain from \refrule{H-PassNext} and the induction step above that $t_1\Next e_2,\sigma\handle{M\simi}t_1'\Next e_2,\sigma'$ with
  $M \simt_1' \Next e_2 \equiv {t_1'}\Next e_2$ and $M\sims'\equiv{\sigma'}$.
  }

    \case{\refrule{SH-PassNextFail}}
    {
    For all mappings $M$ such that $M\phi$, we need to demonstrate that
    $t_1\Next e_2,\sigma\handle{M\simi}t_1'\Next e_2,\sigma'$ with
    $M \simt_1' \Next e_2 \equiv {t_1'}\Next e_2$ and $M\sims'\equiv{\sigma'}$.

    By the induction hypothesis we obtain the following.
    $\forall M_1 . M_1 \phi_1 \implies t_1,\sigma \handle{M_1 \simi} {t_1'},{\sigma'}\land M_1 \simt_1'\equiv{t_1'}\land M_1\sims' \equiv {\sigma'}$.

    Since $M$ satisfies $\phi$ and from the premise of \refrule{SH-PassNextFail} we have $\Failing\ (\simt_2,\sims'')$, we obtain from \refrule{H-PassNextFail} and the induction step above that $t_1\Next e_2,\sigma\handle{M\simi}t_1'\Next e_2,\sigma'$ with
    $M \simt_1' \Next e_2 \equiv {t_1'}\Next e_2$ and $M\sims'\equiv{\sigma'}$.
    }

  \case{\refrule{SH-PassThen}}
  {For all mappings $M$ such that $M\phi$, we need to demonstrate that
  $t_1\Then e_2,\sigma\handle{M\simi}t_1'\Then e_2,\sigma'$ with
  $M \simt_1'\Then e_2\equiv {t_1'}\Then e_2$ and $M\sims'\equiv{\sigma'}$.

  By the induction hypothesis we obtain the following.
  $\forall M_1 . M_1 \phi_1 \implies t_1,\sigma \handle{M_1 \simi} {t_1'},{\sigma'}\land M_1 \simt_1'\equiv{t_1'}\land M_1\sims' \equiv {\sigma'}$

  Since $M$ satisfies $\phi$, we obtain from \refrule{H-PassThen} and the induction step above that $t_1\Then e_2,\sigma\handle{M\simi}t_1'\Then e_2,\sigma'$ with
  $M \simt_1'\Then e_2\equiv {t_1'}\Then e_2$ and $M\sims'\equiv{\sigma'}$.
  }

    \case{\refrule{SH-Pick}}
    {
    We have that $M\phi_1$ and/or $M\phi_2$.
    In the first case, the proof is identical to the SH-PickLeft rule.
    In the second case, the proof is identical to the SH-PickRight rule.
    }

    \case{\refrule{SH-PickLeft}}
    {For all mappings $M$ such that  $M\phi_1$, we need to demonstrate that
    $e_1\Xor e_2,\sigma\handle{\Left}t_1,\sigma'$ with
    $M \simt_1\equiv {t_1}$ and $M\sims'\equiv {\sigma'}$.

    From Lemma 6.3 we obtain that $\forall M_1. M_1 \phi \implies e_1,\sigma{\normalise}{t_1},{\sigma'}\land M \simt_1\equiv{t_1}\land M \sims'\equiv{\sigma'}$.

    Since $M$ satisfies $\phi_1$, we obtain from \refrule{H-PickLeft} and the application of Lemma 6.3 above that $e_1\Xor e_2,\sigma\handle{\Left}t_1,\sigma'$ with
    $M \simt_1\equiv {t_1}$ and $M\sims'\equiv {\sigma'}$.
    }

    \case{\refrule{SH-PickRight}}
    {For all mappings $M$ such that $M\phi_2$, we need to demonstrate that
    $e_1\Xor e_2,\sigma\handle{\Right}t_2,\sigma'$ with
    $M \simt_2\equiv {t_2}$ and $M\sims_2\equiv {\sigma'}$.

    From Lemma 6.3 we obtain that $\forall M_1. M_1 \phi \implies e_2,\sigma{\normalise}{t_2},{\sigma'}\land M \simt_2\equiv{t_2}\land M \sims'\equiv{\sigma'}$.

    Since $M$ satisfies $\phi_2$, we obtain from \refrule{H-PickRight} and the application of Lemma 6.3 above that $e_1\Xor e_2,\sigma\handle{\Right}t_2,\sigma'$ with
    $M \simt_2\equiv {t_2}$ and $M\sims_2\equiv {\sigma'}$.
    }

  \case{\refrule{SH-And}}
  {For all mappings $M$ such that $M\phi_1$, we need to demonstrate that
   $t_1\And t_2,\sigma\handle{M\First \simi}t_1'\And t_2,\sigma'$ with
   $M \simt_1'\And t_2\equiv {t_1'}\And t_2$ and $M\sims'\equiv {\sigma'}$.

   By the induction hypothesis we obtain the following.
   $\forall M_1 . M_1 \phi_1 \implies t_1,\sigma \handle{M_1 \simi} {t_1'},{\sigma'}\land M_1 \simt_1'\equiv{t_1'}\land M_1\sims' \equiv {\sigma'}$.

   Since $M$ satisfies $\phi_1$, we obtain from \refrule{H-FirstAnd} and the induction step above that
   $t_1\And t_2,\sigma\handle{M\First \simi}t_1'\And t_2,\sigma'$ with
   $M \simt_1'\And t_2\equiv {t_1'}\And t_2$ and $M\sims'\equiv {\sigma'}$.


   For all mappings $M$ such that $M\phi_2$, we need to demonstrate that
   $t_1\And t_2,\sigma\handle{M\Second \simi}t_1\And t_2',\sigma'$ with
   $M t_1\And \simt_2'\equiv t_1\And t_2'$ and $M\sims'\equiv {\sigma'}$.

  By the induction hypothesis we obtain the following.
  $\forall M_1 . M_1 \phi_1 \implies t_2,\sims \handle{M_1 \simi} {t_2'},{\sigma'}\land M_1 \simt_2'\equiv{t_2'}\land M_1\sims' \equiv {\sigma'}$

  Since $M$ satisfies $\phi_2$, we obtain from \refrule{H-SecondAnd} and the induction step above that
  $t_1\And t_2,\sigma\handle{M\Second \simi}t_1\And t_2',\sigma'$ with
  $M t_1\And \simt_2'\equiv t_1\And t_2'$ and $M\sims'\equiv {\sigma'}$.
  }

  \case{\refrule{SH-Or}}
  {This case is proven in the same way as \refrule{SH-And}.
  }


\end{proof}


\subsection{Proof of soundness of symbolic interacting semantics}
\begin{proof}
  We prove Theorem 6.1 by induction on $\simt,\sims\siminteract \overline{\simt',
  \sims',\simi,\phi}$.
  There is only one rule that applies, namely
  \refrule{SI-Handle}.

  Provided that $M(\phi_1\land\phi_2)$, we need to demonstrate that
  $t,\sigma\interact{M\simi}t'',\sigma''$ with
  $M \simt'' \equiv {t''}$ and $M\sims''\equiv {\sigma''}$.


  Lemma 6.3 and Lemma 6.2 respectively give us that\\
$\forall M_1 . M_1 \phi_1 \implies t,\sigma \handle{M_1 \simi} {t'},{\sigma'}\land M_1 \simt'\equiv{t'}\land M_1\sims' \equiv {\sigma'}$ and
$\forall M_2 . M_2 \phi_2 \implies t',\sigma' {\normalise} {t''},{\sigma''}\land M_2 \simt''\equiv{t''}\land M_2\sims'' \equiv {\sigma''}$.

Since $M$ satisfies both $\phi_1$ and $\phi_2$,
we obtain exactly what we need to prove,
namely
$t,\sigma\interact{\simi}t'',\sigma''$
$M \simt'' \equiv {t''}$ and $M\sims''\equiv {\sigma''}$.

\end{proof}

% !TEX root=../../main.tex


\section{Completeness proofs}
\label{sec:completeness-proofs}

\subsection{Proof of completeness of the symbolic handling semantics}
\begin{proof}

  We prove Lemma~\ref{lem:completeHandle} by induction over the derivation $t,\sigma\handle{i}t',\sigma'$.

  \case{\refrule{H-Change}}
  {
  By the SH-Change rule, we have $\Edit v,\sigma\simhandle\Edit s,\sims,s,\True$, and $s\sim v'$ holds by definition of input simulation.
  }

  \case{\refrule{H-Fill}}
  {
    By the SH-Fill rule, we have $\Enter \beta ,\sigma\simhandle\Edit s,\sims,s,\True$, and $s\sim v$ holds by definition of input simulation.
  }

  \case{\refrule{H-Update}}
  {
    By the SH-Update rule, we have $\Update l,\sigma\simhandle\Update l ,\sims[l\mapsto s],s,\True$, and $s\sim v$ holds by definition of input simulation.
   }


    \case{\refrule{H-Next}}
    {
      By the SH-Next rule, we have $t_1\Next e_2,\sigma\simhandle\overline{\simt_1'\Next e_2,\sims_1,\simi,\phi_1}\cup\overline{t_2,\sims_2,\Continue,\phi_2}$, and $\Continue \sim \Continue$ holds by definition of input simulation.
    }
    \case{\refrule{H-PassNext}}
    {
    By application of the induction hypothesis, we obtain the following.\\
    For all $t_1,\sigma,i$ such that $t_1,\sigma\handle{i}t_1',\sigma'$ there exists an $\simi\sim i$ such that $t_1,\sigma\simhandle\overline{\simt_1,\sims,\simi,\phi}$.
    From this we can conclude that there exists a symbolic execution $t_1\Next e_2,\sigma\simhandle \overline{\simt_1\Next e_2,\sims,\simi,\phi}$, and that $\simi\sim i$.
    }



  \case{\refrule{H-PassThen}}
  {
  By application of the induction hypothesis, we obtain the following.\\
  For all $t_1,\sigma,i$ such that $t_1,\sigma\handle{i}t_1',\sigma'$ there exists an $\simi\sim i$ such that $t_1,\sigma\simhandle \overline{\simt_1,\sims,\simi,\phi}$.
  From this we can conclude that there exists a symbolic execution $t_1\Then e_2,\sigma\simhandle \overline{\simt_1\Then e_2,\sims,\simi,\phi}$, and $\simi\sim i$.
  }


  \case{\refrule{H-PickLeft}}
    {
      Lemma~\ref{lem:completeNormalise} gives us the following.\\
      There exists a symbolic execution $e_1,\sigma\simnormalise \overline{\simt_1,\sims,\phi_1}$.\\
      There exists a symbolic execution $e_2,\sims\simnormalise \overline{\simt_2,\sims',\phi_2}$.

      We can now conclude that a symbolic execution exists.
      Either by the \refrule{SH-PickLeft} rule, in case $\Failing\ (\simt_2,\sims')$, or by the \refrule{SH-Pick} rule in case $\neg\Failing\ (\simt_2,\sims')$.
      We have that $\Left\sim \Left$ holds by definition.
    }
  \case{\refrule{H-PickRight}}
    {
    Lemma~\ref{lem:completeNormalise} gives us the following.\\
    There exists a symbolic execution $e_1,\sigma\simnormalise\overline{ t_1,\sims,\phi_1}$.\\
    There exists a symbolic execution $e_2,\sims\simnormalise\overline{ t_2,\sims',\phi_2}$.

    We can now conclude that a symbolic execution exists.
    Either by the \refrule{SH-PickRight} rule, in case $\Failing\ (\simt_1,\sims)$, or by the \refrule{SH-Pick} rule in case $\neg\Failing\ (t_1,\sims)$.

    We have that $\Right\sim \Right$ holds by definition.
    }

    \case{\refrule{H-FirstOr}}
    {
    By application of the induction hypothesis, we obtain the following.
    For all $t_1,\sigma,i$ such that $t_1,\sigma\handle{i}t_1',\sigma'$ there exists an $\simi\sim i$ such that $t_1,\sigma\simhandle \simt_1,\sims,\simi,\phi$.

    From \refrule{SH-Or}, and the conclusion of the induction hypothesis,
    we can conclude that there exists a symbolic input, namely $\First \simi$, such that $t_1\Or t_2,\sigma\simhandle\overline{\simt_1'\Or t_2,\sims,\First\simi,\phi}$.
    From $\simi\sim i$ and by definition of input simulation, we can conclude that $\First \simi\sim \First i$.
    }

    \case{\refrule{H-SecondOr}}
    {
    By application of the induction hypothesis, we obtain the following.
    For all $t_2,\sigma,i$ such that $t_2,\sigma\handle{i}t_2',\sigma'$ there exists an $\simi\sim i$ such that $t_2,\sigma\simhandle\simt_2,\sims,\simi,\phi$.

    From \refrule{SH-Or}, and the induction step above,
    we can conclude that there exists a symbolic input such that $t_1\Or t_2,\sigma\simhandle\overline{\simt_1\Or t_2',\sims',\Second\simi,\phi}$, namely $\Second \simi$.
    From $\simi\sim i$ and by definition of input simulation, we can conclude that $\Second \simi\sim \Second i$.
    }

    \case{\refrule{H-FirstAnd}}
    {
    By application of the induction hypothesis, we obtain the following.
    For all $t_1,\sigma,i$ such that $t_1,\sigma\handle{i}t_1',\sigma'$ there exists an $\simi\sim i$ such that $t_1,\sigma\simhandle \simt_1,\sims,\simi,\phi$.

    From \refrule{SH-And}, and the conclusion of the induction step above,
    we can conclude that there exists a symbolic input, namely $\First \simi$ such that $t_1\And t_2,\sigma\simhandle\overline{\simt_1'\And t_2,\sims,\First\simi,\phi}$.
    From $\simi\sim i$ and by definition of input simulation, we can conclude that $\First \simi\sim \First i$.
    }

    \case{\refrule{H-SecondAnd}}
    {
    By application of the induction hypothesis, we obtain the following.
    For all $t_2,\sigma,i$ such that $t_2,\sigma\handle{i}t_2',\sigma'$ there exists an $\simi\sim i$ such that $t_2,\sigma\simhandle \simt_2,\sims,\simi,\phi$.

    From \refrule{SH-And}, and the conclusion of the induction step above,
    we can conclude that there exists a symbolic input, namely $\Second \simi$ such that $t_1\And t_2,\sigma\simhandle \overline{t_1\And \simt_2,\sims,\Second\simi,\phi}$.
    From $\simi\sim i$ and by definition of input simulation, we can conclude that $\Second \simi\sim \Second i$.
    }


\end{proof}

\subsection{Proof of completeness of the symbolic interaction semantics}

\begin{proof} The proof of \cref{thm:completeinteract} consists of one case, since
  the interacting semantics consists of one rule, namely\\
  \userule{I-Handle}.\\
  By Lemma~\ref{lem:completeHandle} we obtain the following.\\
  $t,\sigma\handle{i} t',\sigma'\implies \exists \simi . t,\sigma\simhandle \simt,\sims,\simi,\phi \land \simi\sim i$\\
  Then by Lemma~\ref{lem:completeNormalise} we obtain the following.\\
  $t',\sigma'\normalise t'',\sigma''\implies t',\sigma'\simnormalise \simt',\sims',\phi'$\\
  From the above, together with the SI-Handle rule, we can conclude that there exists a symbolic execution $t,\sigma\siminteract \simt'',\sims'',\simi,\phi\land \simi\sim i$.

\end{proof}



\end{document}
