%% For double-blind review submission, w/o CCS and ACM Reference (max submission space)
%\documentclass[acmsmall,review,anonymous]{acmart}\settopmatter{printfolios=true,printccs=false,printacmref=false}
%% For double-blind review submission, w/ CCS and ACM Reference
% \documentclass[acmsmall,review,anonymous]{acmart}\settopmatter{printfolios=true}
%% For single-blind review submission, w/o CCS and ACM Reference (max submission space)
% \documentclass[acmsmall,review]{acmart}\settopmatter{printfolios=true,printccs=false,printacmref=false}
%% For single-blind review submission, w/ CCS and ACM Reference
% \documentclass[acmsmall,review]{acmart}\settopmatter{printfolios=true}
%% For author draft version, w/o CCS and ACM Reference (max submission space)
% \documentclass[sigconf]{acmart}\settopmatter{printfolios=true,printccs=false,printacmref=false}
%% For final camera-ready submission, w/ required CCS and ACM Reference
%\documentclass[acmsmall]{acmart}\settopmatter{}
\documentclass[sigconf]{acmart}


% !TEX root=../main.tex


%% Basics %%%%%%%%%%%%%%%%%%%%%%%%%%%%%%%%%%%%%%%%%%%%%%%%%%%%%%%%%%%%%%%%%%%%%%

%% Fixes %%

%\usepackage{underscore}


%% Fonts %%

\usepackage[utf8]{inputenc} % on ACM whitelist
% \usepackage[T1]{fontenc}
%%NOTE: T1 doesn't have the `Th` and `Qu` ligatures :-(
\usepackage[OT1]{fontenc} % on ACM whitelist

\usepackage{stmaryrd}  % on ACM whitelist
% \usepackage{mathtools}
% \usepackage{eurosym}

\usepackage{amsthm}  % on ACM whitelist %%NOTE: here because defines \openbox which will also be defined by newtxmath...
% \usepackage{xcolor}


% \usepackage{tgpagella}
% \usepackage{lucidabr}

% \usepackage{libertine}
% \usepackage[varqu]{zi4}
% \usepackage[libertine]{newtxmath}


%% Programming %%

\usepackage{xargs}
\usepackage{ifthen}  % on ACM whitelist


%% Layout %%

% \usepackage{microtype}
% \usepackage{xspace}


%% Additions %%%%%%%%%%%%%%%%%%%%%%%%%%%%%%%%%%%%%%%%%%%%%%%%%%%%%%%%%%%%%%%%%%%

%% Textual %%

% \usepackage{titlesec}
%\usepackage[inline]{enumitem}
%\usepackage{quoting}


%% Maths %%

\usepackage{amsmath}  % on ACM whitelist


%% Graphics %%

\usepackage{graphicx}  % on ACM whitelist
%\usepackage{xcolor}
%\usepackage{dblfloatfix}
%\usepackage[export]{adjustbox}
% \usepackage[xcolor]{mdframed}
%\usepackage{tikz}
%\usetikzlibrary{trees}

%% Tabulations %%

\usepackage{booktabs}  % on ACM whitelist
\usepackage{array}  % on ACM whitelist


%% Listings %%

\usepackage[final]{listings}   % on ACM whitelist


%% References & Bibliography %%

%\usepackage[capitalize]{cleveref}
\usepackage{natbib}   % on ACM whitelist
% \usepackage[natbibapa,nodoi]{apacite}

% !TEX root=../main.tex


%% Fixes %%

\frenchspacing

% \newlength{\hugeskipamount}
% \setlength{\hugeskipamount}  {1.2500\baselineskip plus 0.3750\baselineskip minus 0.3750\baselineskip}
% \setlength{\bigskipamount}   {0.7500\baselineskip plus 0.2500\baselineskip minus 0.2500\baselineskip}
% \setlength{\medskipamount}   {0.3750\baselineskip plus 0.1250\baselineskip minus 0.1250\baselineskip}
% \setlength{\smallskipamount} {0.1875\baselineskip plus 0.0625\baselineskip minus 0.0625\baselineskip}

% \widowpenalty=150
% \clubpenalty=150


%% Section spacing %%
%%NOTE: requires 'titlesec'

% \titlespacing*{\section}{0pt}{\hugeskipamount}{\bigskipamount}
% \titlespacing*{\subsection}{0pt}{\bigskipamount}{\medskipamount}
% \titlespacing*{\paragraph}{0pt}{\medskipamount}{1em}


%% Math spacing %%

\setlength{\abovedisplayskip}{\smallskipamount}
\setlength{\belowdisplayskip}{\smallskipamount}

% \setlength{\topsep}{\smallskipamount}



%% Float spacing %%

\setlength{\abovecaptionskip}{\medskipamount}
\setlength{\floatsep}        {\medskipamount}
\setlength{\textfloatsep}    {\bigskipamount}
\setlength{\intextsep}       {\bigskipamount}
\setlength{\dblfloatsep}     {\medskipamount}
\setlength{\dbltextfloatsep} {\bigskipamount}


%% Description spacing %%

%\setlist{noitemsep}
%\setlist[description]{leftmargin=\parindent}


%% List spacing %%
%% NOTE: requires `paralist`

% \setlength{\pltopsep}   {\medskipamount}
% \setlength{\plpartopsep}{\parskip}
% \setlength{\plitemsep}  {\parskip}
% \setlength{\plparsep}   {\parskip}


%% Listings spacing %%

% \lstset
%   {aboveskip=\smallskipamount
%   ,belowskip=\smallskipamount
%   }


%% Tabular strech %%
%% NOTE: requires `array`

% \renewmacro{arraystretch}
%   {1.1}


%% Quoting %%
%%NOTE: requires 'quoting'

%\quotingsetup
%  {font=itshape
%  ,leftmargin=\parindent
%  ,listvskip}


%% Boxes %%

% \mdfsetup
%   {hidealllines=true
%   ,backgroundcolor=lightgray
%   }

% !TEX root=../main.tex


\input macros/auxiliaries


%% Fixes %%%%%%%%%%%%%%%%%%%%%%%%%%%%%%%%%%%%%%%%%%%%%%%%%%%%%%%%%%%%%%%%%%%%%%%

\let\texttilde\textasciitilde


%% Text %%%%%%%%%%%%%%%%%%%%%%%%%%%%%%%%%%%%%%%%%%%%%%%%%%%%%%%%%%%%%%%%%%%%%%%%

\newmacro{separate}
  {\medskip\noindent}

\providemacro{marginnote}
  {\marginpar}
\providemacro{smallcaps}
  {\textsc}
\providemacro{marginwidth}
  {\marginparwidth}

\newmacro{alert}[1]
  {\textbf{#1}}
\newmacro{divert}[1]
  {\textcolor{gray}{#1}}
\newmacro{enquote}[1]
  {``#1''}
\newmacro{fixme}[1]
  {\colorbox{yellow}{#1}\marginnote{\colorbox{yellow}{$\star$}}}
\newmacro{todo}[1]
  {\textcolor{red}{$\star$}\marginnote{\textcolor{red}{#1}}}
  % {}
\newmacro{type}[1]
  {\texttt{#1}}

\newmacro{add}[1]
  {\textcolor{green}{#1}}
\newmacro{remove}[1]
  {\textcolor{red}{#1}}
\newmacro{change}[1]
  {\textcolor{orange}{#1}}
\newmacro{adjust}[2]
  {\remove{#1} \add{#2}}

\newenvironment{fadeout}
  {\color{gray}}
  {}
\newenvironment{emphasize}
  {\begin{quote}\itshape}
  {\end{quote}}
\newenvironment{margintext}[1]
  {\begin{marginfigure}
     \subsection*{#1}}
  {\end{marginfigure}}


%% Lists %%
%% NOTE: requires `paralist`

% %% Use compact lists by default
% \renewenvironment{itemize}
%   {\begin{compactitem}}
%   {\end{compactitem}}
% \renewenvironment{enumerate}
%   {\begin{compactenum}}
%   {\end{compactenum}}
% \renewenvironment{description}
%   {\begin{compactdesc}}
%   {\end{compactdesc}}
% %% Define starred versions as in-paragraph-lists
% \newenvironment{itemize*}
%   {\begin{inparaitem}}
%   {\end{inparaitem}}
% \newenvironment{enumerate*}[1][1=(i)]
%   {\begin{inparaenum}[#1]}
%   {\end{inparaenum}}
% \newenvironment{description*}
%   {\begin{inparadesc}}
%   {\end{inparadesc}}


%% Quotations %%
%% NOTE: requires `quoting`

\let\quote\quoting
\let\endquote\endquoting
\renewenvironment{quotation}
  {\ClassError{Please use the `quote` environment instead of `quotation`}}


%% Column types %%
%% NOTE: requires `array`

\newcolumntype{L}{>{$}l<{$}}
\newcolumntype{C}{>{$}c<{$}}
\newcolumntype{R}{>{$}r<{$}}
\newcolumntype{T}{>{\ttfamily}l}
\newcolumntype{S}{>{\sffamily}l}


%% References %%
%%NOTE: requires `cleveref`

\let\refer\cref
\let\Refer\Cref


%% Citations %%
%% NOTE: requires `natbib`

\let\cite\citep
\let\Cite\Citep
\let\textcite\citet
\let\Textcite\Citet


%% Blocks and Boxes %%

\newenvironment{block}
  {\begin{center}}
  {\end{center}}
% \newenvironment{box}
%   {\begin{mdframed}}
%   {\end{mdframed}}


%% Logos %%

%% \newlogo[.name.]{.text.}
\newmacro{newlogo}[2][1]
  {\ifthenelse{\isempty{#1}}
     {\newlogoaux{#2}{\smallcaps{\lowercase{#2}}}}
     {\newlogoaux{#1}{#2}}}
\newmacro{newlogoaux}[2]
  {\newmacro{#1}{#2}}


%% Languages %%%%%%%%%%%%%%%%%%%%%%%%%%%%%%%%%%%%%%%%%%%%%%%%%%%%%%%%%%%%%%%%%%%

%%NOTE: `\mathrel` gives a single space width between keywords but removes it after another relational operator.
%%      `\mathop`  gives just a small skip, but doesn't has above bug.
\newmacro{newoperator}[1]
  {\newmathcommand{#1}[op]}
\newmacro{newkeyword}[2][1]
  %%FIXME: this is to complicated: {\newoperator{\ifthenelse{\isempty{#1}}{#2}{#1}}{\text{\sffamily\bfseries #2}}}
  {\ifthenelse{\isempty{#1}}
    {\newoperator{#2}{\text{\normalfont\sffamily\bfseries #2}}}
    {\newoperator{#1}{\text{\normalfont\sffamily\bfseries #2}}}}
\newmacro{newvalue}[2][1]
  {\ifthenelse{\isempty{#1}}
    {\newoperator{#2}{\text{\normalfont\sffamily #2}}}
    {\newoperator{#1}{\text{\normalfont\sffamily #2}}}}
\newmacro{newtype}[2][1]
  {\ifthenelse{\isempty{#1}}
    {\newoperator{#2}{\text{\normalfont\sffamily\scshape #2}}}
    {\newoperator{#1}{\text{\normalfont\sffamily\scshape #2}}}}


%% Math %%%%%%%%%%%%%%%%%%%%%%%%%%%%%%%%%%%%%%%%%%%%%%%%%%%%%%%%%%%%%%%%%%%%%%%%

%% Boxes %%

\newmacro{obox}[2]
  {\makebox[0pt][l]{\ensuremath{#2}}\phantom{\ensuremath{#1}}}

\newmacro{highlight}[1]
  {\colorbox{lightgray}{\ensuremath{#1}}}

%% Spacing %%

\newmacro{Quad}
  {\hspace{1.5em}}
\newmacro{Break}
  {\\[\smallskipamount]}



%% Fractions %%

\newmacro{upon}
  {\genfrac{}{}{0pt}{0}}



%% Symbols %%

%% NOTE: change this to \emptyset when using a font that includes a nice standard emptyset
\let\nothing\varnothing



%% Braces %%

\let\<\langle
\let\>\rangle

\newmathcommand{llbrace}[open] {\{\!|}
\newmathcommand{rrbrace}[close]{|\!\}}

\newmacro{set}[1]
  {\ensuremath{\{#1\}}}
\newmacro{tuple}[1]
  {\ensuremath{\<#1\>}}


%% Operators %%

\let\lt<
\let\gt>
\let\To\Rightarrow

\newmathcommand{pp}[bin]
  {+\!\!+}
\newmathcommand{Mid}
  {\;\mid\;}


%% Shortcuts %%

\newmacro{powerset}[1]
  % {2^{#1}}
  {\mathcal{P}(#1)}

\newmathcommand{n}{\underline{n}}

\newmathcommand{NN}  [bb]{N}
\newmathcommand{ZZ}  [bb]{Z}
\newmathcommand{EE}  [bb]{E}
\newmathcommand{OO}  [bb]{O}
\newmathcommand{QQ}  [bb]{A}
\newmathcommand{RR}  [bb]{R}
\newmathcommand{CC}  [bb]{C}
\newmathcommand{HH}  [bb]{H}

\newmathcommand{LL}  [bb]{L}
\newmathcommand{UU}  [bb]{U}
\newmathcommand{BB}  [bb]{B}
\renewmathcommand{SS}[bb]{S}

\let\to\rightarrow
% \let\implies\Rightarrow
% \let\implies\Longrightarrow
\let\implies\supset
\let\infers\vdash


%% Hints and local definitions %%

\newmacro{hint}[1]
  {\quad\text{\{ #1 \}}}

\newmathcommand{when}[op]
  {\mathbf{when}}
\newmathcommand{where}[op]
  {\mathbf{where}}
\renewmathcommand{and}[op]
  {\mathbf{and}}
\newmathcommand{otherwise}[op]
  {\mathbf{otherwise}}
\newmathcommand{impossible}[op]
  {\mathrm{impossible}}


%% Environments %%

\let\group\begingroup

\newenvironment*{marginequation}[1][1=0pt]
  {\begin{marginfigure}[#1]\equation}
  {\endequation\end{marginfigure}}

\newenvironment*{marginequation*}[1][1=0pt]
  {\begin{marginfigure}[#1]\equation\nonumber}
  {\endequation\end{marginfigure}}


\newenvironment*{function}
  {\begin{tabular}{@{}L@{\ \ }C@{\ \ }L@{}}}
  {\end{tabular}}
\newmacro{signature}[1]
  {\multicolumn{3}{@{}L@{}}{#1}}
\newmacro{inset}[1]
  {\multicolumn{3}{L}{\quad #1}}


\newenvironment*{grammar}
  %%NOTE: the `@{}` suppreses `\tabcolsep` before the first column
  {\begin{block}\begin{tabular}{@{}rRCLl}}
  {\end{tabular}\end{block}}
\newenvironment*{grammar*}
  %%NOTE: the `@{}` suppreses `\tabcolsep` before the first column
  {\begin{block}\begin{tabular}{@{}RLl}}
  {\end{tabular}\end{block}}



%% Theorems %%

% \newtheoremstyle{plain}%
%   {\medskipamount}% space above
%   {\medskipamount}% space below
%   {\itshape}% body font
%   {0pt}% indent amount
%   {\bfseries}% head font
%   {.}% punctuation after head
%   {.5em}% spacing after head
%   {\thmname{#1}\thmnumber{ #2}\thmnote{ {\normalfont(#3)}}}% head spec
% \newtheoremstyle{definition}%
%   {\medskipamount}% space above
%   {\medskipamount}% space below
%   {\normalfont}% body font
%   {0pt}% indent amount
%   {\bfseries}% head font
%   {.}% punctuation after head
%   {.5em}% spacing after head
%   {\thmname{#1}\thmnumber{ #2}\thmnote{ {\normalfont(#3)}}}% head spec
%
%
\theoremstyle{acmplain}

\newtheorem{theorem}{Theorem}[section]
\newtheorem{conjecture}[theorem]{Conjecture}
\newtheorem{proposition}[theorem]{Proposition}
\newtheorem{lemma}[theorem]{Lemma}
\newtheorem{corollary}[theorem]{Corollary}


\theoremstyle{acmdefinition}

\newtheorem{example}[theorem]{Example}
\newtheorem{definition}[theorem]{Definition}



%% Inference rules %%

\newmacro{placerule}[4][1,4]
  {\ensuremath{
    \upon
      {\text{\smallcaps{#1}}\hfill}
      {\dfrac{#2}{#3}\ #4}
  }}

\newmacro{newrule}[4][4]
  {\newmacro{#1}{\placerule[#1]{#2}{#3}[#4]}}
\newmacro{userule}
  {\usemacro}
\newmacro{refrule}[1]
  {\ifthenelse{\isundefined{#1}}
    {\GenericError{}{Rule `#1` is not defined}{}{}}
    {\textsc{#1}}}
% \newmacro{refrule}
%   {\textsc}

% !TEX root=../main.tex


%% Styles %%%%%%%%%%%%%%%%%%%%%%%%%%%%%%%%%%%%%%%%%%%%%%%%%%%%%%%%%%%%%%%%%%%%%%

\lstdefinestyle{common}
  {escapechar=|
  ,numbersep=-9pt % to make numbers appear inside the column; otherwise they are in the margin
  ,aboveskip=0pt
  ,belowskip=0pt
  }

\lstdefinestyle{natural}
  {style=common
  ,columns=fullflexible
  ,gobble=2
  ,breaklines=true
  ,breakatwhitespace=true
  ,literate=
    %{.}{{$\cdot$}}1
    %{.}{{\ }}1
    {<<}{{$\<$}}1
    {>>}{{$\>$}}1
    {->}{{$\to$\ }}2
    % {--}{{--}}1
    %{_}{{\ }}1
    %{\ "}{{\ \textquotedblleft}}2
    %{"\ }{{\textquotedblright\ }}2
  ,basicstyle={\sffamily}
  ,keywordstyle=[1]{\bfseries}
  ,keywordstyle=[2]{\scshape}
  ,keywordstyle=[3]{}
  %,commentstyle={\itshape}
  %,identifierstyle={\itshape}
  ,emphstyle={\itshape}
  %,stringstyle={\rmfamily}
  ,showstringspaces=false
  ,texcl=true
  ,mathescape=true
  %,escapechar=\$
  %,escapeinside={\{\}}
  ,xleftmargin=1\parindent
  }

\lstdefinestyle{flexible}
  {columns=flexible
  ,gobble=2
  ,fontadjust=true
  ,basicstyle={\ttfamily\small}
  ,commentstyle={\itshape}
  ,keywordstyle={\bfseries}
  %,identifierstyle={\itshape}
  %,stringstyle={\ttfamily}
  ,emphstyle={\itshape}
  ,showstringspaces=false
  ,texcl=true
  ,mathescape=true
  %,escapechar=\$
  %,escapeinside={\{\}}
  ,xleftmargin=1\parindent
  }

\lstdefinestyle{literate}
  {style=natural
  ,literate=
    {\\}{{$\lambda$}}1
    {\\\$}{{\$}}1 %NOTE: otherwise eaten by `\`, NOTE: prevents \$ to be parsed as math escape
    {\\/}{{$\vee$}}1
    {/\\}{{$\wedge$}}1
    {A.}{{$\forall$}}1
    {E.}{{$\exist$}}1
    {->}{{$\rightarrow$ }}1
    {<-}{{$\leftarrow$}}1
    {==}{{$\equiv$\ }}1
    {/=}{{$\nequiv$\ }}1
    {<=}{{$\leq$}}1
    {>=}{{$\geq$}}1
    {>>=}{{>>=}}3 %NOTE: otherwise eaten by `>=`
    {\{|}{{$\{\!|\!$}}1
    {|\}}{{$\!|\!\}$}}1
    {\{|*|\}}{{$\{\!|\!\!\star\!\!|\!\}$}}3
  }


%% Definitions %%%%%%%%%%%%%%%%%%%%%%%%%%%%%%%%%%%%%%%%%%%%%%%%%%%%%%%%%%%%%%%%%

%% Tasks %%

\lstdefinelanguage{tasks}
  {sensitive=true
  ,morekeywords=[1]{let,in,if,then,else,case,of,ref,assert,type}
  ,morekeywords=[2]{Bool,Int,String,Unit,List, Ref,Task, Passenger,Seat,Booking, Snack}
  ,moreemph={a,b,c,d,e,f,g,h,i,j,k,l,m,n,o,p,q,r,s,t,u,v,w,x,y,z as,bs,cs,ds,es,fs,gs,hs,is,js,ks,ls,ms,ns,os,ps,qs,rs,ss,ts,us,vs,ws,xs,ys,zs}
  ,morestring=[b]"
  ,morecomment=[l]--
  ,morecomment=[n]{\{-}{-\}}
  }[keywords,strings,comments]
\lstdefinestyle{tasks}
  {style=natural
  ,literate=
    {\\}{{$\lambda$}}1
    {<<}{{$\<$}}1
    {>>}{{$\>$ }}1
    {->}{{$\to$ }}1
    {==}{{$\equiv$ }}1
    {/=}{{$\nequiv$ }}1
    {<=}{{$\leq$ }}1
    {>=}{{$\geq$ }}1
    {*}{{$\times$ }}1
    {`elem`}{{$\in$ }}1
    {\\/}{{$\vee$ }}1
    {/\\}{{$\wedge$ }}1
    {>>=}{{$\Then$ }}1
    {>>?}{{$\Next$ }}1
    {<&>}{{$\And$ }}1
    {<|>}{{$\Or$ }}1
    {<?>}{{$\Xor$ }}1
    {++}{{$\pp$ }}1
    {edit}{{$\Edit$}}1
    {enter}{{$\Enter$}}1
    {update}{{$\Update$}}1
    {fail}{{$\Fail$ }}1
  }

\lstnewenvironment{TASK}[1][]
  {\lstset{language=tasks,style=tasks,#1}}
  {}
\newmacro{TS}[1][]
  {\lstinline[language=tasks,style=tasks,#1]}
\newmacro{includeTASK}[2][]
  {\lstinputlisting[language=tasks,style=tasks,#1]{#2}}


%% Flows %%

\lstdefinelanguage{flows}
  {sensitive=true
  ,morekeywords=[1]{module,where,define,using,as,yielding,share,holding,with,do,for,fork,then,when,next,done,on,and,or,not,readonly,writeonly,readwrite}
  ,morekeywords=[2]{Bool,Int,String,Shared,List, Date,Document,Photo, Citizen,Company,Declaration}
  ,morekeywords=[3]{True,False,Just,Nothing,List}
  ,morestring=[b]"
  ,morecomment=[l]--
  ,morecomment=[n]{\{-}{-\}}
  }[keywords,strings,comments]

% \lstMakeShortInline[language=flows,style=natural] | % |
\lstnewenvironment{FLOW}[1][]
  {\lstset{language=flows,style=natural,#1}}
  {}
\newmacro{FL}[1][]
  {\lstinline[language=flows,style=natural,#1]}
\newmacro{includeFLOW}[2][]
  {\lstinputlisting[language=flows,style=natural,#1]{#2}}


%% Clean %%

\lstdefinelanguage{clean}
  {sensitive=true
  %,alsoletter={ABCDEFGHIJKLMNOPQRSTUVWXYZabcdefghijklmnopqrstuvwxyz_`}
  %,alsoletter={~!@\#$\%^\&*-+=?<>:|\\} %$
  ,morekeywords={from,definition,implementation,import,module,system,code,inline,if,case,of,let,let!,in,where,with,class,instance,generic,derive,dynamic,infix,infixl,infixr}
  ,morestring=[b]"
  ,morestring=[b]'
  ,morecomment=[l]//
  ,morecomment=[n]{/*}{*/}
  }[keywords,strings,comments]

\lstnewenvironment{CLEAN}[1][]
  {\lstset{language=clean,style=flexible,#1}}
  {}
\newmacro{CL}[1][]
  {\lstinline[language=clean,style=flexible,#1]}
\newmacro{includeCLEAN}[2][]
  {\lstinputlisting[language=clean,style=flexible,#1]{#2}}

\input{macros/abbreviations}
% !TEX root=main.tex



\let\phi\varphi

%% Host language %%%%%%%%%%%%%%%%%%%%%%%%%%%%%%%%%%%%%%%%%%%%%%%%%%%%%%%%%%%%%%%


\newkeyword[IF]  {if}
\newkeyword[THEN]{then}
\newkeyword[ELSE]{else}

\newkeyword[Let]{let}
\newkeyword[In]{in}

\newkeyword[Ref] {ref}


\newmacro{If}[3]
  {\IF #1 \THEN #2 \ELSE #3}



%% Values %%


\newmathcommand{unit}{\<\>}


\newvalue{True}
\newvalue{False}
\newvalue[Not]{not}


\newmacro{str}[1]
  {\text{``#1''}}

\newvalue[Map]{map}
\newvalue[Fst]{fst}
\newvalue[Snd]{snd}
\newvalue[Head]{head}
\newvalue[Tail]{tail}
\newvalue[Uniq]{uniq}
\newvalue[Len]{len}



%% Types %%


\newtype{Unit}
\newtype{Bool}
\newtype{Nat}
\newtype{Int}
\newtype{String}
\newtype[Reference]{Ref}
\newtype{Task}
\newtype{Maybe}
\newtype{List}

\newtype{Euro}



%% Object language %%%%%%%%%%%%%%%%%%%%%%%%%%%%%%%%%%%%%%%%%%%%%%%%%%%%%%%%%%%%%


\newmacro{TOPHAT}
  {$\widehat{\text{\smallcaps{top}}}$}
\newmacro{STOPHAT}
  {Symbolic $\widehat{\smallcaps{top}}$}


\let\And\relax
\newoperator{Then}  {\blacktriangleright}
\newoperator{Next}  {\vartriangleright}
\newoperator{And}   {\Join}
\newoperator{Or}    {\blacklozenge}
\newoperator{Xor}   {\lozenge}
\newoperator{Edit}  {\square}
\newoperator{View}  {\overline{\square}}
\newoperator{Enter} {\boxtimes}
\newoperator{Update}{\blacksquare}
\newoperator{Watch} {\overline{\blacksquare}}
\newoperator{Fail}  {\lightning}
\newoperator{At}    {@}

\newoperator{AndOr} {\DEPRECATED}



%% Events %%


\newvalue[Left]   {L}
\newvalue[Right]  {R}


\newvalue[Empty]   {E}
\newvalue[Continue]{C}
\newvalue[Pick]    {P}


\newvalue[First]  {F}
\newvalue[Second] {S}
\newvalue[Here]   {H}



%% Semantic functions %%%%%%%%%%%%%%%%%%%%%%%%%%%%%%%%%%%%%%%%%%%%%%%%%%%%%%%%%%


\newmathcommand{eval}[rel]
  {\;\downarrow\;}
\newmathcommand{stride}[rel]
  % {\;\rightarrow\!\shortmid\;}
  {\;\mapsto\;}
\newmathcommand{normalise}[rel]
  {\;\Downarrow\;}
\newmacro{handle}[1]
  {\mathrel{\;\xrightarrow{#1}\;}}
\newmacro{interact}[1]
  {\mathrel{\;\Rightarrow{#1}\;}}

  \newmathcommand{Leadsto}[rel]
   {\rotatebox[origin=c]{90}{\rotatebox[origin=c]{-90}{$\leadsto$}\!\rotatebox[origin=c]{-90}{$\leadsto$}}}

  \newmathcommand{simeval}[rel]
    {\;\rotatebox[origin=c]{-90}{$\leadsto$}\;}
  \newmathcommand{simstride}[rel]
    {\;\mapstochar\kern+0.08em\leadsto\;}
  \newmathcommand{simnormalise}[rel]
    {\;\rotatebox[origin=c]{-90}{$\Leadsto$}\;}
  \newmathcommand{simhandle}[rel]
    {\;\leadsto\;}
  \newmathcommand{siminteract}[rel]
    {\;\Leadsto\;}


    \newmathcommand{simi}[]
      {\tilde{\imath}}
    \newmathcommand{simt}[]
        {\tilde{t}}
    \newmathcommand{sims}[]
        {\tilde{\sigma}}
      \newmathcommand{sime}[]
        {\tilde{e}}
    \newmathcommand{simv}[]
        {\tilde{v}}

\newmathcommand{Simulate}[it]
  {simulate}
\newmathcommand{Again}[it]
  {again}

\newmathcommand{Value}[cal]
  {V}
\newmathcommand{Inputs}[cal]
  {I}
\newmathcommand{Interface}[cal]
  {U}
\newmathcommand{Failing}[cal]
  {F}
\newmathcommand{Watching}[cal]
  {W}
\newmathcommand{Dirty}
  {\Delta}
\newmathcommand{UserInterface}[cal]
  {U}
\newmathcommand{Sat}[cal]
  {S}



%% Proofs %%%%%%%%%%%%%%%%%%%%%%%%%%%%%%%%%%%%%%%%%%%%%%%%%%%%%%%%%%%%%%%%%%%%%%


% \newcommand{\case}[2]{
%     \noindent\textbf{Case} #1\\
%     \vspace{5mm}
%     \indent\begin{minipage}{\dimexpr\textwidth-3cm}
%     #2
%   \end{minipage}\\\\}

\newcommand{\case}[2]{
  \bigskip
  \noindent\textbf{Case} #1
  \nopagebreak[4]
  \smallskip
  \par
  \begingroup
    \leftskip\parindent
    \noindent
    #2
    \par
  \endgroup
}

% \newcommand{\case}[2]{
%   \noindent
%   \begin{tabular*}{\textwidth}{lp{0.8\textwidth}}
%     \textbf{Case} & #1 \\
%     \addlinespace
%     & #2
%   \end{tabular*}
%   \medskip
% }



%% Depricated %%%%%%%%%%%%%%%%%%%%%%%%%%%%%%%%%%%%%%%%%%%%%%%%%%%%%%%%%%%%%%%%%%

% !TEX root=main.tex


%% Typing %%%%%%%%%%%%%%%%%%%%%%%%%%%%%%%%%%%%%%%%%%%%%%%%%%%%%%%%%%%%%%%%%%%%%%

\newrule{T-Sym}
  {s:\beta \in \Gamma}
  {\Gamma,\Sigma \infers s:\tau}
  {}


%% Evaluation %%%%%%%%%%%%%%%%%%%%%%%%%%%%%%%%%%%%%%%%%%%%%%%%%%%%%%%%%%%%%%%%%%


\newmacro{RelationSE}
  {\sime,\sims \simeval \overline{\simv,\sims',\highlight{\phi}}}


\newrule{SE-Value}
  {}
  {\simv,\sims\simeval \simv,\highlight{\sims,\True}}
  {}


\newrule{SE-App}
  {\sime_1,\sims\simeval \overline{\lambda x:\tau.\sime_1',\sims',\highlight{\phi_1}}\Quad
   \sime_2,\sims'\simeval \overline{\simv_2,\sims'',\highlight{\phi_2}}
   \sime_1'[x\mapsto \simv_2],\sims''\simeval \overline{\simv_1,\sims''',{\phi_3}}}
  {\sime_1 \sime_2,\sims \simeval \overline{\simv_1,\sims''',\highlight{\phi_1\land\phi_2\land\phi_3}}}
  {}

\newrule{SE-If}
  {\sime_1,\sims\simeval \overline{\simv_1,\sims',\highlight{\phi_1}} \Quad
   \highlight{\sime_2,\sims'\simeval \overline{\simv_2,\sims'',\phi_2}} \Quad
   \highlight{\sime_3,\sims'\simeval \overline{\simv_3,\sims''',\phi_3}}}
  {\If{\sime_1}{\sime_2}{\sime_3},\sims\simeval \highlight{\overline{\simv_2,\sims'',\phi_1 \land \phi_2\land \simv_1} \cup \overline{\simv_3,\sims''',\phi_1 \land \phi_3 \land \lnot \simv_1}}}
  {}



\newrule{SE-Pair}
  {\upon{\sime_1,\sims\simeval \overline{\simv_1,\sims',\highlight{\phi_1}}}
   {\sime_2,\sims'\simeval \overline{\simv_2,\sims'',\highlight{\phi_2}}}}
  {\tuple{\sime_1,\sime_2},\sims\simeval\overline{\tuple{\simv_1,\simv_2},\sims'',\highlight{\phi_1\land\phi_2}}}
  {}

\newrule{SE-First}
  {\sime,\sims\simeval\overline{\tuple{\simv_1,\simv_2},\sims',\highlight{\phi}}}
  {\Fst \sime,\sims\simeval\overline{\simv_1,\sims',\highlight{\phi}} }
  {}

\newrule{SE-Second}
  {\sime,\sims\simeval\overline{\tuple{\simv_1,\simv_2},\sims',\highlight{\phi}}}
  {\Snd\sime,\sims\simeval\overline{\simv_2,\sims',\highlight{\phi}} }
  {}


%%%%%%%

\newrule{SE-Cons}
  {\sime_1,\sims \simeval \simv_1,\sims',\highlight{\phi_1}
   \sime_2,\sims' \simeval \simv_2,\sims'',\highlight{\phi_2}}
  {\sime_1 :: \sime_2,\sims \simeval \simv_1:: \simv_2,\sims'',\highlight{\phi_1\land\phi_2}}
  {}

\newrule{SE-Head}
  {\sime,\sims \simeval \simv_1::\simv_2,\sims',\highlight{\phi}}
  {\Head \sime,\sims \simeval \simv_1,\sims',\highlight{\phi}}
  {}

\newrule{SE-Tail}
  {\sime,\sims \simeval \simv_1::\simv_2,\sims',\highlight{\phi}}
  {\Tail \sime,\sims \simeval \simv_2,\sims',\highlight{\phi}}
  {}


%%%%%
\newrule{SE-Ref}
  {\sime,\sims\simeval \overline{\simv,\sims',\highlight{\phi}}\Quad
   l\not\in Dom(\sigma')}
  {\Ref \sime,\sims\simeval \overline{l,\sims'[l\mapsto \simv],\highlight{\phi}}}
  {}


\newrule{SE-Deref}
  {\sime,\sims\simeval \overline{l,\sims',\highlight{\phi}}}
  {!\sime,\sims\simeval \overline{\sims'(l),\sims',\highlight{\phi}}}
  {}

\newrule{SE-Assign}
  {\sime_1,\sims\simeval \overline{l,\sims',\highlight{\phi_1}} \Quad
   \sime_2,\sims'\simeval \overline{\simv_2,\sims'',\highlight{\phi_2}}}
  {\sime_1:=\sime_2,\sims\simeval \overline{\unit,\sims''[l\mapsto \simv_2],\highlight{\phi_1\wedge\phi_2}}}
  {}

\newrule{SE-Edit}
  {\sime,\sims \simeval \overline{\simv,\sims',\highlight{\phi}}}
  {\Edit \sime , \sims\simeval \overline{\Edit \simv,\sims',\highlight{\phi}}}
  {}

\newrule{SE-Update}
  {\sime,\sims\simeval \overline{l,\sims',\highlight{\phi}}}
  {\Update \sime ,\sims\simeval \overline{\Update l,\sims',\highlight{\phi}}}
  {}


\newrule{SE-Fail}
  {}
  {\Fail,\sims \simeval \Fail,\sims,\highlight{\True}}
  {}


\newrule{SE-Then}
  {\sime_1 ,\sims\simeval \overline{\simt_1,\sims',\highlight{\phi}}}
  {\sime_1 \Then \sime_2,\sims \simeval \overline{\simt_1 \Then \sime_2,\sims',\highlight{\phi}}}
  {}

\newrule{SE-Next}
  {\sime_1 ,\sims\simeval \overline{\simt_1,\sims',\highlight{\phi}}}
  {\sime_1 \Next \sime_2 ,\sims\simeval \overline{\simt_1 \Next \sime_2,\sims',\highlight{\phi}}}
  {}


\newrule{SE-And}
  {\sime_1 ,\sims\simeval \overline{\simt_1 ,\sims',\highlight{\phi_1}} \Quad
   \sime_2 ,\sims'\simeval \overline{\simt_2,\sims'',\highlight{\phi_2}}}
  {\sime_1 \And \sime_2 ,\sims\simeval \overline{\simt_1 \And \simt_2,\sims'',\highlight{\phi_1\land\phi_2}}}
  {}


\newrule{SE-Or}
  {\sime_1 ,\sims\simeval \overline{\simt_1 ,\sims',\highlight{\phi_1}} \Quad
   \sime_2 ,\sims'\simeval \overline{\simt_2,\sims'',\highlight{\phi_2}}}
  {\sime_1 \Or \sime_2 ,\sims\simeval \overline{\simt_1 \Or \simt_2,\sims'',\highlight{\phi_1\land\phi_2}}}
  {}

%% Normalisation %%%%%%%%%%%%%%%%%%%%%%%%%%%%%%%%%%%%%%%%%%%%%%%%%%%%%%%%%%%%%%%


\newmacro{RelationSS}
  {\simt,\sims\simstride \overline{\simt',\sims',\highlight{\phi}}}


\newrule{SS-Edit}
  { }
  {\Edit \simv,\sims \simstride \Edit \simv,\sims,\highlight{\True}}
  {}

\newrule{SS-Fill}
  { }
  {\Enter \beta,\sims \simstride \Enter \beta,\sims,\highlight{\True}}
  {}

\newrule{SS-Update}
  { }
  {\Update l,\sims \simstride \Update l,\sims,\highlight{\True}}
  {}


\newrule{SS-Fail}
  { }
  {\Fail,\sims \simstride \Fail,\sims,\highlight{\True}}
  {}


\newrule{SS-ThenStay}
  {\simt_1,\sims \simstride \overline{\simt_1',\sims',\highlight{\phi}}}
  {\simt_1 \Then \sime_2,\sims \simstride \overline{\simt_1' \Then \sime_2,\sims',\highlight{\phi}}}
  {{\Value\ (\simt_1',\sims') = \bot}}

\newrule{SS-ThenFail}
  {\simt_1,\sims \simstride \overline{\simt_1',\sims',\highlight{\phi}} \Quad
   \sime_2\ \simv_1,\sims' \simeval \overline{\simt_2,\sims'',\highlight{\_}}}
  {\simt_1 \Then \sime_2,\sims \simstride \overline{\simt_1' \Then \sime_2,\sims',\highlight{\phi}}}
  {\Value\ (\simt_1',\sims') = \simv_1 \land \Failing\ (\simt_2,\sims'')}

\newrule{SS-ThenCont}
  {\simt_1,\sims \simstride \overline{\simt_1',\sims',\highlight{\phi_1}} \Quad
   \sime_2\ \simv_1,\sims' \simeval \overline{\simt_2 ,\sims'',\highlight{\phi_2}}}
   % t_2,\sigma'' \stride t_2',\sigma'''}
  {\simt_1 \Then \sime_2,\sims \simstride \overline{t_2,\sigma'',\highlight{\phi_1\land\phi_2}}}
  {\Value\ (\simt_1',\sims') = \simv_1 \land \lnot\Failing\ (\simt_2,\sims'')}

\newrule{SS-Next}
  {\simt_1,\sims \simstride \overline{\simt_1',\sims',\highlight{\phi}}}
  {\simt_1 \Next \sime_2,\sims \simstride \overline{\simt_1' \Next \sime_2,\sims',\highlight{\phi}}}
  {}


\newrule{SS-And}
  {\simt_1,\sims  \simstride \overline{\simt_1',\sims',\highlight{\phi_1 }} \Quad
   \simt_2,\sims' \simstride \overline{\simt_2',\sims'',\highlight{\phi_2}}}
  {\simt_1 \And \simt_2,\sims \simstride \overline{\simt_1' \And \simt_2',\sims'',\highlight{\phi_1\land\phi_2}}}
  {}


\newrule{SS-OrLeft}
  {\simt_1,\sims  \simstride \overline{\simt_1',\sims',\highlight{\phi}}}
  {\simt_1 \Or \simt_2,\sims \simstride \overline{\simt_1',\sims',\highlight{\phi}}}
  {{\Value\ (\simt_1',\sims') = \simv_1}}

\newrule{SS-OrRight}
  {\simt_1,\sims  \simstride \overline{\simt_1',\sims',\highlight{\phi_1}}  \Quad
   \simt_2,\sims' \simstride \overline{\simt_2',\sims'',\highlight{\phi_2}}}
  {\simt_1 \Or \simt_2,\sims \simstride \overline{\simt_2',\sims'',\highlight{\phi_1\land\phi_2}}}
  {\Value\ (\simt_1',\sims') = \bot\land \Value\ (\simt_2',\sims'') = \simv_2}

\newrule{SS-OrNone}
  {\simt_1,\sims  \simstride \overline{\simt_1',\sims' ,\highlight{\phi_1}} \Quad
   \simt_2,\sims' \simstride \overline{\simt_2',\sims'',\highlight{\phi_2}}}
  {\simt_1 \Or \simt_2,\sims \simstride \overline{\simt_1' \Or \simt_2',\sims'',\highlight{\phi_1\land\phi_2}}}
  {\Value\ (\simt_1',\sims') = \bot \land \Value\ (\simt_2',\sims'') = \bot}

\newrule{SS-Xor}
  {\ }
  {\sime_1 \Xor \sime_2,\sims \simstride \sime_1 \Xor \sime_2,\sims,\highlight{\True}}
  {}


%% Normalisation %%


\newmacro{RelationSN}
  {\sime,\sims \simnormalise \overline{\simt,\sims',\highlight{\phi}}}


\newrule{SN-Done}
  {\sime,\sims \simeval \overline{\simt,\sims',\highlight{\phi_1}} \Quad
   \simt,\sims' \simstride \overline{\simt',\sims'',\highlight{\phi_2}}}
  {\sime,\sims \simnormalise \overline{\simt,\sims',\highlight{\phi_1\land\phi_2}}}
  {\sims'=\sims'' \land \simt=\simt'}

\newrule{SN-Repeat}
  {\upon{\sime,\sims \simeval \overline{\simt,\sims',\highlight{\phi_1}}}
   {{\simt,\sims' \simstride \overline{\simt',\sims'',\highlight{\phi_2}}}
   {\simt',\sims'' \simnormalise \overline{\simt'',\sims''',\highlight{\phi_3}}}}}
  {\sime,\sims \simnormalise \overline{\simt'',\sims''',\highlight{\phi_1 \land \phi_2 \land \phi_3}}}
  {\sims'\neq \sims''\vee \simt\neq \simt'}


%% Handling %%


\newmacro{RelationSH}
  {\simt,\sims \simhandle \overline{\simt',\sims',\highlight{\simi,\phi}}}


\newrule{SH-Change}
  { \text{fresh }s}
  {\Edit \simv,\sims \simhandle \Edit s,\sims,\highlight{s,\True}}
  {\simv,s:\beta}

\newrule{SH-Fill}
  { \text{fresh }s \Quad
    s:\beta}
  {\Enter \beta,\sims \simhandle \Edit s,\sims,\highlight{s,\True}}
  {}

\newrule{SH-Update}
  { \text{fresh }s \Quad
    \sims(l),s:\beta}
  {\Update l,\sims \simhandle \Update l,\sims[l \mapsto s],\highlight{s,\True}}
  {}

\newrule{SH-PassThen}
  {\simt_1,\sims \simhandle \overline{\simt_1',\sims',\simi,\phi}}
  {\simt_1 \Then \sime_2,\sims \simhandle \overline{\simt_1' \Then \sime_2,\sims',\highlight{\simi,\phi}}}
  {}

\newrule{SH-PassNext}
  {\simt_1,\sims \simhandle \overline{\simt_1',\sims',\simi,\phi} \Quad
   \Value\ {(\simt_1',\sims')} = \bot}
  {\simt_1 \Next \sime_2,\sims \simhandle \overline{\simt_1' \Next \sime_2,\sims',\highlight{\simi,\phi}}}
  {}


\newrule{SH-PassNextFail}
  {\upon{
   \simt_1,\sims \simhandle \overline{\simt_1',\sims_1,\simi,\phi} \Quad
   \Value\ {(\simt_1',\sims_1)} = \simv_1 }
   {\sime_2\ \simv_1,\sims_1 \simnormalise \overline{\simt_2,\sims_2,{\vphantom{i}\_}} \Quad
   \Failing\ (\simt_2,\sims_2)}}
  {\simt_1 \Next \sime_2,\sims \simhandle \overline{\simt_1' \Next \sime_2,\sims_1,\simi,\phi}}
  {}


\newrule{SH-Next}
  {\simt_1,\sims \simhandle \overline{\simt_1',\sims_1,\highlight{\simi,\phi_1}} \Quad
   \sime_2\ \simv_1,\sims_1 \simnormalise \overline{\simt_2,\sims_2,\highlight{\phi_2}}}
  {\simt_1 \Next \sime_2,\sims \simhandle\highlight{\overline{\simt_1' \Next \sime_2,\sims_1,\simi,\phi_1} \cup\overline{\simt_2,\sims_2,\Continue,\phi_2}}}
  {\Value\ {(\simt_1',\sims_1)} = \simv_1\land \neg\Failing\ (\simt_2,\sims_2)}

\newrule{SH-And}
  {\simt_1,\sims \simhandle \overline{\simt_1',\sims_1,\highlight{\simi_1,\phi_1}} \Quad
   \simt_2,\sims \simhandle \overline{\simt_2',\sims_2,\highlight{\simi_2,\phi_2}}}
  {\simt_1 \And \simt_2,\sims \simhandle \highlight{\overline{\simt_1' \And \simt_2,\sims_1,\First \simi_1,\phi_1}\cup \overline{\simt_1 \And \simt_2',\sims_2,\Second \simi_2,\phi_2}}}
  {}

\newrule{SH-Or}
  {\simt_1,\sims \simhandle \overline{\simt_1',\sims_1,\highlight{\simi_1,\phi_1}}\Quad
   \simt_2,\sims \simhandle \overline{\simt_2',\sims_2,\highlight{\simi_2,\phi_2}}}
  {\simt_1 \Or \simt_2,\sims \simhandle \highlight{\overline{\simt_1' \Or \simt_2,\sims_1,\First \simi_1,\phi_1}\cup\overline{\simt_1 \Or \simt_2',\sims_2,\Second \simi_2,\phi_2}}}
  {}


\newrule{SH-PickLeft}
  {\sime_1,\sims\simnormalise \overline{\simt_1,\sims_1,\highlight{\phi_1}} \Quad
   \sime_2,\sims \simnormalise \overline{\simt_2,\sims_2,\highlight{\phi_2}}}
  {\sime_1 \Xor \sime_2,\sims \simhandle \simt_1,\sims_1,\highlight{\Left,\phi_1}}
  {\neg\Failing\ (\simt_1,\sims_1)\land \Failing\ (\simt_2,\sims_2) }

\newrule{SH-PickRight}
  {\sime_1,\sims \simnormalise \overline{\simt_1,\sims_1,\highlight{\phi_1}} \Quad
   \sime_2,\sims \simnormalise \overline{\simt_2,\sims_2,\highlight{\phi_2}}}
  {\sime_1 \Xor \sime_2,\sims \simhandle \simt_2,\sims_2,\highlight{\Right,\phi_2}}
  {\Failing\ (\simt_1,\sims_1)\land \neg\Failing\ (\simt_2,\sims_2)}

\newrule{SH-Pick}
  {\sime_1,\sims \simnormalise \overline{\simt_1,\sims_1,\highlight{\phi_1}} \Quad
   \sime_2,\sims \simnormalise \overline{\simt_2,\sims_2,\highlight{\phi_2}}}
  {\sime_1 \Xor \sime_2,\sims \simhandle \highlight{\overline{\simt_1,\sims_1,\Left,\phi_1}\cup\overline{\simt_2,\sims_2,\Right,\phi_2}}}
  { \neg\Failing\ (\simt_1,\sims_1)\land \neg\Failing\ (\simt_2,\sims_2) }

%% Driving %%


\newmacro{RelationSI}
  {\simt,\sims \siminteract \overline{\simt',\sims',\highlight{\simi,\phi}}}


\newrule{SI-Handle}
  {\simt,\sims \simhandle \overline{\simt',\sims',\highlight{\simi,\phi_1}} \Quad
   \simt',\sims' \simnormalise \overline{\simt'',\sims'',\highlight{\phi_2}}}
  {\simt,\sims \siminteract \overline{\simt'',\sims'',\highlight{\simi,\phi_1 \land \phi_2}}}
  {}

%% Firsts %%

\newrule{R-Firsts}
  {t,\sigma\simulate\overline{\simv,\simi:\tilde{is},\Phi}}
  {\Firsts(t,\sigma,g) = \overline{\simi,\Phi\land g \simv}}
  {\Sat(\Phi\land g\simv)}

% !TEX root=main.tex


%% Typing %%%%%%%%%%%%%%%%%%%%%%%%%%%%%%%%%%%%%%%%%%%%%%%%%%%%%%%%%%%%%%%%%%%%%%


\newmacro{RelationT}
  {\Gamma,\Sigma \infers e : \tau}


\newrule{T-ConstBool}
  {c\in B}
  {\Gamma,\Sigma\infers c : \Bool}
  {}

\newrule{T-ConstInt}
  {c\in I}
  {\Gamma,\Sigma\infers c : \Int}
  {}

\newrule{T-ConstString}
  {c\in S}
  {\Gamma,\Sigma\infers c : \String}
  {}


\newrule{T-Unit}
  { }
  {\Gamma,\Sigma\infers \unit : \Unit}
  {}


\newrule{T-Var}
  {x:\tau\in\Gamma}
  {\Gamma,\Sigma\infers x:\tau}
  {}


\newrule{T-Abs}
  {\Gamma[x:\tau_1] ,\Sigma \infers e:\tau_2}
  {\Gamma,\Sigma \infers \lambda x : \tau_1 . e :\tau_1 \to \tau_2}
  {}

\newrule{T-App}
  { {\Gamma,\Sigma \infers e_1:\tau_1\to\tau_2 } \Quad
    {\Gamma,\Sigma \infers e_2:\tau_1}}
  {\Gamma,\Sigma \infers e_1 e_2 :\tau_2}
  {}


\newrule{T-If}
  { {\Gamma,\Sigma \infers e_1:\Bool}\Quad
   { {\Gamma,\Sigma \infers e_2:\tau}\Quad
   {\Gamma,\Sigma \infers e_3:\tau}}}
  {\Gamma,\Sigma \infers \If{e_1}{e_2}{e_3}:\tau}
  {}


\newrule{T-Pair}
  {\Gamma,\Sigma \infers e_1 : \tau_1  \Quad
   \Gamma,\Sigma \infers e_2 : \tau_2}
  {\Gamma,\Sigma \infers \tuple{e_1, e_2} :\tau_1 \times \tau_2}
  {}

\newrule{T-First}
  {\Gamma,\Sigma\infers e_1:\tau}
  {\Gamma,\Sigma\infers \Fst \tuple{e_1,e_2}:\tau}
  {}

\newrule{T-Second}
  {\Gamma,\Sigma\infers e_2:\tau}
  {\Gamma,\Sigma\infers \Snd \tuple{e_1,e_2}:\tau}
  {}

%%%%%
\newrule{T-ListEmpty}
  { }
  {\Gamma,\Sigma\infers [\ ]_\beta : \List\beta}
  {}

\newrule{T-ListCons}
  { {\Gamma,\Sigma\infers e_1:\beta}\Quad
   {\Gamma,\Sigma\infers e_2:\List\beta}}
  {\Gamma,\Sigma\infers e_1 :: e_2 : \List \beta}
  {}

\newrule{T-ListHead}
  {\Gamma,\Sigma\infers e:\List\beta}
  {\Gamma,\Sigma\infers \Head e:\beta}
  {}

\newrule{T-ListTail}
  {\Gamma,\Sigma\infers e:\List\beta}
  {\Gamma,\Sigma\infers \Tail e:\List\beta}
  {}

%%%%%


\newrule{T-Ref}
  {\Gamma,\Sigma \infers e:\beta}
  {\Gamma,\Sigma \infers \Ref e :\Reference \beta}
  {}

\newrule{T-Deref}
  {\Gamma,\Sigma \infers e:\Reference \beta}
  {\Gamma,\Sigma\infers\ !e:\beta}
  {}

\newrule{T-Assign}
  { {\Gamma,\Sigma\infers e_1:\Reference \beta}\Quad
   {\Gamma,\Sigma\infers e_2:\beta}}
  {\Gamma,\Sigma\infers e_1 := e_2:\Unit}
  {}

\newrule{T-Loc}
  {\Sigma(l) = \beta}
  {\Gamma,\Sigma\infers l:\Reference \beta}
  {}


\newrule{T-Edit}
  {\Gamma,\Sigma \infers e : \beta}
  {\Gamma,\Sigma \infers \Edit e : \Task \beta}
  {}

\newrule{T-Enter}
  {}
  {\Gamma,\Sigma \infers \Enter \beta : \Task \beta}
  {}

\newrule{T-Update}
  {\Gamma,\Sigma \infers e : \Reference \beta}
  {\Gamma,\Sigma \infers \Update e : \Task \beta}
  {}


\newrule{T-Fail}
  {}
  {\Gamma,\Sigma \infers \Fail : \Task \tau}
  {}


\newrule{T-Then}
  { {\Gamma,\Sigma \infers e_1 : \Task \tau_1}\Quad
   {\Gamma,\Sigma \infers e_2 : \tau_1 \to \Task \tau_2}}
  {\Gamma,\Sigma \infers e_1 \Then e_2 : \Task \tau_2}
  {}


\newrule{T-Next}
  { {\Gamma,\Sigma \infers e_1 : \Task \tau_1}\Quad
   {\Gamma,\Sigma \infers e_2 : \tau_1 \to \Task \tau_2}}
  {\Gamma,\Sigma \infers e_1 \Next e_2 : \Task \tau_2}
  {}


\newrule{T-And}
  { {\Gamma,\Sigma \infers e_1 : \Task \tau_1}\Quad
   {\Gamma,\Sigma \infers e_2 : \Task \tau_2}}
  {\Gamma,\Sigma \infers e_1 \And e_2 : \Task\,(\tau_1 \times \tau_2)}
  {}


\newrule{T-Or}
  { {\Gamma,\Sigma \infers e_1 : \Task \tau}\Quad
   {\Gamma,\Sigma \infers e_2 : \Task \tau}}
  {\Gamma,\Sigma \infers e_1 \Or e_2 : \Task \tau}
  {}


\newrule{T-Xor}
  { {\Gamma,\Sigma \infers e_1 : \Task \tau}\Quad
   {\Gamma,\Sigma \infers e_2 : \Task \tau}}
  {\Gamma,\Sigma \infers e_1 \Xor e_2 : \Task \tau}
  {}


%% Evaluation %%%%%%%%%%%%%%%%%%%%%%%%%%%%%%%%%%%%%%%%%%%%%%%%%%%%%%%%%%%%%%%%%%

\newmacro{RelationE}
  {e,\sigma \eval v,\sigma'}


\newrule{E-Value}
  {}
  {v,{\sigma}{\eval} v,{\sigma}}
  {}


\newrule{E-App}
  { {e_1               ,\sigma   \eval \lambda x:\tau.e_1',\sigma'}\Quad
   { {e_2               ,\sigma'  \eval v_2                ,\sigma''}\Quad
   {e_1'[x\mapsto v_2],\sigma'' \eval v_1                ,\sigma'''}}}
  {e_1 e_2           ,\sigma   \eval v_1                ,\sigma'''}
  {}


\newrule{E-IfTrue}
  { {e_1,{\sigma}{\eval} \True,{\sigma}'}\Quad
   {e_2,{\sigma}'{\eval} {v_2},{\sigma}''}}
  {\If{e_1}{e_2}{e_3},{\sigma}{\eval} {v_2},{\sigma}''}
  {}

\newrule{E-IfFalse}
  { {e_1,{\sigma}{\eval} \False ,{\sigma}'}\Quad
   {e_3,{\sigma}'{\eval} {v_3},{\sigma}''}}
  {\If{e_1}{e_2}{e_3},{\sigma}{\eval} {v_3},{\sigma}''}
  {}


\newrule{E-Pair}
  {e_1,{\sigma}{\eval} {v_1},{\sigma}' \Quad
   e_2,{\sigma}'{\eval} {v_2},{\sigma}''}
  {\tuple{e_1,e_2},{\sigma}{\eval}\tuple{{v_1},{v_2}},{\sigma}''}
  {}

\newrule{E-First}
  {e,\sigma\eval \tuple{v_1,v_2},\sigma'}
  {\Fst e,\sigma\eval v_1,\sigma'}
  {}

\newrule{E-Second}
  {e,\sigma\eval\tuple{v_1,v_2},\sigma'}
  {\Snd e,\sigma \eval v_2,\sigma' }
  {}

%%%%%%%%%

\newrule{E-Cons}
  {e_1,{\sigma}{\eval}{v_1},{\sigma}'\Quad
   e_2,{\sigma}'{\eval}{v_2},{\sigma}''}
  {e_1 :: e_2,{\sigma}{\eval}{v_1}::{v_2},{\sigma}''}
  {}

\newrule{E-Head}
  {e,{\sigma}{\eval} {v_1}::{v_2},{\sigma}'}
  {\Head e,{\sigma}{\eval}{v_1},{\sigma}'}
  {}

\newrule{E-Tail}
  {e,{\sigma}{\eval} {v_1}::{v_2},{\sigma}'}
  {\Tail e,{\sigma}{\eval}{v_2},{\sigma}'}
  {}

%%%%%%


\newrule{E-Ref}
  {e,{\sigma}{\eval} {v},{\sigma}' \Quad
   l\not\in Dom({\sigma}')}
  {\Ref e,{\sigma}{\eval} l,{\sigma}'[l\mapsto {v}]}
  {}

\newrule{E-Deref}
  {e,{\sigma}{\eval} l,{\sigma}'}
  {!e,{\sigma}{\eval} {\sigma}'(l),{\sigma}'}
  {}

\newrule{E-Assign}
  {e_1,{\sigma}{\eval} l,{\sigma}' \Quad
   e_2,{\sigma}'{\eval} {v_2},{\sigma}''}
  {e_1:=e_2,{\sigma}{\eval} \unit,{\sigma}''[l\mapsto {v_2}]}
  {}

\newrule{E-Edit}
  {e,{\sigma} {\eval} {v},{\sigma}'}
  {\Edit e , {\sigma}{\eval} \Edit {v},{\sigma}'}
  {}

\newrule{E-Update}
  {e,{\sigma}{\eval} l,{\sigma}'}
  {\Update e ,{\sigma}{\eval} \Update l,{\sigma}'}
  {}


\newrule{E-Fail}
  {}
  {\Fail,{\sigma} {\eval} \Fail,{\sigma}}
  {}


\newrule{E-Then}
  {e_1 ,{\sigma}{\eval} {t_1},{\sigma}'}
  {e_1 \Then e_2,{\sigma}{\eval}{t_1} \Then e_2,{\sigma}'}
  {}

\newrule{E-Next}
  {e_1 ,{\sigma}{\eval} {t_1},{\sigma}'}
  {e_1 \Next e_2 ,{\sigma}{\eval} {t_1} \Next e_2,{\sigma}'}
  {}


\newrule{E-And}
  {e_1 ,{\sigma}{\eval}{ t_1 },{\sigma}'\Quad
   e_2 ,{\sigma}'{\eval} {t_2},{\sigma}''}
  {e_1 \And e_2 ,{\sigma}{\eval}{ t_1} \And {t_2},{\sigma}''}
  {}


\newrule{E-Or}
  {e_1 ,{\sigma}{\eval}{ t_1} ,{\sigma}'\Quad
   e_2 ,{\sigma}'{\eval} {t_2},{\sigma}''}
  {e_1 \Or e_2 ,{\sigma}{\eval} {t_1} \Or {t_2},{\sigma}''}
  {}


%% Normalisation %%%%%%%%%%%%%%%%%%%%%%%%%%%%%%%%%%%%%%%%%%%%%%%%%%%%%%%%%%%%%%%

\newmacro{RelationS}
  {t,\sigma \stride t',\sigma'}


\newrule{S-Edit}
  { }
  {\Edit v,{\sigma} {\stride} \Edit v,{\sigma}}
  {}

\newrule{S-Fill}
  { }
  {\Enter \beta,{\sigma} {\stride} \Enter \beta,{\sigma}}
  {}

\newrule{S-Update}
  { }
  {\Update l,{\sigma} {\stride} \Update l,{\sigma}}
  {}


\newrule{S-Fail}
  { }
  {\Fail,{\sigma} {\stride} \Fail,{\sigma}}
  {}


\newrule{S-ThenStay}
  {t_1,{\sigma} {\stride} {t_1}',{\sigma}'}
  {t_1 \Then e_2,{\sigma} {\stride} {t_1}' \Then e_2,{\sigma}'}
  {\Value\ ({t_1}',{\sigma}') = \bot}

\newrule{S-ThenFail}
  { t_1,{\sigma} {\stride} {t_1}',{\sigma}' \Quad
    e_2\ {v_1},{\sigma}' {\eval} {t_2},{\sigma}''}
  {t_1 \Then e_2,{\sigma} {\stride} {t_1}' \Then e_2,{\sigma}'}
  {\Value\ ({t_1}',{\sigma}') = {v_1} \land \Failing\ ({t_2},{\sigma}'')}

\newrule{S-ThenCont}
  {t_1,{\sigma} {\stride} {t_1}',{\sigma}' \Quad
   e_2\ {v_1},{\sigma}' {\eval} {t_2 },{\sigma}''}
  {t_1 \Then e_2,{\sigma} {\stride} {t_2},{\sigma}''}
  {\Value\ ({t_1}',{\sigma}') = {v_1} \land \lnot\Failing\ ({t_2},{\sigma}'') }

\newrule{S-Next}
  {t_1,{\sigma} {\stride} {t_1}',{\sigma}'}
  {t_1 \Next e_2,{\sigma} {\stride} {t_1}' \Next e_2,{\sigma}'}
  {}


\newrule{S-And}
  {t_1,{\sigma}  {\stride} {t_1}',{\sigma}'  \Quad
   t_2,{\sigma}' {\stride} {t_2}',{\sigma}''}
  {t_1 \And t_2,{\sigma} {\stride} {t_1}' \And {t_2}',{\sigma}''}
  {}


\newrule{S-OrLeft}
  {t_1,{\sigma}  {\stride} {t_1}',{\sigma}'}
  {t_1 \Or t_2,{\sigma} {\stride} {t_1}',{\sigma}'}
  {\Value\ ({t_1}',{\sigma}') = {v_1}}

\newrule{S-OrRight}
  { {t_1,{\sigma}  {\stride} {t_1}',{\sigma}'}\Quad
   {t_2,{\sigma}' {\stride} {t_2}',{\sigma}''}}
  {t_1 \Or t_2,{\sigma} {\stride} {t_2}',{\sigma}''}
  {\Value\ ({t_1}',{\sigma}') = \bot \land \Value\ ({t_2}',{\sigma}'') = {v_2}}

\newrule{S-OrNone}
  { {t_1,{\sigma}  {\stride }{t_1}',{\sigma}'}\Quad
   { t_2,{\sigma' }{\stride} {t_2}',{\sigma}''}}
  {t_1 \Or t_2,{\sigma} {\stride} {t_1}' \Or {t_2}',{\sigma}''}
  {\Value\ ({t_1}',{\sigma}') = \bot \land \Value\ ({t_2}',{\sigma}'') = \bot }


\newrule{S-Xor}
  { }
  {e_1 \Xor e_2,{\sigma} {\stride} e_1 \Xor e_2,{\sigma}}
  {}

\newrule{S-Eval}
  {e,{\sigma} {\eval} {e}',{\sigma}'  \Quad
   e',{\sigma}' {\stride} {e}'',{\sigma}''}
  {e,{\sigma} {\stride} {e}'',{\sigma}''}
  {e \neq {e}'}


%% Normalisation %%

\newmacro{RelationN}
  {e,\sigma \normalise t,\sigma'}


\newrule{N-Done}
  { {e,{\sigma} {\eval} {t},{\sigma}'}\Quad
   { {t,\sigma' \stride t',\sigma''}\Quad
   {\sigma'=\sigma'' \land t=t'}}}
  {e,{\sigma} {\normalise} {t},{\sigma}'}
  {}
    %[\sigma'=\sigma'' \land t=t']

\newrule{N-Repeat}
  { e,{\sigma} {\eval} {t},{\sigma}'\Quad
    t,\sigma' \stride t',{\sigma}''\Quad
    t',\sigma'' \normalise t'',\sigma'''}
  {e,{\sigma} {\normalise} {t}'',{\sigma}'''}
  {{\sigma}'\neq {\sigma}''\lor {t}\neq {t}'}


%% Handling %%

\newmacro{RelationH}
  {t,\sigma \handle{i} t',\sigma'}


\newrule{H-Change}
  {v,v':\beta}
  {\Edit v,{\sigma} \xrightarrow[]{v'} \Edit v',{\sigma}}
  {}

\newrule{H-Fill}
  { v:\beta}
  {\Enter \beta,{\sigma} \xrightarrow[]{v} \Edit v,{\sigma}}
  {}

\newrule{H-Update}
  {\sigma(l),v:\beta }
  {\Update l,{\sigma} \xrightarrow[]{v} \Update l,{\sigma}[l \mapsto v]}
  {}

\newrule{H-PassThen}
  {t_1,\sigma \xrightarrow[]{i} {t_1'},\sigma'}
  {t_1 \Then e_2,\sigma \xrightarrow[]{i} {t_1'} \Then e_2,\sigma'}
  {}

\newrule{H-PassNext}
  {t_1,\sigma \xrightarrow[]{i} {t_1'},\sigma'}
  {t_1 \Next e_2,\sigma \xrightarrow[]{i} {t_1'} \Next e_2,\sigma'}
  {}

\newrule{H-Next}
  { {e_2\ {v_1},\sigma {\normalise} {t_2},{\sigma}'}}
  {t_1 \Next e_2,\sigma \xrightarrow[]{\Continue} {t_2},{\sigma}'}
  {\Value\ {(t_1,\sigma)} = {v_1} \land \neg\Failing\ ({t_2},{\sigma}')}

\newrule{H-FirstAnd}
  {t_1,\sigma \xrightarrow[]{i} {t_1}',{\sigma}'}
  {t_1 \And t_2,\sigma \xrightarrow[]{\First i} {t_1}' \And t_2,{\sigma}'}
  {}

\newrule{H-SecondAnd}
  {t_2,\sigma \xrightarrow[]{i} {t_2}',{\sigma}'}
  {t_1 \And t_2,\sigma \xrightarrow[]{\Second i} t_1 \And {t_2}',{\sigma}'}
  {}


\newrule{H-FirstOr}
  {t_1,\sigma \xrightarrow[]{i} {t_1}',{\sigma}'}
  {t_1 \Or t_2,\sigma \xrightarrow[]{\First i} {t_1}' \Or t_2,{\sigma}'}
  {}

\newrule{H-SecondOr}
  {t_2,\sigma \xrightarrow[]{i} {t_2}',{\sigma}' }
  {t_1 \Or t_2,\sigma \xrightarrow[]{\Second i} t_1 \Or {t_2}',{\sigma}'}
  {}


\newrule{H-PickLeft}
  {e_1,\sigma \normalise {t_1},{\sigma}'}
  {e_1 \Xor e_2,\sigma \xrightarrow[]{\Left} {t_1},{\sigma}'}
  {\neg\Failing\ ({t_1},{\sigma}')}

\newrule{H-PickRight}
  {e_2,\sigma {\normalise} {t_2},{\sigma}'}
  {e_1 \Xor e_2,\sigma \xrightarrow[]{\Right} {t_2},{\sigma}'}
  { \neg\Failing\ ({t_2},{\sigma}')}



%% Driving %%

\newmacro{RelationI}
  {t,\sigma \interact{i} t',\sigma'}



\newrule{I-Handle}
  {t,\sigma \xrightarrow[]{i} {t}',{\sigma}' \Quad
   {t}',{\sigma}' {\normalise} {t}'',{\sigma}''}
  {t,\sigma \interact{i} {t}'',{\sigma}''}
  {}

% !TEX root=main.tex


%% Language %%%%%%%%%%%%%%%%%%%%%%%%%%%%%%%%%%%%%%%%%%%%%%%%%%%%%%%%%%%%%%%%%%%%

\newmacro{G-Language}{
  \begin{grammar}
    Expressions
      & e    &::= & \lambda x:\tau.\ e   & – abstraction \\
      &      &\mid& e_1\ e_2             & – application \\
      &      &\mid& x                    & – variable \\
      &      &\mid& s                    & – symbol \\
      &      &\mid& c                    & – constant \\
    \addlinespace
      &      &\mid& u\ e_1               & – unary operation \\
      &      &\mid& e_1\ o\ e_2          & – binary operation \\
      &      &\mid& \If{e_1}{e_2}{e_3}   & – conditional \\
    \addlinespace
      &      &\mid& \unit                & – unit \\
      &      &\mid& \tuple{e_1, e_2}     & – pair \\
      &      &\mid& \Fst e               & – first projection \\
      &      &\mid& \Snd e               & – second projection \\
    \addlinespace
      &      &\mid& \Ref e               & – reference \\
      &      &\mid& !e                   & – dereference \\
      &      &\mid& e_1 := e_2           & – assignment \\
      % &      &\mid& e_1; e_2             & – sequence \\
      &      &\mid& l                    & – location \\
    \addlinespace
      &      &\mid& p                    & – pretask \\
    \addlinespace
    Constants
      & c    &::= & B                    & – boolean \\
      &      &\mid& I                    & – integer \\
      &      &\mid& S                    & – string \\
    \addlinespace
    Unary operations
      & u    &::= & \lnot                & – logical \\
      &      &\mid& -                    & – numerical \\
      &      &\mid& \Len                 & – sequential \\
    \addlinespace
    Binary operations
      & o    &::= & \land \mid \lor                                     & – logical \\
      &      &\mid& < \mid \le \mid \equiv \mid \nequiv \mid \ge \mid > & – equational \\
      &      &\mid& + \mid - \mid \times \mid /                         & – numerical \\
      &      &\mid& \pp                                                 & – sequential \\
  \end{grammar}
}

\newmacro{G-Language-Compact}{
  \begin{grammar}
    \noalign{Expressions}
    \addlinespace
    & \sime &::= & \lambda x:\tau.\ \sime  \Mid  \sime_1\ \sime_2                  & – abstraction, application \\
    &       & \mid  & x  \Mid  c \Mid \unit                        & – variable, constant, unit \\
    &       & \mid  & u\ \sime_1 \Mid \sime_1\ o\ \sime_2                      & – unary, binary operation \\
    &       & \mid  & \If{\sime_1}{\sime_2}{\sime_3}                           & – conditional \\
    &       & \mid  & \tuple{\sime_1, \sime_2}  \Mid  \Fst \sime  \Mid  \Snd \sime & – pair, projections \\
    &       & \mid  & [\ ]_\beta \Mid \sime_1 :: \sime_2                   & – nil, cons \\
    &       & \mid  & \Head\ \sime \Mid \Tail\ \sime                       & – first element, list tail \\
    &       & \mid  & \Ref \sime  \Mid  !\sime  \Mid  \sime_1 := \sime_2  \Mid  l  & – references, location \\
    &       & \mid  & \tilde{p} \Mid \highlight{s}                         & – pretask, symbol \\
    \\
    \noalign{Constants}
    \addlinespace
    & c& ::= &  B  \Mid  I  \Mid  S                                & – boolean, integer, string \\
    \\
    \noalign{Unary Operations}
    \addlinespace
    & \highlight{u} &::= &  \lnot \Mid - \Mid \Len \Mid \Uniq                  & – not, negate, length, unique \\
    \\
    \noalign{Binary Operations}
    \addlinespace
    & \highlight{o} &::= & < \Mid \le \Mid \equiv \Mid \nequiv \Mid \ge \Mid > & – equational \\
    &       & \mid  & + \Mid - \Mid \times \Mid /                  & – numerical \\
    &       & \mid  & \land  \Mid \lor                             & – conjunction, disjunction \\
    &       & \mid  & \pp  \Mid \in                                & – append, elementhood \\
  \end{grammar}
}


\newmacro{G-Pretasks}{
  \begin{grammar}
    Pretasks
      & \tilde{p}    &::= & \Edit \sime              & – valued editor \\
      % &      &\mid& \View e              & – valued read-only editor \\
      &      &\mid& \Enter \beta          & – unvalued editor \\
      &      &\mid& \Update \sime            & – shared editor \\
      % &      &\mid& \Watch e             & – shared read-only editor \\
    \addlinespace
      &      &\mid& e_1 \Then e_2        & – step \\
      &      &\mid& e_1 \Next e_2        & – user step \\
    \addlinespace
      &      &\mid& e_1 \And e_2         & – composition \\
    \addlinespace
      &      &\mid& e_1 \Or e_2          & – choice \\
      &      &\mid& e_1 \Xor e_2         & – user choice \\
    \addlinespace
      &      &\mid& \Fail                & – fail task \\
  \end{grammar}
}

\newmacro{G-Pretasks-Compact}{
  \begin{grammar}
    \noalign{Pretasks}
    \addlinespace
      & \tilde{p}    &::= & \Edit \sime \Mid \highlight{\Enter \beta} \Mid \Update \sime            & – editors: valued, unvalued, shared \\
      &      &\mid& \sime_1 \Then \sime_2 \Mid \sime_1 \Next \sime_2                   & – steps: internal, external \\
      &      &\mid& \Fail \Mid \sime_1 \And \sime_2                            & – fail, composition \\
      &      &\mid& \sime_1 \Or \sime_2 \Mid \sime_1 \Xor \sime_2                      & – choice: internal, external\\
  \end{grammar}
}


\newmacro{G-Types}{
  \begin{grammar}
    Types
      & \tau &::= & \tau_1 \to \tau_2    & – function type \\
      &      &\mid& \Reference \tau      & – reference type \\
      &      &\mid& \Task \tau           & – task type \\
      &      &\mid& \beta                & – basic type \\
      % &      &\mid& \alpha               & – universal type \\
    Basic types
      &\beta &::= & \tau_1 \times \tau_2 & – product type \\
      &      &\mid& \text{List} \beta    & - list type\\
      &      &\mid& \Unit                & – unit type \\
      &      &\mid& \Bool                & – boolean type \\
      &      &\mid& \Int                 & – integer type \\
      &      &\mid& \String              & – string type \\
  \end{grammar}
}

\newmacro{G-Types-Compact}{
  \begin{grammar}
    \noalign{Types}
    \addlinespace
      & \tau  &  ::= & \tau_1 \to \tau_2 \Mid \beta                      & – function, basic \\
      &       & \mid & \Reference \tau \Mid \Task \tau                   & – reference, task \\
      \\
    \noalign{Basic types}
    \addlinespace
      & \beta &  ::= & \highlight{\beta_1 \times \beta_2 \Mid \List \beta} \Mid \Unit  & – product, list, unit \\
      &       & \mid & \Bool \Mid \Int \Mid \String                      & – boolean, integer, string \\
  \end{grammar}
}


\newmacro{G-Values}{
  \begin{grammar}
    Values
      & v    &::= & \lambda x:\tau.\ e   & – abstraction \\
      &      &\mid& c                    & – constant \\
      &      &\mid& l                    & – location \\
    \addlinespace
      &      &\mid& s                    & – symbol \\
      &      &\mid& u\ v                 & – symbolic unary operation \\
      &      &\mid& v_1\ o\ v_2          & – symbolic binary operation \\
    \addlinespace
      &      &\mid& \tuple{v_1, v_2}     & – pair value \\
      &      &\mid& \unit                & – unit \\
    \addlinespace
      &      &\mid& t                    & – task \\
  \end{grammar}
}

\newmacro{G-Values-Compact}{
  \begin{grammar}
    \noalign{Values}
    \addlinespace
      & \simv &  ::= & \lambda x:\tau.\ \sime \Mid \tuple{\simv_1, \simv_2} \Mid \unit & – abstraction, pair, unit \\
      &   & \mid & [\ ]_\beta \Mid \simv_1 :: \simv_2 \Mid c                   & - nil, cons, constant \\
      &   & \mid & l \Mid \simt \Mid \highlight{s}                         & – location, task, symbol \\
      &   & \mid & \highlight{u\ \simv \Mid \simv_1\ o\ \simv_2}                   & – unary/binary operation \\
  \end{grammar}
}


\newmacro{G-Tasks}{
  \begin{grammar}
    Tasks
      & t    &::= & \Edit v              & – valued editor \\
      % &      &\mid& \View v              & – valued read-only editor \\
      &      &\mid& \Enter \tau          & – unvalued editor \\
      &      &\mid& \Update l            & – shared editor \\
      % &      &\mid& \Watch l             & – shared read-only editor \\
    \addlinespace
      &      &\mid& t_1 \Then e_2        & – step \\
      &      &\mid& t_1 \Next e_2        & – user step \\
    \addlinespace
      &      &\mid& t_1 \And t_2         & – composition \\
    \addlinespace
      &      &\mid& t_1 \Or t_2          & – choice \\
      &      &\mid& e_1 \Xor e_2         & – user choice \\
    \addlinespace
      &      &\mid& \Fail                & – fail task \\
  \end{grammar}
}

\newmacro{G-Tasks-Compact}{
  \begin{grammar}
    \noalign{Tasks}
    \addlinespace
      & \simt &  ::= & \Edit \simv \Mid \highlight{\Enter \beta} \Mid \Update l           & – editors \\
      &   & \mid & \simt_1 \Then \sime_2 \Mid \simt_1 \Next \sime_2                  & – steps \\
      &   & \mid & \Fail \Mid \simt_1 \And \simt_2                           & – fail, combination \\
      &   & \mid & \simt_1 \Or \simt_2 \Mid \sime_1 \Xor \sime_2                     & – choices \\
  \end{grammar}
}

\newmacro{G-Inputs}{
  \begin{grammar}
    Symbolic inputs
      & i    & ::=& $s$                  & – symbolic action \\
      &      &\mid& \First i             & – pass to first \\
      &      &\mid& \Second i            & – pass to second
  \end{grammar}
}

\newmacro{G-Inputs-Compact}{
  \begin{grammar}
    \noalign{Symbolic inputs}
    \addlinespace
      & \simi    & ::=& \tilde{a} \Mid \First \simi \Mid \Second \simi  & – symbolic action, to first, to second \\
      \\
    \noalign{Symbolic actions}
    \addlinespace
      & \tilde{a}  & ::=& \highlight{s}                     & – symbol \\
      &    &\mid& \Continue \Mid \Left \Mid \Right  & – continue, go left, go right \\
  \end{grammar}
}

\newmacro{G-CInputs}{
  \begin{grammar}
    Concrete inputs
      & i    & ::=& a              & – concrete action \\
      &      &\mid& \First i             & – pass to first \\
      &      &\mid& \Second i           & – pass to second \\
    Concrete actions
      & a & ::=& c                    & – constant \\
      &         &\mid& \Continue            & – continue with next task \\
      &         &\mid& \Left                & – go left \\
      &         &\mid& \Right               & – go right \\
  \end{grammar}
}

\newmacro{G-CInputs-Compact}{
  \begin{grammar}
    \noalign{Concrete inputs}
    \addlinespace
      & j    & ::=& a \Mid \First j \Mid \Second j             & – action, to first, to second \\
      \\
    \noalign{Concrete actions}
    \addlinespace
      & a  & ::=& c                    & – constant \\
      &      &\mid& \Continue  \Mid \Left \Mid \Right          & – continue, go left, go right \\
  \end{grammar}
}


\newmacro{G-Predicates}{
  \begin{grammar}
    Predicates
      & \phi &::= & c                    & – constant \\
      &      &\mid& s                    & – symbol \\
      &      &\mid& \Continue            & – continue\\
      &      &\mid& \Left                & – go left \\
      &      &\mid& \Right               & – go right \\
      &      &\mid& u\ \phi              & – symbolic unary operation \\
      &      &\mid& \phi_1\ o\ \phi_2    & – symbolic binary operation \\
  \end{grammar}
}

\newmacro{G-Predicates-Compact}{
  \begin{grammar}
    \noalign{Path conditions}
    \addlinespace
      & \highlight{\phi} &::= & c \Mid s              & – constant, symbol\\
      % &      &\mid& \Continue  \Mid \Left \Mid \Right          & – continue, go left, go right\\
      &      &\mid& \unit \Mid \tuple{\phi_1, \phi_2} & – unit, pairs \\
      &      &\mid& [\ ]_\beta \Mid \phi_1 :: \phi_2   & – nil, cons \\
      &      &\mid& u\ \phi \Mid \phi_1\ o\ \phi_2    & – symbolic unary/binary operation \\
  \end{grammar}
}

% !TEX root=main.tex


%% Language %%%%%%%%%%%%%%%%%%%%%%%%%%%%%%%%%%%%%%%%%%%%%%%%%%%%%%%%%%%%%%%%%%%%

\newmacro{O-Value}{
  \begin{function}
    \signature{\Value : \mathrm{Tasks} \times \mathrm{States} \rightharpoonup \mathrm{Values}} \\
    \Value(\Edit \simv, \sims)                &=& \simv \\
    \Value(\Enter \beta, \sims)            &=& \bot \\
    \Value(\Update l, \sims)              &=& \sims(l) \\
    \Value(\Fail, \sims)                  &=& \bot \\
    \Value(\simt_1 \Then \sime_2, \sims)          &=& \bot \\
    \Value(\simt_1 \Next \sime_2, \sims)          &=& \bot \\
    \Value(\simt_1 \And \simt_2, \sims)           &=& \left\{
      \begin{tabular}{ll}
        $\tuple{\simv_1, \simv_2}  $&$ \ \when\ \Value(\simt_1, \sims) = \
        \simv_1 \land \Value(\simt_2, \sims) = \simv_2 $\\
        $\bot                          $&$ \ \otherwise$
      \end{tabular}
    \right. \\
    \Value(\simt_1 \Or \simt_2, \sims)            &=& \left\{
      \begin{tabular}{ll}
        $\simv_1  $                         & $\ \when\ \Value(\simt_1, \sims) = \simv_1 $\\
        $\simv_2 $                          & $\ \when\ \Value(\simt_1, \sims) = \bot \land \Value(\simt_2, \sims) = \simv_2 $\\
        $\obox{\tuple{\simv_1, \simv_2}}{\bot}$ &$ \ \otherwise$
      \end{tabular}
    \right. \\
    \Value(\simt_1 \Xor \simt_2, \sims)           &=& \bot
  \end{function}
}

\newmacro{O-Failing}{
  \begin{function}
    \signature{\Failing : \mathrm{Tasks} \times \mathrm{States} \to \mathrm{Booleans}} \\
    \Failing(\Edit \simv,\sims)       &=& \False \\
    \Failing(\Enter \beta,\sims)   &=& \False \\
    \Failing(\Update l,\sims)     &=& \False \\
    \Failing(\Fail,\sims)         &=& \True \\
    \Failing(\simt_1 \Then \sime_2,\sims) &=& \Failing(\simt_1,\sims) \\
    \Failing(\simt_1 \Next \sime_2,\sims) &=& \Failing(\simt_1,\sims) \\
    \Failing(\simt_1 \And \simt_2,\sims)  &=& \Failing(\simt_1,\sims) \land \Failing(\simt_2,\sims) \\
    \Failing(\simt_1 \Or \simt_2,\sims)   &=& \Failing(\simt_1,\sims) \land \Failing(\simt_2,\sims) \\
    \Failing(\sime_1 \Xor \sime_2,\sims)  &=& \highlight{\bigwedge \Big( \set{\Failing(\simt_1,\sims_1') \mid \sime_1, \sims \normalise \overline{\simt_1,\sims_1'}}\ \cup} \\
                          & & \highlight{\phantom{\bigwedge \Big(}\set{\Failing(\simt_2,\sims_2') \mid \sime_2, \sims \normalise \overline{\simt_2,\sims_2'}} \Big)} \\
    % \Failing(e_1 \Xor e_2,\sims)  &=& \Failing(t_1,s_1') \land \Failing(t_2,s_2')\\
    % &&\quad \where\ e_1,\sims \normalise t_1,s_1' \mathbf{\ and\ } e_2,\sims \normalise t_2,s_2'
  \end{function}
}


\newmacro{O-Inputs}{
  \begin{function}
    \signature{\Inputs : \mathrm{Tasks} \times \mathrm{States} \to \powerset{\mathrm{Inputs}}} \\
    \Inputs(\Edit v,\sims)             &=& \set{\simv':\tau}                       \quad\where\ \Edit \simv : \Task\tau \\
    \Inputs(\Enter \beta,\sims)         &=& \set{\simv':\tau} \\
    \Inputs(\Update l,\sims)           &=& \obox{\set{\simv':\tau, \Empty}}{\set{\simv':\tau}} \quad\where\ \Update l : \Task\tau \\
    \Inputs(\Fail,\sims)               &=& \nothing \\
    \Inputs(\simt_1 \Then \sime_2,\sims)       &=& \Inputs(\simt_1,\sims) \\
    \Inputs(\simt_1 \Next \sime_2,\sims)       &=& \Inputs(\simt_1,\sims) \cup \set{\Continue \mid \Value(\simt_1, \sims) = \simv_1 \land \\
                                         && \sime_2\ \simv_1, \sims \normalise \simt_2, \sims',\phi \land \lnot\Failing(\simt_2, \sims')} \\
    \Inputs(\simt_1 \And \simt_2,\sims)        &=& \set{\First\ \simi \mid \simi \in \Inputs(\simt_1,\sims)} \cup \set{\Second\ \simi \mid i \in \Inputs(\simt_2,\sims)} \\
    \Inputs(\simt_1 \Or t_2,\sims)         &=& \set{\First\ \simi \mid \simi \in \Inputs(\simt_1,\sims)} \cup \set{\Second\ \simi \mid \simi \in \Inputs(\simt_2,\sims)} \\
    \Inputs(\sime_1 \Xor \sime_2,\sims)        &=& \set{\Left \mid \sime_1, \sims \normalise \simt_1, \sims',\phi \land \lnot\Failing(\simt_1, \sims')} \cup\\
                                         && \set{\Right \mid \sime_2, \sims \normalise \simt_2, \sims',\phi \land \lnot\Failing(\simt_2, \sims')}
  \end{function}
}




%% Journal information
%% Supplied to authors by publisher for camera-ready submission;
%% use defaults for review submission.
% \acmJournal{PACMPL}
% \acmVolume{1}
% \acmNumber{ICFP} % CONF = POPL or ICFP or OOPSLA
% \acmArticle{1}
% \acmYear{2018}
% \acmMonth{1}
% \acmDOI{} % \acmDOI{10.1145/nnnnnnn.nnnnnnn}
% \startPage{1}


\acmConference[IFL'19]{International Symposium on Implementation and Application of Functional Languages}{September 2019}{Singapore}
\acmYear{2020}
\copyrightyear{2020}
\acmISBN{978-1-4503-7562-7}

%% Copyright information
%% Supplied to authors (based on authors' rights management selection;
%% see authors.acm.org) by publisher for camera-ready submission;
%% use 'none' for review submission.
\setcopyright{none}
%\setcopyright{acmcopyright}
%\setcopyright{acmlicensed}
%\setcopyright{rightsretained}
%\copyrightyear{2018}           %% If different from \acmYear

%% Bibliography style
\bibliographystyle{ACM-Reference-Format}
%% Citation style
%% Note: author/year citations are required for papers published as an
%% issue of PACMPL.
\citestyle{acmauthoryear}   %% For author/year citations






% version.tex must define the command \version
\IfFileExists{version.tex}
  {\input{version.tex}}
  {\newcommand{\version}{unknown version}}

\hypersetup
{ pdfcreator=\version
}

\usepackage{fancyhdr}
\fancyfoot[C]{\thepage}
%\fancyfoot[R]{v.\version}






\begin{document}

%% Title information
\title{A symbolic execution semantics for TopHat}
% \title{TopHat: A calculus for modular interactive workflows}
                                        %% [Short Title] is optional;
                                        %% when present, will be used in
                                        %% header instead of Full Title.
%\titlenote{with title note}             %% \titlenote is optional;
                                        %% can be repeated if necessary;
                                        %% contents suppressed with 'anonymous'
%\subtitle{Revisited edition}            %% \subtitle is optional
%\subtitlenote{with subtitle note}       %% \subtitlenote is optional;
                                        %% can be repeated if necessary;
                                        %% contents suppressed with 'anonymous'


%% Author information
%% Contents and number of authors suppressed with 'anonymous'.
%% Each author should be introduced by \author, followed by
%% \authornote (optional), \orcid (optional), \affiliation, and
%% \email.
%% An author may have multiple affiliations and/or emails; repeat the
%% appropriate command.
%% Many elements are not rendered, but should be provided for metadata
%% extraction tools.

\author{Nico Naus}
%\authornote{with author1 note}          %% \authornote is optional; can be repeated if necessary
%\orcid{nnnn-nnnn-nnnn-nnnn}             %% \orcid is optional
\affiliation{
  %\position{PhD}
  \department{Computer Science}
                                        %% \department is recommended
  \institution{Open University}      %% \institution is required
  \streetaddress{Valkenburgerweg 177}
  \postcode{6419 AT}
  \city{Heerlen}
  %\state{State1}
  \country{The Netherlands}
}
\email{nico.naus@ou.nl}                    %% \email is recommended

\author{Tim Steenvoorden}
%\authornote{with author1 note}          %% \authornote is optional; can be repeated if necessary
%\orcid{nnnn-nnnn-nnnn-nnnn}             %% \orcid is optional
\affiliation{
  %\position{PhD}
  \department{Software Science}
  %\department{Institute for Computing and Information Sciences}
                                        %% \department is recommended
  \institution{Radboud University}      %% \institution is required
  \streetaddress{Toernooiveld 212}
  \postcode{6525 EC}
  \city{Nijmegen}
  %\state{State1}
  \country{The Netherlands}
}
\email{tim@cs.ru.nl}                     %% \email is recommended

\author{Markus Klinik}
%\authornote{with author1 note}          %% \authornote is optional; can be repeated if necessary
%\orcid{nnnn-nnnn-nnnn-nnnn}             %% \orcid is optional
\affiliation{
  %\position{PhD}
  \department{Software Science}
  %\department{Institute for Computing and Information Sciences}
                                        %% \department is recommended
  \institution{Radboud University}
                                        %% \institution is required
  \streetaddress{Toernooiveld 212}
  \postcode{6525 EC}
  \city{Nijmegen}
  %\state{State1}
  \country{The Netherlands}
}
\email{m.klinik@cs.ru.nl}               %% \email is recommended




%% Paper note
%% The \thanks command may be used to create a "paper note" ---
%% similar to a title note or an author note, but not explicitly
%% associated with a particular element.  It will appear immediately
%% above the permission/copyright statement.
%\thanks{with paper note}                %% \thanks is optional
                                        %% can be repeated if necesary
                                        %% contents suppressed with 'anonymous'


%% Abstract
%% Must come before \maketitle command
\begin{abstract}
  \input{sections/abstract}
\end{abstract}

% \begin{teaserfigure}
%    \includegraphics[width=\textwidth]{figures/declrequest-part.pdf}
%    \caption{This is a teaser}
%    \label{fig:teaser}
% \end{teaserfigure}

%% 2012 ACM Computing Classification System (CSS) concepts
%% Generate at 'http://dl.acm.org/ccs/ccs.cfm'.

%% End of generated code


%% Keywords
%% comma separated list, optional
%\keywords{workflow, dataflow, visual programming, program generation}


%% Note: \maketitle command must come after title commands, author
%% commands, abstract environment, Computing Classification System
%% environment and commands, and keywords command.
\maketitle

% !TEX root=../main.tex

\section{Introduction}

The Task-Oriented Programming paradigm (\TOP) is an abstraction over workflow specifications.
The idea of \TOP is to describe the work that needs to be done, in which order, by which person.
From this specification, an application can be generated that helps to coordinate people and machines to execute the work.
The \ITASKS framework~\cite{DBLP:conf/ppdp/PlasmeijerLMAK12} is an implementation of the paradigm in the functional programming language Clean.
In earlier work \cite{DBLP:conf/ppdp/SteenvoordenNK19}, we presented the programming language TopHat, written \TOPHAT, to distill the core features of \TOP into a language suitable for formal treatment.
%
The usefulness of \TOP has been demonstrated in several projects that applied it to implement various applications.
It has been used by the Netherlands Royal Navy~\cite{jansen2018dynamic}, the Dutch Tax Office~\cite{conf/sfp/StutterheimAP17} and the Dutch Coast Guard~\cite{lijnse2012incidone}. % of moeten we hier verwijzen naar Consequence Management – Declarative Modelling of Maritime C2-systems
Furthermore, it can potentially be applied in domains like healthcare and Internet of Things~\cite{DBLP:conf/cgo/KoopmanLP18}.

Applications in these kinds of domains are often mission critical, where programming mistakes can have severe consequences.
% Currently, iTasks programs are verified by running manually written test-cases. % do we have a source?
% It is evident that testing alone is not sufficient.
To verify that a \TOPHAT program behaves as intended, we would like to show that it satisfies a given property.
A common way to do this is to write test cases, or to generate random input, and verify that all outcomes satisfy the property.
Writing tests manually is time consuming and cumbersome.
Testing interactive applications needs people to operate the application, maybe making use of a way to record and replay interactions.
With this kind of testing there is no guarantee that all possible execution paths are covered.

To overcome these issues, we apply symbolic execution.
Instead of executing tasks with test input, or letting a user interactively test the application,
we run tasks on symbolic input.
Symbolic input consists of tokens that represent any value of a certain type.
When a program branches, the execution engine records the conditions over the symbolic input that lead to the different branches.
These conditions can then be compared to a given predicate to check if the predicate holds under all conditions.
We let an \SMT solver verify these statements.

In this way we can guarantee that given predicates over the outcome of a \TOP program always hold.
Since iTasks is not suitable for formal reasoning, we instead apply symbolic execution to \TOPHAT~\cite{DBLP:conf/ppdp/SteenvoordenNK19}, by systematically changing the semantic rules of the original language.



\subsection{Contributions}

This paper makes the following contributions.

\begin{itemize}
  \item We present a symbolic execution semantics for \TOPHAT, a programming language for workflows embedded in the simply typed $\lambda$-calculus.
  \item We prove soundness and completeness of the symbolic semantics with respect to the original \TOPHAT\ semantics.
  \item We present an implementation of the symbolic execution semantics in Haskell.
\end{itemize}



\subsection{Structure}

Section~\ref{sec:intuition} gives a brief overview of \TOPHAT\ and its concepts.
Section~\ref{sec:examples} introduces some examples to demonstrate the goal of our symbolic execution analysis.
In Section~\ref{sec:language}, the \TOPHAT\ language is defined.
Section~\ref{sec:semantics} goes on to define the formal semantics of the symbolic execution.
In Section~\ref{sec:properties}, soundness and completeness are shown for the symbolic execution semantics with respect to the original \TOPHAT\ semantics.
In Section~\ref{sec:relatedwork} related work is discussed, and Section~\ref{sec:conclusion} concludes.

% !TEX root=../main.tex

\section{TopHat}
\label{sec:intuition}

This section briefly introduces the task-oriented programming language \TOPHAT,
and discusses our vision about symbolic evaluation of this language.

The \TOPHAT\ language consists of two parts, the host language and the task language.
Programs in \TOPHAT\ are called \emph{tasks}.
The basic elements of tasks are editors.
Using combinators, tasks can be combined into larger tasks.

The task language is embedded in a simply typed lambda calculus with references, conditionals, booleans, integers, strings, pairs, lists and unary and binary operations on these types.
References allow tasks to communicate with each other, sharing information across task boundaries.
The simply typed $\lambda$-calculus does not have recursion.
By restricting references to only hold basic types,
strong normalisation of the calculus is guaranteed.
The full syntax of the host language is listed in section~\ref{sec:language}.
Next, we discuss the main constructs of the task language.


\subsection{Editors}

Editors are the most basic tasks.
They are used to communicate with the outside world.
Editors are an abstraction over widgets in a \GUI\ library or on webpage forms.
Users can change the value held by an editor, in the same way they can manipulate widgets in a \GUI.

When a \TOP\ implementation generates an application from a task specification, it derives user interfaces for the editors.
The appearance of an editor is influenced by its type.
For example, an editor for a string can be represented by a simple input field, a date by a calendar, and a location by a pin on a map.

There are three different editors in \TOPHAT.
\begin{description}
  \item[$\Edit v$] Valued editor.\\
    This editor holds a value $v$ of a certain type.
    The user can replace the value by a new value of the same type.
  \item[$\Enter \tau$] Unvalued editor.\\
    This editor holds no value, and can receive a value of type $\tau$.
    When that happens, it turns into a valued editor.
  \item[$\Update l$] Shared editor.\\
    This editor refers to a store location $l$.
    Its observable value is the value stored at that location.
    When it receives a new value, this value will be stored at location $l$.
\end{description}



\subsection{Combinators}

Editors can be combined into larger tasks using combinators.
Combinators describe the way people collaborate.
Tasks can be performed in sequence or in parallel, or there is a choice between two tasks.

The following combinators are available in \TOPHAT.
Here, $t$ stands for tasks and $e$ for arbitrary expressions.
The concrete syntax of the language is described in section~\ref{sec:expressions}
\begin{description}
  \item[$t \Then e$] Step.\\
    Users can work on task $t$.
    As soon as $t$ has a value, that value is passed on to the right hand side $e$.
    The expression $e$ is a function, taking the value as an argument, resulting in a new task.
  \item[$t \Next e$] User Step.\\
    Users can work on task $t$.
    When $t$ has a value, the step becomes enabled.
    Users can then send a continue event to the combinator.
    When that happens, the value of $t$ is passed to the right hand side, with which it continues.
  \item[$t_1 \And t_2$] Composition.\\
    Users can work on tasks $t_1$ and $t_2$ in parallel.
  \item[$t_1 \Or t_2$] Choice.\\
    The system chooses between $t_1$ or $t_2$,
    based on which task first has a value.
    If both tasks have a value, the system chooses the left one.
  \item[$e_1 \Xor e_2$] User choice.\\
    A user has to make a choice between either the left or the right hand side.
    The user continues to work on the chosen task.
\end{description}

In addition to editors and combinators, \TOPHAT\ also contains the fail task ($\Fail$).
Programmers can use this task to indicate that a task is not reachable or viable.
When the right hand side of a step combinator evaluates to $\Fail$, the step will not proceed to that task.



\subsection{Observations}

Several observations can be made on tasks.
Using the value function $\Value$, the current value of a task can be determined.
The value function is a partial function, since not all tasks have a value.
For example empty editors and steps do not have a value.

One can also observe whether or not a task is failing, by means of the failing function $\Failing$.
The task $\Fail$ is failing, as is a parallel combination of failing tasks ($\Fail \And \Fail$).

The step combinator makes use of both functions in order to determine if it can step.
First, it uses $\Value$ to see if the left hand side produces a value.
If that is the case, it uses the $\Failing$ function to see if it is safe to step to the right hand side.
The complete definition of the value and failing function are discussed in section~\ref{subsec:observations}.



\subsection{Input}

Input events drive evaluation of tasks.
Because tasks are typed, input is typed as well.
Editors only accept input of the correct type.
Examples are replacing a value in an editor,
or sending a continue event to a user step.
When the system receives a valid event, it gives this event to the current task, which reduces to a new task.
Everything in between interaction steps is evaluated atomically with respect to inputs.
% In this way the system communicates with the environment.

Input events are synchronous, which means the order of execution is completely determined by the order of the events.
In particular, the order of input events determine the progression of parallel branches.

% !TEX root=../main.tex

\section{Examples}
\label{sec:examples}

In this section we study three examples to illustrate how the language \TOPHAT works and what kind of properties we would like to prove.



\subsection{Positive value}

This example demonstrates how we can prove that the first observable value of a program can only be a positive number.
Consider the program in \cref{lst:abs}.

\begin{TASK}[float=h
            ,caption=A task that only steps on a positive input value.
            ,label=lst:abs
            ]
  enter Int >>= \ x. if x > 0 then edit x else fail
\end{TASK}

It asks the user to input a value of type $\Int$.
This value is then passed on to the right hand side.
If the value is greater than zero, an editor containing the entered value is returned.
At this point, the task has an observable value, and we consider it done.
Otherwise the step does not proceed and the task does not have an observable value.
The user can enter a different input value.

Imagine that we want to prove that no matter which value is given as input,
the first observable value is a value greater than zero.

Symbolic execution of this program proceeds as follows.
The symbolic execution engine generates a fresh symbolic input $s$ for the editor on the left.
The engine then arrives at the conditional.
To take the then-branch, the condition $s > 0$ needs to hold.
This branch will then result in $\Edit s$, in which case the program has an observable value.
The engine records this endpoint together with its path condition $s > 0$.
The else-branch applies if the condition does not hold, but this leads to a failing task.
Therefore, the step is not taken and the task expression is not altered.
No additional program state is generated.

Symbolic execution returns a list of all possible program end states, together with the path conditions that led to them.
If all end states satisfy the desired property, it is guaranteed that the property holds for all possible inputs.

In this example, the only end state is the expression $\Edit s$ with path condition $s > 0$.
From that we can conclude that no matter what input is given, the only result value possible is greater than zero.



\subsection{Tax subsidy request}

\citet{conf/sfp/StutterheimAP17} worked with the Dutch tax office to develop a demonstrator for a fictional but realistic law about solar panel subsidies.
In this section we study a simplified version of this, translated to \TOPHAT, to illustrate how symbolic execution can be used to prove that the program implements the law.

This example proves that a citizen will get subsidy only under the following conditions.
\begin{itemize}
\item The roofing company has confirmed that they installed solar panels for the citizen.
\item The tax officer has approved the request.
\item The tax officer can only approve the request if the roofing company has confirmed, and the request is filed within one year of the invoice date.
\item The amount of the granted subsidy is at most 600 EUR.
\end{itemize}

\lstset{emph={invoiceDate,date,confirmed,invoiceAmount,approved}}
\begin{TASK}[float=ht
            ,numbers=right
            ,caption=Subsidy request and approval workflow at the Dutch tax office.
            ,label=lst:tax
            ]
  let today = $\text{25 Sept 2020}$ in
  let provideDocuments = enter Amount <&> enter Date in |\label{lst:tax:citizen-info}|
  let companyConfirm = edit True <?> edit False in
  let officerApprove = \ invoiceDate. \ date. \ confirmed.
    edit False <?> if (date - invoiceDate < 365 /\ confirmed) |\label{lst:tax:officer-approve-def}|
      then edit True
      else fail in
  provideDocuments <&> companyConfirm >>= |\label{lst:tax:documents-and-company-confirm}|
    \ <<<<invoiceAmount, invoiceDate>>, confirmed>>.
  officerApprove invoiceDate today confirmed >>= \ approved.|\label{lst:tax:officer-approve}|
  let subsidyAmount = if approved
    then min 600 (invoiceAmount / 10) else 0 in
  edit <<subsidyAmount, approved, confirmed, invoiceDate, today>>|\label{lst:tax:result}|
\end{TASK}

\begin{figure}[ht]
  \includegraphics[width=\columnwidth]{figures/tax-enter}
  \caption{
    Graphical user interface for the task in \cref{lst:tax}.
    In parallel, the citizen is asked to enter the invoice amount and the invoice date of the installed solar panels,
    and the roofing company is asked to deny or confirm they actually installed the solar panels.
  }
  \label{fig:tax}
\end{figure}

\Cref{lst:tax} shows the program.
To enhance readability of the example,
we omit type annotations and make use of pattern matching on tuples.
The program works as follows.
First, the citizen has to enter their personal information (\cref{lst:tax:citizen-info}).
In the original demonstrator this included the citizen service number, name, and home address.
Here, we simplified the example so that the citizen only has to enter the invoice date.
A date is specified using an integer representing the number of days since 1 January 2000.

In the next step (\cref{lst:tax:documents-and-company-confirm}), in parallel the citizen has to provide the invoice documents of the installed solar panels, while the roofing company has to confirm that they have actually installed solar panels at the citizen's address.
Once the invoice and the confirmation are there, the tax officer has to approve the request (\cref{lst:tax:officer-approve}).
The officer can always decline the request, but they can only approve it if the roofing company has confirmed and the application date is within one year of the invoice date (\cref{lst:tax:officer-approve-def}).
The result of the program is the amount of the subsidy, together with all information needed to prove the required properties (\cref{lst:tax:result}).
The graphical user interface belonging to two steps in this process are shown in \cref{fig:tax}.

The result of the overall task is a tuple with the subsidy amount, the officer's approval, the roofing company's confirmation, the invoice amount, the invoice date, and today's date.
Returning all this information allows the following predicate to be stated, which verifies the correctness of the implementation.
The predicate has 5 free variables, which correspond to the returned values.
\setcounter{equation}{0}
\begin{align}
\psi(s,a,c,i,t)
   & =      s \geq 0 \implies c \label{for:tax:psi-confirmed}
\\ & \wedge s > 0 \implies a \label{for:tax:psi-approved}
\\ & \wedge a \implies (c \wedge t - i < 365) \label{for:tax:psi-approve-conditions}
\\ & \wedge s \leq 600 \label{for:tax:psi-max-subsidy}
\\ & \wedge \lnot a \implies s \equiv 0 \label{for:tax:psi-unapproved}
\end{align}
The predicate $\psi$ states that (\ref{for:tax:psi-confirmed}) if subsidy $s$ has been payed, the roofing company must have confirmed $c$, (\ref{for:tax:psi-approved}) if subsidy has been payed, the officer must have approved $a$, (\ref{for:tax:psi-approve-conditions}) the officer can approve only if the roofing company has confirmed and today's date $t$ is within 356 days of the invoice date $i$, and (\ref{for:tax:psi-max-subsidy}) the subsidy is maximal 600 EUR.
Finally, (\ref{for:tax:psi-unapproved}) if the officer has not approved, the subsidy must be 0.



\subsection{Flight booking}

In this section we develop a small flight booking system.
The purpose of this example is to demonstrate how symbolic execution handles references and lists.
We prove that when the program terminates, every passenger has exactly one seat, and that no two passengers have the same seat.
This program is a simplified version of what we presented in earlier work \cite{DBLP:conf/ppdp/SteenvoordenNK19}.

\begin{TASK}[float=ht
            ,numbers=right
            ,caption=Flight booking.
            ,label=lst:flight-booking
            ]
  let maxSeats = 50 in
  let bookedSeats = ref [] in|\label{lst:flight:make-ref}|
  let bookSeat = enter Int >>= \ x .|\label{lst:flight:enter-seat-number}|
    if not (x `elem` !bookedSeats) /\ x <= maxSeats|\label{lst:flight:guard-invalid-seats}|
      then bookedSeats := x :: !bookedSeats >>= \ _ . edit x |\label{lst:flight:update-seats}|
      else fail in
  bookSeat <&> bookSeat <&> bookSeat >>= \ _ .|\label{lst:flight:main-expression}|
  edit (!bookedSeats)
\end{TASK}

The program, shown in \cref{lst:flight-booking}, consists of three parallel seat booking tasks (\cref{lst:flight:main-expression}).
There is a shared list that stores all booked seats so far (\cref{lst:flight:make-ref}).
To book a seat, a passenger has to enter a seat number (\cref{lst:flight:enter-seat-number}).
A guard expression makes sure that only free seats can be booked (\cref{lst:flight:guard-invalid-seats}).
The exclamation mark denotes dereferencing.
When the guard is satisfied, the list of booked seats is updated, and the user can see his booked seat (\cref{lst:flight:update-seats}).
The main expression runs the seat booking task three times in parallel (\cref{lst:flight:main-expression}), simulating three concurrent customers.
The program returns the list of booked seats.
The graphical user interface, generated from the specification in \cref{lst:flight-booking}, is shown in \cref{fig:flight-booking}.

\begin{figure}[t]
  \includegraphics[width=\columnwidth]{figures/flight-booking.png}
  \caption{
    Graphical user interface generated from the specification in \cref{lst:flight-booking}.
    Three users are booking seats in parallel.
    The first user booked seat 9, the second did not enter a seat number yet, and the third is about to book seat 4.
  }
  \label{fig:flight-booking}
\end{figure}

With the returned list, we can state the predicate to verify the correctness of the booking process.
\setcounter{equation}{0}
\begin{align}
\psi(l)
   & =      \Len l \equiv 3 \label{flight-psi-exactly-three-seats}
\\ & \wedge \Uniq l \label{flight-psi-unique-seats}
\end{align}
The predicate specifies that all three passengers booked exactly one seat (\ref{flight-psi-exactly-three-seats}), and that all seats are unique (\ref{flight-psi-unique-seats}), which means that no two passengers booked the same seat.
The unary operators for list length ($\Len$) and uniqueness ($\Uniq$) are available in the predicate language.
List length is a capability of \SMTLIB, while $\Uniq$ is our own addition.

% !TEX root=../main.tex


\section{Language}
\label{sec:language}

The language presented in this section is nearly identical to the original \TOPHAT\ language presented by \citet{DBLP:conf/ppdp/SteenvoordenNK19}.
The main difference with the original grammar is the addition of symbolic values.

Symbolic execution for functional programming languages struggles with higher order features.
This topic is under active study, and is not the focus of our work \cite{HallahanXP2017, DBLP:conf/pldi/HallahanXBJP19}.
Therefore, we restrict symbols to only represent values of basic types.
This restriction is of little importance in the domains we are interested in.
Allowing users to enter higher order values is not useful in most workflow applications.
By restricting the input grammar to first-order values only, we ensure that no higher-order user input can be entered.
Apart from input, all other higher order features are unrestricted.

The following subsections describe in detail how all elements of the \TOPHAT\ language deal with the addition of symbols.



\subsection{Expressions, values, and types}
\label{sec:expressions}

The syntax of \STOPHAT\ is listed in \cref{fig:syntaxtophat}.
Two main changes have been made with regards to the original \TOPHAT\ grammar.
The differences with the original syntax are highlighted in grey boxes.
First, symbols $s$ have been added to the syntax of expressions.
However, they are not intended to be used by programmers, similar to locations $l$.
Instead, they are generated by the semantics as placeholders for symbolic inputs.
Second, unary and binary operations have been made explicit.

\begin{figure}[ht]
  \small
  \usemacro{G-Language-Compact}
  \usemacro{G-Pretasks-Compact}
  \caption{Syntax of Symbolic \TOPHAT\ expressions.}
  \label{fig:syntaxtophat}
\end{figure}

Symbols are treated as values (\cref{fig:syntaxvalues}).
They have therefore been added to the grammar of values.
Also, every symbol has a type, and basic operations can take symbols as arguments.
As a result, we must now also regard unary and binary operations as values.
Therefore we make these operations explicit in this language description,
where in the original they were left implicit.

\begin{figure}[ht]
  \small
  \usemacro{G-Values-Compact}
  \usemacro{G-Tasks-Compact}
  \caption{Syntax of values in Symbolic \TOPHAT.}
  \label{fig:syntaxvalues}
\end{figure}

The types of \STOPHAT\ remain the same (\cref{fig:syntaxtypes}).
However, we do need an additional typing rule, \refrule{T-Sym} in \cref{fig:typingsymbol}, to type symbols,
since they are now part of our expression syntax.
The type of symbols is kept track of in the environment $\Gamma$.

\begin{figure}[t]
  \small
  \usemacro{G-Types-Compact}
  \caption{Syntax of Symbolic \TOPHAT\ types.}
  \label{fig:syntaxtypes}
\end{figure}

\begin{figure}[t]
  \small
  \highlight{\userule{T-Sym}}
  \caption{Additional typing rule for Symbolic \TOPHAT.}
  \label{fig:typingsymbol}
\end{figure}



\subsection{Inputs}

In symbolic execution, we do not know what the input of a program will be.
In our case this means that we do not know which events will be sent to editors.
This is reflected in the definition of symbolic inputs and actions in \cref{fig:syntaxinputs}

\begin{figure}[ht]
  \small
  \usemacro{G-Inputs-Compact}
  \caption{Syntax of inputs and actions in Symbolic \TOPHAT.}
  \label{fig:syntaxinputs}
\end{figure}

Inputs are still the same and consist of paths and actions.
Paths are tagged with one or more $\First$ (first) and $\Second$ (second) tags.
Actions no longer contain concrete values, but only symbols.
This means that instead of concrete values, editors can only hold symbols.



\subsection{Path conditions}

Concrete execution of \TOPHAT\ programs is driven by concrete inputs, which select one branch of conditionals, or make a choice.
Since no concrete information is available during symbolic execution, the symbolic execution semantics records how each execution path depends on the symbolic input.
This is done by means of path conditions.
\Cref{fig:syntaxpredicates} lists the syntax of path conditions.

\begin{figure}[ht]
  \small
  \usemacro{G-Predicates-Compact}
  \caption{Syntax of path conditions.}
  \label{fig:syntaxpredicates}
\end{figure}

Path conditions are a subset of the values of basic type $\beta$.
They can contain symbols, constants, pairs, lists, and operations on them.

% !TEX root=../main.tex

\section{Semantics}
\label{sec:semantics}

In this section we discuss the symbolic execution semantics for \TOPHAT.
The structure of the symbolic semantics closely resembles that of the concrete semantics.
It consists of three layers, a big step symbolic evaluation semantics for the host language, a big step symbolic normalisation semantics for the task language, and a small step driving semantics that processes user inputs.
\Cref{fig:semantic-functions} gives an overview of the relations between the different semantics.

\begin{figure}[h]
  \centering
  \includegraphics[width=\columnwidth,page=5]{figures/drawings-crop.pdf}
  \caption{
    Semantic functions defined in this report and their relation.
  }
  \label{fig:semantic-functions}
\end{figure}

They are described in the following sections.
We will study their interesting aspects, and the changes made with respect to the concrete semantics.



\subsection{Symbolic evaluation}

The host language is a simply typed lambda calculus with references and basic operations.
Most of the symbolic evaluation rules closely resemble the concrete semantics.
The original evaluation relation ($\hat{\eval}$) had the form $\RelationE$,
where an expression $e$ in a state $\hat{\sigma}$ evaluates to a value $\hat{v}$ in state $\hat{\sigma}'$.
The new relation ($\eval$) adds path conditions $\phi$ to the output and has the form $\RelationSE$.
The hat distinguishes the old concrete and the new symbolic variants.``

The symbolic semantics can generate multiple outcomes.
This is denoted in the evaluation with a line over the result, which can be read as $\overline{v,\sigma',\phi} = \{(v_1,\sigma'_1,\phi_1),\cdots,(v_n,\sigma'_n,\phi_n)\}$.
The set that results from symbolic execution can be interpreted as follows.
Each element is a possible endpoint in the execution of a task.
It is guarded by a condition $\phi$ over the symbolic input.
Execution only arrives at the symbolic value $v$ and symbolic state $\sigma'$ when the path condition $\phi$ is satisfied.

To illustrate the difference between concrete and symbolic evaluation, \cref{fig:oldToNewSemantics} lists one rule from the concrete semantics and its corresponding symbolic counterpart.

\begin{figure}[ht]
  \small
  \begin{gather*}
    \userule{E-Edit}\Quad
    \userule{SE-Edit}
  \end{gather*}
  \caption{The evaluation rule from the concrete and the symbolic semantics for the editor expression.}
  \label{fig:oldToNewSemantics}
\end{figure}

The \refrule{E-Edit} rule evaluates the expression held in an editor to a value.
The \refrule{SE-Edit} does the same, but since it is concerned with symbolic execution, the expression can contain symbols.
We therefore do not know beforehand which concrete value will be produced, or even which path the execution will take.
If the expression contains a conditional that depends on a symbol, there can be multiple possible result values.

% Figure~\ref{fig:eval} lists the rule for conditional.
% From this rule, one can clearly see that both branches are calculated, since at this point we do not know, what the condition will evaluate to.

Most symbolic rules closely resemble their concrete counterparts, and follow directly from them.
The rules are not listed here, a full overview can be found in \cref{sec:symbolic-evaluation-rules}.

The only interesting rule is the one for conditionals, listed in \cref{fig:eval}.
\begin{figure}[ht]
  \small
  \begin{gather*}
    \boxed{\RelationSE} \Break
    \userule{SE-If}
  \end{gather*}
  \caption{Part of the symbolic evaluation semantics.}
  \label{fig:eval}
\end{figure}
The concrete semantics has two separate rules for the $\THEN$ and the $\ELSE$ branch.
The symbolic semantics has one combined rule \refrule{SE-If}.
Since $e_1$ can contain symbols, it can evaluate to multiple values.
The rule keeps track of all options.
It calculates the $\THEN$-branch, and records in the path condition that execution can only reach this branch if $v_1$ becomes $\True$.
The rule does the same for the $\ELSE$-branch, except it requires that $v_1$ becomes $\False$.
Note that both $e_2$ and $e_3$ are evaluated using the same state $\sigma'$,
which is the resulting state after evaluating $e_1$.



\subsection{Observations}
\label{subsec:observations}

The symbolic normalisation and driving semantics make use of observations on tasks, just like the concrete semantics.

The partial function $\Value$ can be used to observe the value of a task.
Its definition is given in \cref{fig:value}.
It is unchanged with respect to the original.

\begin{figure}[ht]
  \small
  \begin{center}
    \usemacro{O-Value}
  \end{center}
  \caption{Task value observation function $\Value$.}
  \label{fig:value}
\end{figure}

\begin{figure}[ht]
  \small
  \begin{center}
    \usemacro{O-Failing}
  \end{center}
  \caption{Task failing observation function $\Failing$.}
  \label{fig:failing}
\end{figure}

The function $\Failing$ observes if a task is failing.
Its definition is given in \cref{fig:failing}.
A task is failing if it is the fail task ($\Fail$), or if it consists of only failing tasks.
This function differs from its concrete counterpart in the clause for user choice.
As symbolic normalisation can yield multiple results, all of the results must be failing to make a user choice failing.



\subsection{Normalisation}

Normalization ($\normalise$) reduces tasks until they are ready to receive input.
Very little has to be changed to accommodate symbolic execution.
Just like the evaluation semantics it now gathers sets of results, each element guarded by a path condition.
\Cref{fig:normalising} lists the normalisation semantics.

\begin{figure}[ht]
  % \begin{minipage}{\textwidth}
    \small
    \begin{gather*}
      \boxed{\RelationSN} \Break
      \userule{SN-Done} \Break
      \userule{SN-Repeat}
    \end{gather*}
  % \end{minipage}
  \caption{Symbolic normalisation semantics.}
  \label{fig:normalising}
\end{figure}

Normalisation makes use of the small step striding semantics ($\stride$).
Its details are not important here.
For more background, we refer to the appendix.



\subsection{Handling}

The handling semantics ($\handle{}$) deals with user input.
In the symbolic case there are symbols instead of concrete inputs.
A complete overview of the rules can be found in \cref{sec:symbolic-handling-rules}.
\Cref{fig:handling} lists the interesting rules of the symbolic handling semantics.

\begin{figure*}[t]
  \begin{minipage}{\textwidth}
    \small
    \begin{gather*}
      \boxed{\RelationSH} \Break
      \userule{SH-Change} \Quad
      \userule{SH-Fill} \Quad
      \userule{SH-Update}\Break
      \userule{SH-PickLeft} \Quad
      \userule{SH-PickRight} \Break
      \userule{SH-Pick} \Quad
      \userule{SH-Next} \Break
      \userule{SH-And}\Quad
      \userule{SH-Or}
    \end{gather*}
  \end{minipage}
  \caption{Symbolic handling semantics.}
  \label{fig:handling}
\end{figure*}

\begin{figure*}[t]
  \begin{function}
    \signature{\Simulate : \mathrm{Tasks} \times \mathrm{States} \times [\mathrm{Inputs}]  \times \mathrm{Predicates}
      \rightarrow \powerset{\mathrm{Values} \times [\mathrm{Inputs}] \times \mathrm{Predicates}}} \\
    \Simulate\ (t, \sigma, I, \phi) &=&
      \bigcup \set{ \Simulate'\ (\True, t, t', \sigma', I\oplus[i'], \phi\land\phi') \mid t, \sigma \drive{} t', \sigma', i', \phi' } \\
    \addlinespace
    \signature{\Simulate' : \mathrm{Booleans} \times \mathrm{Tasks} \times \mathrm{Tasks} \times \mathrm{States} \times [\mathrm{Inputs}] \times \mathrm{Predicates}
      \rightarrow \powerset{\mathrm{Values} \times [\mathrm{Inputs}] \times \mathrm{Predicates}}} \\
    \Simulate'\ (\Again, t, t', \sigma', I, \phi) &=& \\
      \multicolumn{3}{L}{ \left\{
        \begin{array}{lr@{\ }c@{\ }l@{\ }c@{\ }l@{\ }c@{\ }r}
          \nothing                                                                                    & \neg\Sat(\phi) &&&&&&\\
          \set{(v, I, \phi)}                                                                & \Sat(\phi)     &\land& \Value(t',\sigma') = v &&&& \\
          \Simulate\ (t', \sigma', I, \phi)                                                           & \Sat(\phi)     &\land& \Value(t',\sigma') = \bot &\land& t' \neq t &&\\
          \bigcup \set{\Simulate'\ (\False, t', t'', \sigma'', I\oplus[i'], \phi\land\phi')
            \mid  t',\sigma' \drive{} t'', \sigma'', i', \phi'}                                       & \Sat(\phi)     &\land& \Value(t',\sigma') = \bot &\land& t' = t    &\land& \Again\\
          \nothing                                                                                    & \Sat(\phi)     &\land& \Value(t',\sigma') = \bot &\land& t' = t    &\land& \neg\Again
        \end{array}
        \right.}
  \end{function}
  \caption{Simulation function definition.}
  \label{fig:simulate}
\end{figure*}

The three rules for the editors (\refrule{SH-Change}, \refrule{SH-Fill}, \refrule{SH-Update})
clearly show how symbols enter the symbolic execution.
The first one for example generates a fresh symbol $s$ and returns an editor containing it.

There are several task combinators where the result depends on user input.
For example, the parallel combinator ($\And$) receives an input for either the left or the right branch.
To accommodate for all possibilities, the \refrule{SH-And} rule generates both cases.
It tags the inputs for the first branch with $\First$ and inputs for the second branch with $\Second$.

The same principle applies to the external choice combinator ($\Xor$).
The three rules \refrule{SH-PickLeft}, \refrule{SH-PickRight}, and \refrule{SH-Pick} are needed to disallow choosing failing tasks.
There is one rule for the case where only the right is failing, one rule when the left is failing, and one for when none of the options are failing.

After input has been handled, tasks are normalised.
The combination of those two steps is taken care of by the driving ($\drive{}$) semantics, listed in \cref{fig:driving}.

\begin{figure}[ht]
  \small
  \begin{gather*}
    \boxed{\RelationSI} \Break
    \userule{SI-Handle}
  \end{gather*}
  \caption{Symbolic driving semantics.}
  \label{fig:driving}
\end{figure}



\subsection{Simulating}
\label{subsec:driving}

The symbolic driving semantics is a small step semantics.
Every step simulates one symbolic input.
To compute every possible execution, the driving semantics needs to be applied repeatedly, until the task is done.
We define a task to be done when it has an observable value: $\Value(t', \sigma') \neq \bot$.
The simulation function listed in \cref{fig:simulate} is recursively called to produce a list of end states and path conditions.
It accumulates all symbolic inputs and returns for each possible execution the observable task value $v$, the path condition $\phi$, and the state $\sigma$.
We consider a task, state and path condition to be an end state if the task value can be observed,
and the path condition is satisfiable, represented by the function $\Sat$.

The recursion terminates when one of the following conditions is met.

\begin{description}
  \item[$\neg\Sat(\phi)$]
    When the path condition cannot be satisfied, we know that all future steps will not be satisfiable either.
    All future steps will only add more restrictions to the path condition.
    No future path condition will be satisfiable, and we can therefore safely remove it.

  \item[$\Value(t,\sigma)$]
    When the current task has a value it is an end state, which we can return.

  \item[$\Value(t',\sigma')=\bot\land t=t'\land \neg \Again$]
    When the current task does not produce a value, and it is equal to the previous task except from symbol names in editors, the $\Simulate$ function performs one look-ahead step in case the task does proceed when a fresh symbol is entered.
    This one step look-ahead is encoded by the parameter $\Again$.
    When this parameter is set to $\False$, one step look-ahead has been performed and $\Simulate$ does not continue further.
    If the task has a value it is returned, otherwise the branch is pruned.
\end{description}

\begin{figure*}[t]
  \small
\tikzstyle{level 1}=[level distance=4.75cm, sibling distance=3cm]
\tikzstyle{bag} = [text width=4em, text centered]
\tikzstyle{end} = [circle, minimum width=3pt,fill, inner sep=0pt]
\hspace*{-1em}
\begin{tikzpicture}[grow=right, sloped]
\node[bag] {$i=s_0$}
    child[missing] {node {}}
    child {
        node[bag] {$i'=s_1$}
            child[missing] {node {}}
            child {
                node[bag] {$i''=s_2$}
                child[missing] {node {}}
                child {
                    node[end, label=right:
                        {$\nothing$}] {}
                    edge from parent
                    node[above] {$\phi''=s_2\leq 0$}
                    node[below] {\begin{tabular}{R@{\ }C@{\ }L}
                      \Value(\Edit s_0\Then\ldots,\sigma) &=& \bot \\
                      \Edit s_0 \Then\ldots               &=& \Edit s_1\Then\ldots \\
                      \Again                              &=& \False
                    \end{tabular}}
                }
                child {
                    node[end, label=right:
                        % {\begin{tabular}{l} $s_2$\\ $[s_0,s_1,s_2]$\\ $s_0\leq 0$\\$\quad\land\ s_1\leq 0$\\$\quad\land\ s_2>0$ \end{tabular}}] {}
                        {\begin{tabular}{l} $s_2$\\ $[s_0,s_1,s_2]$\\ $s_0\leq 0\land s_1\leq 0\land s_2>0$ \end{tabular}}] {}
                    edge from parent
                    node[above] {$\phi''=s_2>0$}
                    node[below]  {$\Value(\Edit s_2,\sigma)=s_2$}
                }
                edge from parent
                node[above] {$\phi'=s_1\leq0$}
                node[below] {\begin{tabular}{R@{\ }C@{\ }L}
                  \Value(\Edit s_0\Then\ldots,\sigma) &=& \bot \\
                  \Edit s_0 \Then\ldots               &=& \Edit s_1\Then\ldots \\
                  \Again                              &=& \True
                \end{tabular}}
            }
            child {
                node[end, label=right:
                    {\begin{tabular}{l} $s_1$\\ $[s_0,s_1]$\\ $s_0\leq 0\land s_1>0$\end{tabular}}] {}
                edge from parent
                node[above] {$\phi'=s_1>0$}
                node[below]  {$\Value(\Edit s_1,\sigma)=s_1$}
            }
            edge from parent
            node[above] {$\phi=s_0\leq 0$}
            node[below] {\begin{tabular}{R@{\ }C@{\ }L}
              \Value(\Edit s_0\Then\ldots,\sigma) &=   & \bot \\
              \Enter \Int \Then\ldots             &\neq& \Edit s_0\Then\ldots
            \end{tabular}}
    }
    child {
        node[end, label=right:
            {\begin{tabular}{l} $s_0$\\ $[s_0]$\\ $s_0>0$\end{tabular}}] {}
        edge from parent
            node[above] {$\phi=s_0>0$}
            node[below]  {$\Value(\Edit s_0,\sigma)=s_0$}
    };
\end{tikzpicture}
\caption{Application of the simulation function to \cref{lst:abs}.}
\label{diagram:simapp}
\end{figure*}

To better illustrate how the $\Simulate$ function works, we study how it simulates \cref{lst:abs}.
\Cref{diagram:simapp} gives a schematic overview of the application of $\Simulate$.
First, it calls the drive semantics to calculate what input the task takes.
Users can enter a fresh symbol $s_0$, as listed on the left.
The symbolic execution then branches, since it reaches a conditional.
Two cases are generated. Either $s_0>0$, the upper branch, or $s_0\leq0$, the  branch to the right.
In the first case, the resulting task has a value, and the symbolic execution ends returning that value and the input.
In the second case, the resulting task does not have a value, and the new task is different from the previous task.
Therefore, it recurses, and $\Simulate$ is called again.

A fresh symbol $s_1$ is generated.
Again, $s_1$ can either be greater than zero, or less or equal.
In the first case, the resulting task has a value, and the execution ends.
In the second case however, the task does not have a value, and we find that the task has not been altered (apart from the new symbol).
This results in a recursive call to $\Simulate'$ with $\Again$ set to $\False$.

Once more a fresh symbol $s_2$ is generated, and $s_2$ can be greater than zero, or less or equal.
In the first case, the task has a value and we are done.
In the second case, it does not have a value, the task again has not changed, but $\Again$ is $\False$ and therefore symbolic execution prunes this branch.

This example demonstrates a couple of things.
From manual inspection, it is clear that only the first iteration returns an interesting result.
When $s_0$ is greater than zero, the task results in a value that is greater than zero.
When the input is less than or equal to zero, simulation continues with the task unchanged.

Why does the simulation still proceed then?
Since the editor $\Enter$ changes to $\Edit$, the tasks are not the same after the first step.
This causes $\Simulate$ to run an extra iteration.
It finds that the task still does not have a value, but now the task has changed.
Then $\Simulate$ performs one look-ahead step, by setting the $\Again$-parameter to $\False$.
When this look-ahead does not return a value, the branch is pruned.



\subsection{Solving}

To check the satisfiability of path conditions $\Sat(\phi)$, as well as the properties stated about a program,
we make use of an external \SMT~solver.
In the implementation we use \ZTHREE, although any other \SMT~solver supporting \SMTLIB could be used.

For \cref{lst:abs}, we would like to prove that after any input sequence $I$,
the path conditions $\phi$ imply that the value $v$ of the resulting task $t'$ is greater than $0$.
\begin{equation*}
  \phi \implies v  > 0 \Quad \where v = \Value(t',\sigma')
\end{equation*}
As shown in \cref{diagram:simapp}, there are three paths we need to verify.
Therefore, we send the following three statements to the \SMT~solver for verification:
\begin{enumerate}
  \item $s_0 > 0                                   \implies s_0 > 0  $
  \item $s_0 \leq 0 \land s_1 > 0                  \implies s_1 > 0  $
  \item $s_0 \leq 0 \land s_1 \leq 0 \land s_2 > 0 \implies s_2 > 0  $
\end{enumerate}
In this example all are trivially solvable.



\subsection{Implementation}

We implemented our language and its symbolic execution semantics in \HASKELL.\footnote{\url{https://github.com/timjs/symbolic-tophat-haskell}}
With the help of a couple of \GHC extensions, the grammar, typing rules and semantics are almost one-to-one translatable into code.
Our tool generates execution trees like the one shown in \cref{diagram:simapp},
which keep track of intermediate normalisations, symbolic inputs, and path conditions.
All path conditions are converted to \SMTLIB compatible statements and are verified using the \ZTHREE \SMT~solver.
As of now we do not have a parser, programs must be specified directly as abstract syntax trees.

As is usually the case with symbolic execution, the number of paths grows quickly.
The examples in \cref{lst:tax,lst:flight-booking} generate respectively 2112 and 1166 paths,
which takes about a minute to calculate.
Solving them, however, is almost instantaneous.



\subsection{Outlook}
\label{subsec:outlook}

\paragraph{Assertions}

Other work on symbolic execution often uses assertions, which are included in the program itself.
One could imagine an assertion statement \TS{assert $\psi$ t} in \TOPHAT that roughly works as follows.
First the \SMT solver verifies the property $\psi$ against the current path condition.
If the assertion fails, an error message is generated.
Then the program continues with task $t$.

\begin{example}
  Consider the following small example program.
  \begin{TASK}
    enter Int >>= \ x . edit (ref x ) >>= \ l. assert (!l == x) (edit "Done")
  \end{TASK}

  This program asks the user to enter an integer.
  The entered value is then stored in a reference.
  The assertion that follows ensures that the store has been updated correctly.
  Finally the string "Done" is returned.
\end{example}

Assertions have access to all variables in scope, unlike properties as we have currently implemented them.
We can overcome this by returning all values of interest at the end of the program.
\begin{TASK}
  enter Int >>= \ x . edit (ref x ) >>= \ store . edit "Done" >>= \ _ . edit (x,!store)
\end{TASK}
It is now possible to verify that the property $\psi(x,s) = x \equiv s$ holds.
This demonstrates that our approach has expressive power similar to assertions.
Having assertions in our language would be more convenient for programmers however, and we would like add them in the future.



\paragraph{Input-dependent predicates}

Another feature we would like to support in the future are input-dependent predicates.

\begin{example}
  Consider the following small program.

  \begin{TASK}
    enter Int >>= \ x . if x > 0 then edit "Thank you" else edit "Error"
  \end{TASK}

  The user inputs an integer.
  If the integer is larger than zero, the program prints a thank you message.
  If the integer is smaller than zero, an error is returned.
\end{example}

If we want to prove that given a positive input, the program never returns "Error", we need to be able to talk about inputs directly in predicates.
Currently our symbolic execution does not support this.

% !TEX root=../main.tex



\section{Properties}
\label{sec:properties}

In this section we describe what it means for the symbolic execution semantics to be correct.
We prove it sound and complete with respect to the concrete semantics of \TOPHAT.

To relate the two semantics, we use the concrete inputs listed in \cref{fig:inputsConcrete}.

\begin{figure}[h]
  \usemacro{G-CInputs-Compact}
  \caption{Syntax of concrete inputs.}
  \label{fig:inputsConcrete}
\end{figure}


\subsection{Soundness}
\label{sec:soundess}


To validate the symbolic execution semantics,
we want to show that for every individual symbolic execution step there exists a corresponding concrete one.
This soundness property is expressed by \cref{thm:sounddrive}.

\begin{theorem}[Soundness of interact]
  \label{thm:sounddrive}

  For all concrete tasks $t$, concrete states $\sigma$ and mappings $M=[s_0\mapsto c_0,\cdots,s_n\mapsto c_n]$,
    we have for all tuples $(\tilde{t}',\tilde{\sigma}',\simi,\phi)$ in $t,\sigma\siminteract \overline{\tilde{t}',\tilde{\sigma}',\simi,\phi}$ that
    $M\phi$ implies
    $t,\sigma \interact{M \simi} t',\sigma''$ and $M\tilde{t}' \equiv t'$ and $M\tilde{\sigma}' \equiv \sigma''$.
\end{theorem}

The proof for this theorem is rather straightforward.
Since the driving semantics makes use of the handling and the normalisation semantics, we require two lemmas.
One showing that the handling semantics is sound, \cref{lem:soundhandle}, and one showing that the normalisation semantics is sound, \cref{lem:soundnorm}.

\begin{lemma}[Soundness of handling]
  \label{lem:soundhandle}

  For all concrete tasks $t$, concrete states $\sigma$ and mappings $M = [s_0\mapsto c_0,\cdots,s_n\mapsto c_n]$,
    we have for all tuples $(\tilde{t}',\tilde{\sigma}',\simi,\phi)$ in
    $t,\sigma\simhandle \overline{\tilde{t}',\tilde{\sigma}',\simi,\phi}$,
    that $M\phi$ implies
    $t,\sigma \handle{M \simi} t',\sigma'$ and $M\tilde{t}' \equiv t' $ and $M\tilde{\sigma}' \equiv \sigma'$.
\end{lemma}

\Cref{lem:soundhandle} is proven by induction over $t$.
The full proof is listed in \cref{sec:soundness-proofs}.

\begin{lemma}[Soundness of normalisation]
  \label{lem:soundnorm}

  For all concrete expressions $e$, concrete states $\sigma$ and mappings $M=[s_0\mapsto c_0,\cdots,s_n\mapsto c_n]$,
  we have for all tuples $(\tilde{t},\tilde{\sigma}',\phi)$ in
  $e,\sigma\simnormalise \overline{\tilde{t},\tilde{\sigma}',\phi}$,
  that $M\phi$ implies
  $e,\sigma \normalise t',\sigma''$ and $M \tilde{t} \equiv t'$ and $M\tilde{\sigma}' \equiv \sigma''$.

\end{lemma}

Since \cref{lem:soundnorm} makes use of both the striding and the evaluation semantics,
we must show soundness for those too.

\begin{lemma}[Soundness of striding]
  \label{lem:soundstride}
  For all concrete tasks $t$, concrete states $\sigma$ and mappings $M=[s_0\mapsto c_0,\cdots,s_n\mapsto c_n]$,
    we have for all tuples $(\tilde{t}',\tilde{\sigma}',\phi)$
    in $t,\sigma\simstride \overline{\tilde{t}',\tilde{\sigma}',\phi}$,
    that $M \phi$ implies
    $t,\sigma \stride t',\sigma'$ and $M \tilde{t}' \equiv t' \land M\tilde{\sigma}' \equiv \sigma'$.

\end{lemma}

\begin{lemma}[Soundness of evaluation]
  \label{lem:soundeval}

  For all concrete expressions $e$, concrete sta\-tes $\sigma$ and mappings $M=[s_0\mapsto c_0,\cdots,s_n\mapsto c_n]$,
    we have for all tuples $(\tilde{v},\tilde{\sigma}',\phi)$
    in $e,\sigma\simeval \overline{\tilde{v},\tilde{\sigma}',\phi}$,
    that $M\phi$ implies
    $e,\sigma \eval v,\sigma' \land M\tilde{v} \equiv v \wedge M\tilde{\sigma}' \equiv \sigma'$.

\end{lemma}

The full proofs of \cref{lem:soundnorm,lem:soundstride,lem:soundeval} are listed in the appendix.





\subsection{Completeness}

We also want to show that for every concrete execution, a symbolic one exists.

To state this Theorem, we require a simulation relation $i\sim j$, which means that the symbolic input $i$ follows the same direction as the concrete input $j$.
This relation is defined below.

\begin{definition}[Input simulation]
  A symbolic input $\simi$ simulates a concrete input $i$ denoted as $\simi\sim i$ in the following cases.\\
  $s\sim a$, where $s$ is a symbol and $a$ a concrete action.\\
  $\simi\sim i\implies \First \simi \sim \First i$\\
  $\simi\sim i\implies \Second \simi \sim \Second i$
\end{definition}

This allows us to define the completeness property as listed in \cref{thm:completeDrive}.

\begin{theorem}[Completeness of interact]
  \label{thm:completeDrive}
  For all concrete tasks $t$, concrete states $\sigma$ and concrete inputs $i$ such that $t,\sigma \interact{i} t',\sigma'$
there exists an $\simi\sim i$, $\simt$, $\sims$ and $\phi$ such that $(\simt,\sims,\simi,\phi)$ in $t,\sigma\siminteract \overline{\simt,\sims,\simi,\phi}$.
\end{theorem}


The proof of \cref{thm:completeDrive} is rather simple.
We show that handling is complete (\cref{lem:completeHandle})
and that the subsequent normalisation is complete (\cref{lem:completeNormalise}).


\begin{lemma}[Completeness of handling]
  \label{lem:completeHandle}
  For all concrete tasks $t$, concrete states $\sigma$ and concrete inputs $i$ such that $t,\sigma \handle{i} t',\sigma'$
  there exists an $\simi\sim i$, $\simt$, $\sims$ and $\phi$ such that $(\simt,\sims,\simi,\phi)$ in $t,\sigma\simhandle \overline{\simt,\sims,\simi,\phi}$.
\end{lemma}

\Cref{lem:completeHandle} is proved by induction over $t$.
We only need to show that every concrete execution is also a symbolic one.
The only change needed to convert from concrete to symbolic is the adaption of the input.

Since handling makes use of normalisation and evaluation, we need to prove that they too are complete.
These properties are listed in \cref{lem:completeNormalise,lem:completeEval}

\begin{lemma}[Completeness of normalisation]
  \label{lem:completeNormalise}
  For all concrete expressions $e$ and concrete states $\sigma$ such that $e,\sigma\normalise t,\sigma$
  there exists a symbolic execution result $(t,\sigma,\True)$ in $e,\sigma\simnormalise \overline{\simt,\sims,\phi}$.

\end{lemma}

\begin{lemma}[Completeness of evaluation]
  \label{lem:completeEval}
  For all concrete expressions $e$ and concrete states $\sigma$ such that $e,\sigma \eval v, \sigma$
  there exists a symbolic execution result $(v,\sigma,\True)$ in $e,\sigma\simeval \overline{\simv,\sims,\phi}$.

\end{lemma}

\Cref{lem:completeNormalise,lem:completeEval} follow trivially, since every concrete execution in these semantics is also a symbolic one.

% !TEX root=../main.tex



\section{Related work}
\label{sec:relatedwork}

\paragraph{Symbolic execution}
Symbolic execution \cite{King1975,Boyer1975} is typically being applied to imperative programming languages, for example \citet{BucurKC2014} prototype a symbolic execution engine for interpreted imperative languages.
\citet{CadarDE2008} use it to generate test cases for programs that can be compiled to LLVM byte-code.
\citet{JaffarMNS2012} use it for verifying safety properties of C programs.

In recent years it has been used for functional programming languages as well.
To name some examples, there is ongoing work by \citet{HallahanXP2017} and \citet{Xue2019} to implement a symbolic execution engine for Haskell.
\citet{GiantsiosPS2017} use symbolic execution for a mix of concrete and symbolic testing of programs written in a subset of Core Erlang.
Their goal is to find executions that lead to a runtime error, either due to an assertion violation or an unhandled exception.
\citet{ChangKT2018} present a symbolic execution engine for a typed lambda calculus with mutable state where only some language constructs recognize symbolic values.
They claim that their approach is easier to implement than full symbolic execution and simplifies the burden on the solver, while still considering all execution paths.

The difficulty of symbolic execution for functional languages lies in symbolic higher-order values, that is functions as arguments to other functions.
Hallahan et al solve this with a technique called \emph{defunctionalization}, which requires all source code to be present, so that a symbolic function can only be one of the present lambda expressions or function definitions.
Giantosis et al also require all source code to be present, but they only analyze first-order functions.
They can execute higher-order functions, but only with concrete arguments.
Our method also requires closed well-typed terms, so we never execute a higher-order function in isolation.
Furthermore, we currently do not allow functions and tasks as task values.
Together, this means that symbolic values can never be functions.



\paragraph{Contracts}
Another method for guaranteeing correctness of programs are \emph{contracts}.
Contracts refine static types with additional conditions.
They are enforced at runtime.
Contracts were first presented by \citet{DBLP:journals/computer/Meyer92} for the Eiffel programming language.
\citet{FindlerF2002} applied this technique to functional programming by implementing a contract checker for Scheme.
Their contracts are assertions for higher-order programs.
Contracts can be used to specify properties more fine-grained than what a static type system could check.
It is possible, for example, to refine the arguments or return values of functions to numbers in a certain range, to positive numbers or non-empty lists.

\citet{NguyenTH2017} combine contracts and symbolic execution to provide \emph{soft contract checking}.
The two ideas go hand in hand in that contracts aid symbolic execution with a language for specifications and properties for symbolic values, and symbolic execution provides compile-time guarantees and test case generation.
They present a prototype implementation to verify Racket programs.


\paragraph{Axiomatic program verification}
One of the classical methods of proving partial correctness of programs is Hoare's axiomatic approach \cite{Hoare1969}, which is based on pre- and postconditions.
See \citet{NielsonN1992} for a nice introduction to the topic.
The axiomatic approach is usually applied to imperative programs, requires manually stating loop invariants, and manually carrying out proofs.

Some work has been done to bring the axiomatic method to functional programming.
The current state of SMT solving allows for automated extraction and solving of a large amount of proof obligations.
Notable works in this field are for example the Hoare Type Theory by \citet{NanevskiMB2006}, the Hoare and Dijkstra Monads by \citet{NanevskiMSGB08, SwamyWSCL2013}, or the Hoare logic for the state monad by \citet{Swierstra2009}.

The difference between the work cited here and our work is that the axiomatic method focuses on stateful computations, while we try to incorporate input as well.

% !TEX root=../main.tex

\section{Conclusion}

\label{sec:conclusion}

In this paper, we have demonstrated how to apply symbolic execution to \TOPHAT\ to verify individual programs.
We have developed both a formal system and an implementation of a symbolic execution semantics.
Our approach has been validated by proving the formal system correct, and by running the implementation on example programs.
For these two example programs, a subsidy request workflow and a flight booking workflow, we have verified that they adhere to their specifications.


\subsection{Future work}

There are many ways in which we would like to continue this line of work.

First, we believe that more can be done with symbolic execution.
Our current approach only allows proving predicates over task results and input values.
We cannot, however, prove properties that depend on the order of the inputs.
Since the symbolic execution currently returns a list of symbolic inputs, we think this extension is feasible.

Second, our symbolic execution only applies to \TOPHAT.
We would like to see if we can fit it to iTasks.
This poses several challenges.
iTasks does not have a formal semantics in the sense that \TOPHAT\ has.
The current implementation in Clean is the closest thing available to a formal specification.
There are also a few language features in iTasks that are not covered by \TOPHAT, for example loops.

Third, we would like to apply different kinds of analyses altogether.
Can a certain part of the program be reached?
Does a certain property hold at every point in the program?
Are two programs equal? And what does it mean for two programs to be equal?
We think that these properties require a different approach.



%% Acknowledgments
\begin{acks}                            %% acks environment is optional
                                        %% contents suppressed with 'anonymous'
  %% Commands \grantsponsor{<sponsorID>}{<name>}{<url>} and
  %% \grantnum[<url>]{<sponsorID>}{<number>} should be used to
  %% acknowledge financial support and will be used by metadata
  %% extraction tools.
  \small
  \input{sections/acknowledgements}
\end{acks}


%% Bibliography
\bibliography{bibliography}


\end{document}
